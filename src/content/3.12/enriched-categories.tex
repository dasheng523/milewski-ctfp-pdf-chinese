% !TEX root = ../../ctfp-print.tex

\lettrine[lhang=0.17]{一}{个范畴是小的} (small),如果其对象构成一个集合 (set)。但我们知道有些事物比集合更大。著名的是,在标准集合论(策梅洛-弗兰克尔理论 (Zermelo-Fraenkel theory),可选地加上选择公理 (Axiom of Choice))中无法形成所有集合之集。因此,所有集合之范畴 (category of all sets) 必定是大范畴 (large category)。有一些数学技巧,如格罗滕迪克全集 (Grothendieck universes),可用于定义超越集合的聚集 (collections)。这些技巧让我们能够讨论大范畴。

如果任意两个对象之间的态射 (morphisms) 构成一个集合,则该范畴是 \emph{局部小的} (locally small)。如果它们不构成集合,我们就必须重新思考一些定义。特别是,如果我们甚至无法从集合中选取态射,那么复合 (compose) 态射意味着什么?解决方案是通过将作为 $\Set$ 中对象的 hom-集 (hom-sets) 替换为来自某个其他范畴 $\cat{V}$ 的 \emph{对象} 来进行自举。不同之处在于,通常对象没有元素,所以我们不再被允许谈论单个态射。我们必须根据可以对 hom-对象 (hom-objects) 整体执行的操作来定义 \emph{丰} (enriched) 范畴的所有属性。为此,提供 hom-对象的范畴必须具有额外的结构——它必须是一个幺半范畴 (monoidal category)。如果我们将这个幺半范畴称为 $\cat{V}$,我们就可以讨论在 $\cat{V}$ 上丰化的范畴 (category enriched over $\cat{V}$) $\cat{C}$。

除了规模的原因,我们可能还想将 hom-集推广到比单纯的集合具有更多结构的东西。例如,传统范畴没有对象之间距离 (distance) 的概念。两个对象要么通过态射相连,要么不相连。所有连接到给定对象的对象都是它的邻居 (neighbors)。与现实生活不同;在范畴中,朋友的朋友的朋友与我最亲密的朋友一样近。在一个适当丰化的范畴中,我们可以定义对象之间的距离。

还有一个非常实际的原因让我们需要获得一些关于丰范畴的经验,那就是一个非常有用的范畴论知识在线资源,\urlref{https://ncatlab.org/}{nLab},主要是用丰范畴的术语编写的。

\section{为何需要幺半范畴?}

在构造丰范畴时,我们必须记住,当我们用 $\Set$ 替换幺半范畴,用 hom-集替换 hom-对象时,应该能够恢复通常的定义。实现这一点的最佳方法是从通常的定义开始,并不断以无点方式 (point-free manner)——即不命名集合元素——重新表述它们。

让我们从复合 (composition) 的定义开始。通常,它接受一对态射,一个来自 $\cat{C}(b, c)$,一个来自 $\cat{C}(a, b)$,并将它们映射到一个来自 $\cat{C}(a, c)$ 的态射。换句话说,它是一个映射:
\[\cat{C}(b, c)\times{}\cat{C}(a, b) \to \cat{C}(a, c)\]
这是一个集合之间的函数——其中一个集合是两个 hom-集的笛卡尔积 (Cartesian product)。这个公式可以通过用更一般的东西替换笛卡尔积来轻松推广。范畴积 (categorical product) 可以工作,但我们可以更进一步,使用完全一般的张量积 (tensor product)。

接下来是恒等态射 (identity morphisms)。我们可以使用来自单例集 (singleton set) $\cat{1}$ 的函数来定义它们,而不是从 hom-集中挑选单个元素:
\[j_a \Colon \cat{1} \to \cat{C}(a, a)\]
同样,我们可以用终对象 (terminal object) 替换单例集,但我们可以更进一步,用张量积的单位元 (unit) $i$ 替换它。

如你所见,从某个幺半范畴 $\cat{V}$ 中取出的对象是 hom-集替换的良好候选者。

\section{幺半范畴}

我们之前讨论过幺半范畴,但值得重申其定义。幺半范畴定义了一个作为双函子 (bifunctor) 的张量积:
\[\otimes \Colon \cat{V}\times{}\cat{V} \to \cat{V}\]
我们希望张量积是结合的 (associative),但满足直到自然同构 (natural isomorphism) 的结合性就足够了。这个同构称为结合子 (associator)。其分量为:
\[\alpha_{a b c} \Colon (a \otimes b) \otimes c \to a \otimes (b \otimes c)\]
它必须在所有三个参数上都是自然的。

幺半范畴还必须定义一个特殊的单位对象 (unit object) $i$,它作为张量积的单位元;同样,是直到自然同构。这两个同构分别称为左单位子 (left unitor) 和右单位子 (right unitor),它们的分量是:
\begin{align*}
  \lambda_a & \Colon i \otimes a \to a \\
  \rho_a    & \Colon a \otimes i \to a
\end{align*}
结合子和单位子必须满足相干条件 (coherence conditions):

\begin{figure}[H]
  \centering
  \begin{tikzcd}[row sep=large]
    ((a \otimes b) \otimes c) \otimes d
    \arrow[d, "\alpha_{(a \otimes b)cd}"]
    \arrow[rr, "\alpha_{abc} \otimes \id_d"]
    & & (a \otimes (b \otimes c)) \otimes d
    \arrow[d, "\alpha_{a(b \otimes c)d}"] \\
    (a \otimes b) \otimes (c \otimes d)
    \arrow[rd, "\alpha_{ab(c \otimes d)}"]
    & & a \otimes ((b \otimes c) \otimes d)
    \arrow[ld, "\id_a \otimes \alpha_{bcd}"] \\
    & a \otimes (b \otimes (c \otimes d))
  \end{tikzcd}
\end{figure}

\begin{figure}[H]
  \centering
  \begin{tikzcd}[row sep=large]
    (a \otimes i) \otimes b
    \arrow[dr, "\rho_{a} \otimes \id_b"']
    \arrow[rr, "\alpha_{aib}"]
    & & a \otimes (i \otimes b)
    \arrow[dl, "\id_a \otimes \lambda_b"] \\
    & a \otimes b
  \end{tikzcd}
\end{figure}

\noindent
如果存在一个具有分量的自然同构:
\[\gamma_{a b} \Colon a \otimes b \to b \otimes a\]
其“平方为一”:
\[\gamma_{b a} \circ \gamma_{a b} = \idarrow[a \otimes b]\]
并且与幺半结构一致,则称该幺半范畴为 \newterm{对称的} (symmetric)。

关于幺半范畴的一个有趣之处在于,你也许能够将内部 hom (internal hom)(函数对象 (function object))定义为张量积的右伴随 (right adjoint)。你可能记得函数对象或指数对象 (exponential) 的标准定义是通过范畴积的右伴随。对于任何一对对象,都存在这样一个对象的范畴称为笛卡尔闭范畴 (Cartesian closed)。以下是在幺半范畴中定义内部 hom 的伴随关系:
\[\cat{V}(a \otimes b, c) \sim \cat{V}(a, [b, c])\]
遵循 \urlref{http://www.tac.mta.ca/tac/reprints/articles/10/tr10.pdf}{G. M. Kelly} 的文章,我使用记号 ${[}b, c{]}$ 表示内部 hom。这个伴随的余单位 (counit) 是其分量被称为求值态射 (evaluation morphisms) 的自然变换:
\[\varepsilon_{a b} \Colon ([a, b] \otimes a) \to b\]
注意,如果张量积不是对称的,我们可能定义另一个内部 hom,记作 ${[}{[}a, c{]}{]}$,使用以下伴随关系:
\[\cat{V}(a \otimes b, c) \sim \cat{V}(b, [[a, c]])\]
两者都被定义的幺半范畴称为 \newterm{双闭的} (biclosed)。一个非双闭范畴的例子是 $\Set$ 中的自函子范畴,其张量积为函子复合。那就是我们用来定义单子 (monads) 的范畴。

\section{丰范畴}

在幺半范畴 $\cat{V}$ 上丰化的范畴 $\cat{C}$ 用 hom-对象替换了 hom-集。对于 $\cat{C}$ 中的每一对对象 $a$ 和 $b$,我们关联一个 $\cat{V}$ 中的对象 $\cat{C}(a, b)$。我们对 hom-对象使用与 hom-集相同的表示法,但要理解它们不包含态射。另一方面,$\cat{V}$ 是一个具有 hom-集和态射的常规(非丰化)范畴。所以我们并没有完全摆脱集合——我们只是把它们藏在了地毯下。

由于我们不能谈论 $\cat{C}$ 中的单个态射,态射的复合被 $\cat{V}$ 中的一族态射所取代:
\[\circ \Colon \cat{C}(b, c) \otimes \cat{C}(a, b) \to \cat{C}(a, c)\]

\begin{figure}[H]
  \centering
  \includegraphics[width=0.45\textwidth]{images/composition.jpg}
\end{figure}

\noindent
类似地,恒等态射被 $\cat{V}$ 中的一族态射所取代:
\[j_a \Colon i \to \cat{C}(a, a)\]
其中 $i$ 是 $\cat{V}$ 中的张量单位元。

\begin{figure}[H]
  \centering
  \includegraphics[width=0.4\textwidth]{images/id.jpg}
\end{figure}

\noindent
复合的结合性是用 $\cat{V}$ 中的结合子来定义的:

\begin{figure}[H]
  \centering
  \begin{tikzcd}[column sep=large]
    (\cat{C}(c,d) \otimes \cat{C}(b,c)) \otimes \cat{C}(a,b)
    \arrow[r, "\circ\otimes\id"]
    \arrow[dd, "\alpha"]
    & \cat{C}(b,d) \otimes \cat{C}(a,b)
    \arrow[dr, "\circ"] \\
    & & \cat{C}(a,d) \\
    \cat{C}(c,d) \otimes (\cat{C}(b,c) \otimes \cat{C}(a,b))
    \arrow[r, "\id\otimes\circ"]
    & \cat{C}(c,d) \otimes \cat{C}(a,c)
    \arrow[ur, "\circ"]
  \end{tikzcd}
\end{figure}

\noindent
单位元定律同样用单位子来表示:

\begin{figure}[H]
  \centering
  \begin{subfigure}
    \centering
    \begin{tikzcd}[row sep=large]
      \cat{C}(a,b) \otimes i
      \arrow[rr, "\id \otimes j_a"]
      \arrow[dr, "\rho"]
      & & \cat{C}(a,b) \otimes \cat{C}(a,a)
      \arrow[dl, "\circ"] \\
      & \cat{C}(a,b)
    \end{tikzcd}
  \end{subfigure}
  \hspace{1cm}
  \begin{subfigure}
    \centering
    \begin{tikzcd}[row sep=large]
      i \otimes \cat{C}(a,b)
      \arrow[rr, "j_b \otimes \id"]
      \arrow[dr, "\lambda"]
      & & \cat{C}(b,b) \otimes \cat{C}(a,b)
      \arrow[dl, "\circ"] \\
      & \cat{C}(a,b)
    \end{tikzcd}
  \end{subfigure}
\end{figure}

\section{预序}

预序 (preorder) 被定义为一个瘦范畴 (thin category),即其中每个 hom-集要么是空的,要么是单例。我们将非空的集合 $\cat{C}(a, b)$ 解释为 $a$ 小于或等于 $b$ 的证明。这样的范畴可以解释为在一个非常简单的幺半范畴上丰化,该幺半范畴仅包含两个对象,$0$ 和 $1$(有时称为 $\mathit{False}$ 和 $\mathit{True}$)。除了强制性的恒等态射,这个范畴还有一个从 $0$ 到 $1$ 的单个态射,我们称之为 $0 \to 1$。可以在其中建立一个简单的幺半结构,其张量积模拟 $0$ 和 $1$ 的简单算术(即,唯一非零的积是 $1 \otimes 1$)。这个范畴中的单位对象是 $1$。这是一个严格幺半范畴 (strict monoidal category),也就是说,结合子和单位子都是恒等态射。

由于在预序中,hom-集要么是空的要么是单例,我们可以很容易地用我们这个微小范畴中的 hom-对象替换它。丰化预序 $\cat{C}$ 对于任何一对对象 $a$ 和 $b$ 都有一个 hom-对象 $\cat{C}(a, b)$。如果 $a$ 小于或等于 $b$,这个对象是 $1$;否则它是 $0$。

让我们看一下复合。任何两个对象的张量积都是 $0$,除非它们都是 $1$,此时它是 $1$。如果它是 $0$,那么对于复合态射我们有两个选项:它可以是 $\idarrow[0]$ 或 $0 \to 1$。但如果它是 $1$,那么唯一的选项是 $\idarrow[1]$。将其翻译回关系,这表示如果 $a \leqslant b$ 且 $b \leqslant c$,则 $a \leqslant c$,这正是我们需要的传递律 (transitivity law)。

那么恒等呢?它是从 $1$ 到 $\cat{C}(a, a)$ 的一个态射。只有一个从 $1$ 出发的态射,那就是恒等 $\idarrow[1]$,所以 $\cat{C}(a, a)$ 必须是 $1$。这意味着 $a \leqslant a$,这是预序的自反律 (reflexivity law)。因此,如果我们将预序实现为丰范畴,传递性和自反性都会自动强制执行。

\section{度量空间}

一个有趣的例子来自 \urlref{http://www.tac.mta.ca/tac/reprints/articles/1/tr1.pdf}{William Lawvere}。他注意到度量空间 (metric spaces) 可以用丰范畴来定义。度量空间定义了任意两个对象之间的距离。这个距离是一个非负实数。包含无穷大作为一个可能的值是很方便的。如果距离是无穷大,则无法从起始对象到达目标对象。

距离必须满足一些明显的属性。其中之一是对象到自身的距离必须为零。另一个是三角不等式 (triangle inequality):直接距离不大于经过中间站点的距离之和。我们不要求距离是对称的,这起初可能看起来很奇怪,但是正如 Lawvere 解释的那样,你可以想象在一个方向上你是上坡行走,而在另一个方向上你是下坡行走。无论如何,对称性可以稍后作为附加约束施加。

那么如何将度量空间转换成范畴语言呢?我们必须构造一个范畴,其中 hom-对象是距离。请注意,距离不是态射而是 hom-对象。hom-对象怎么能是一个数字呢?除非我们能构造一个幺半范畴 $\cat{V}$,其中这些数字是对象。非负实数(加上无穷大)构成一个全序 (total order),所以它们可以被视为一个瘦范畴。两个这样的数字 $x$ 和 $y$ 之间的态射存在当且仅当 $x \geqslant y$(注意:这与预序定义中传统使用的方向相反)。幺半结构由加法给出,零作为单位对象。换句话说,两个数字的张量积是它们的和。

度量空间是在这样的幺半范畴上丰化的范畴。从对象 $a$ 到 $b$ 的 hom-对象 $\cat{C}(a, b)$ 是一个非负(可能无限)的数,我们称之为从 $a$ 到 $b$ 的距离。让我们看看在这种范畴中我们能得到什么关于恒等和复合的信息。

根据我们的定义,从张量单位元(即数字零)到 hom-对象 $\cat{C}(a, a)$ 的态射是关系:
\[0 \geqslant \cat{C}(a, a)\]
由于 $\cat{C}(a, a)$ 是一个非负数,这个条件告诉我们从 $a$ 到 $a$ 的距离总是零。完成!

现在让我们谈谈复合。我们从两个相邻 hom-对象的张量积开始,$\cat{C}(b, c) \otimes \cat{C}(a, b)$。我们已将张量积定义为两个距离的和。复合是从这个积到 $\cat{C}(a, c)$ 的 $\cat{V}$ 中的一个态射。$\cat{V}$ 中的态射被定义为大于或等于关系。换句话说,从 $a$ 到 $b$ 和从 $b$ 到 $c$ 的距离之和大于或等于从 $a$ 到 $c$ 的距离。但这正是标准的三角不等式。完成!

通过用丰范畴的术语重新表述度量空间,我们“免费”获得了三角不等式和零自距离。

\section{丰函子}

函子的定义涉及态射的映射。在丰化的设置中,我们没有单个态射的概念,所以我们必须批量处理 hom-对象。Hom-对象是幺半范畴 $\cat{V}$ 中的对象,我们有它们之间的态射可供使用。因此,当两个范畴都在同一个幺半范畴 $\cat{V}$ 上丰化时,定义它们之间的丰函子 (enriched functors) 是有意义的。然后我们可以使用 $\cat{V}$ 中的态射来映射两个丰范畴之间的 hom-对象。

两个范畴 $\cat{C}$ 和 $\cat{D}$ 之间的 \newterm{丰函子} (enriched functor) $F$,除了将对象映射到对象之外,还为 $\cat{C}$ 中的每一对对象 $a$ 和 $b$ 分配一个 $\cat{V}$ 中的态射:
\[F_{a b} \Colon \cat{C}(a, b) \to \cat{D}(F a, F b)\]
函子是保持结构的映射。对于常规函子,这意味着保持复合和恒等。在丰化的设置中,保持复合意味着以下图表交换:

\begin{figure}[H]
  \centering
  \begin{tikzcd}[column sep=large, row sep=large]
    \cat{C}(b,c) \otimes \cat{C}(a,b)
    \arrow[r, "\circ"]
    \arrow[d, "F_{bc} \otimes F_{ab}"]
    & \cat{C}(a,c)
    \arrow[d, "F_{ac}"] \\
    \cat{D}(F b, F c) \otimes \cat{D}(F a, F b)
    \arrow[r,  "\circ"]
    & \cat{D}(F a, F c)
  \end{tikzcd}
\end{figure}

\noindent
保持恒等被替换为保持 $\cat{V}$ 中“选择”恒等的态射:

\begin{figure}[H]
  \centering
  \begin{tikzcd}[row sep=large]
    & i \arrow[dl, "j_a"'] \arrow[dr, "j_{F a}"] & \\
    \cat{C}(a,a)
    \arrow[rr, "F_{aa}"]
    & & \cat{D}(F a, F a)
  \end{tikzcd}
\end{figure}

\section{自丰化}

闭对称幺半范畴 (closed symmetric monoidal category) 可以通过用内部 hom 替换 hom-集来进行自丰化 (self-enriched)(参见上面的定义)。为了使其工作,我们必须为内部 hom 定义复合律。换句话说,我们必须实现一个具有以下签名的态射:
\[[b, c] \otimes [a, b] \to [a, c]\]
这与其他任何编程任务没有太大区别,只是在范畴论中,我们通常使用无点实现。我们从指定它应该是其元素的集合开始。在这种情况下,它是 hom-集的成员:
\[\cat{V}([b, c] \otimes [a, b], [a, c])\]
这个 hom-集同构于:
\[\cat{V}(([b, c] \otimes [a, b]) \otimes a, c)\]
我刚刚使用了定义内部 hom ${[}a, c{]}$ 的伴随关系。如果我们能在这个新集合中构建一个态射,伴随关系将指向原始集合中的态射,然后我们可以将其用作复合。我们通过复合几个我们可用的态射来构造这个态射。首先,我们可以使用结合子 $\alpha_{{[}b, c{]}\ {[}a, b{]}\ a}$ 重新组合左侧的表达式:
\[([b, c] \otimes [a, b]) \otimes a \to [b, c] \otimes ([a, b] \otimes a)\]
我们可以接着应用伴随的余单位 $\varepsilon_{a b}$:
\[[b, c] \otimes ([a, b] \otimes a) \to [b, c] \otimes b\]
并再次使用余单位 $\varepsilon_{b c}$ 到达 $c$。因此我们构造了一个态射:
\[\varepsilon_{b c}\ .\ (\idarrow[{[b, c]}] \otimes \varepsilon_{a b})\ .\ \alpha_{[b, c] [a, b] a}\]
它是 hom-集的元素:
\[\cat{V}(([b, c] \otimes [a, b]) \otimes a, c)\]
伴随关系将给出我们正在寻找的复合律。

类似地,恒等:
\[j_a \Colon i \to [a, a]\]
是以下 hom-集的成员:
\[\cat{V}(i, [a, a])\]
通过伴随关系,它同构于:
\[\cat{V}(i \otimes a, a)\]
我们知道这个 hom-集包含左恒等 $\lambda_a$。我们可以将 $j_a$ 定义为它在伴随关系下的像。

自丰化的一个实际例子是范畴 $\Set$,它作为编程语言中类型的原型。我们之前已经看到,关于笛卡尔积,它是一个闭幺半范畴。在 $\Set$ 中,任何两个集合之间的 hom-集本身就是一个集合,因此它是 $\Set$ 中的一个对象。我们知道它同构于指数集 (exponential set),因此外部和内部 hom 是等价的。现在我们也知道,通过自丰化,我们可以使用指数集作为 hom-对象,并用指数对象的笛卡尔积来表示复合。

\section{与 2-范畴的关系}

我在讨论 $\Cat$(小范畴之范畴 (category of (small) categories))的背景下谈到了 2-范畴 ($\cat{2}$-categories)。范畴之间的态射是函子,但还有一个额外的结构:函子之间的自然变换。在 2-范畴中,对象通常称为 0-胞腔 (zero-cells);态射称为 1-胞腔 (1-cells);态射之间的态射称为 2-胞腔 (2-cells)。在 $\Cat$ 中,0-胞腔是范畴,1-胞腔是函子,2-胞腔是自然变换。

但请注意,两个范畴之间的函子也构成一个范畴;因此,在 $\Cat$ 中,我们实际上有一个 \emph{hom-范畴} (hom-category) 而不是 hom-集。事实证明,就像 $\Set$ 可以被视为在 $\Set$ 上丰化的范畴一样,$\Cat$ 可以被视为在 $\Cat$ 上丰化的范畴。更一般地说,就像每个范畴都可以被视为在 $\Set$ 上丰化一样,每个 2-范畴都可以被视为在 $\Cat$ 上丰化。