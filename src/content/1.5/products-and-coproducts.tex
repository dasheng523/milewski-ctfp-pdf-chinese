% !TEX root = ../../ctfp-print.tex

\lettrine[lhang=0.17]{古}{希腊}剧作家欧里庇得斯 (Euripides) 曾说:“人如其友。” (Every man is like the company he is wont to keep.) 我们的关系定义了我们。这一点在 category theory (范畴论) 中体现得尤为真切。如果我们想在范畴中挑选出某个特定的对象,我们只能通过描述它与其他对象(以及自身)的关系模式来做到这一点。这些关系是由 morphisms (态射) 定义的。

范畴论中有一个常见的构造,称为 \newterm{universal construction} (泛构造),用于根据对象的关系来定义对象。一种方法是选择一个模式,一个由对象和态射构成的特定形状,然后在范畴中寻找它的所有出现。如果这是一个足够常见的模式,并且范畴足够大,那么很可能会有很多很多的匹配项。诀窍在于在这些匹配项之间建立某种排序,并挑选出可以被认为是最佳匹配的那一个。

这个过程让人联想到我们进行网络搜索的方式。一个 query (查询) 就像一个模式。一个非常通用的查询会给你带来很高的 \emph{recall} (召回率):大量的匹配项。有些可能相关,有些则不相关。为了消除不相关的匹配项,你会 refine (优化) 你的查询。这会提高它的 \emph{precision} (精确率)。最后,搜索引擎会对匹配项进行 rank (排序),并且,希望你感兴趣的那个结果会排在列表的顶部。

\section{Initial Object}

最简单的形状是单个对象。显然,在一个给定的范畴中,这种形状的实例数量与对象的数量一样多。这可供选择的太多了。我们需要建立某种排序,并尝试找到位于这个层级结构顶端的对象。我们唯一可用的手段是态射。如果你把态射想象成箭头,那么可能存在一个从范畴的一端到另一端的整体箭头净流向。例如,在 ordered categories (有序范畴) 中,比如 partial orders (偏序) 中,情况就是如此。我们可以通过说对象 $a$ 比对象 $b$ “更初始”,如果存在一个从 $a$ 到 $b$ 的箭头(一个态射),来推广对象优先级的概念。然后,我们会将\emph{那个} \newterm{initial object} (始对象) 定义为具有指向所有其他对象的箭头的对象。显然,不能保证这样的对象存在,但这没关系。一个更大的问题是可能存在太多这样的对象:召回率很好,但精确率不足。解决方法是从有序范畴中获得启示——它们在任意两个对象之间最多只允许一个箭头:小于或等于另一个对象的方式只有一种。这引导我们得出始对象的这个定义:

\begin{quote}
  \textbf{始对象} 是指对于范畴中任何对象,都存在唯一一个从始对象出发指向该对象的态射的对象。
\end{quote}

\begin{figure}[H]
  \centering
  \includegraphics[width=0.4\textwidth]{images/initial.jpg}
\end{figure}

\noindent
然而,即使这样也不能保证始对象(如果存在)的唯一性。但它保证了次好的情况:\newterm{up to isomorphism} (同构意义下唯一)。Isomorphisms (同构) 在范畴论中非常重要,所以我稍后会谈到它们。现在,让我们先认同同构意义下的唯一性证明了在始对象的定义中使用定冠词“the (那个)”是合理的。

以下是一些例子:一个 partially ordered set (偏序集)(通常称为 \newterm{poset})中的始对象是它的最小元素。有些偏序集没有始对象——比如所有整数(正数和负数)的集合,以小于或等于关系作为态射。

在集合与函数的范畴中,始对象是 empty set (空集)。记住,空集对应于 Haskell 类型 \code{Void}(C++ 中没有对应的类型),而从 \code{Void} 到任何其他类型的唯一多态函数被称为 \code{absurd}:

\src{snippet01}
正是这一族态射使得 \code{Void} 成为类型范畴中的始对象。

\section{Terminal Object}

让我们继续使用单对象模式,但改变我们对对象进行排序的方式。如果存在一个从 $b$ 到 $a$ 的态射(注意方向的反转),我们将说对象 $a$ 比对象 $b$ “更终”。我们将寻找一个比范畴中任何其他对象都更终的对象。同样,我们将坚持唯一性:

\begin{quote}
  \textbf{Terminal object} (终对象) 是指对于范畴中任何对象,都存在唯一一个从该对象出发指向终对象的态射的对象。
\end{quote}

\begin{figure}[H]
  \centering
  \includegraphics[width=0.4\textwidth]{images/final.jpg}
\end{figure}

\noindent
同样,终对象在同构意义下是唯一的,我稍后会证明这一点。但首先让我们看一些例子。在一个偏序集中,终对象(如果存在)是最大的对象。在集合范畴中,终对象是 singleton (单例集)。我们已经讨论过单例集——它们对应于 C++ 中的 \code{void} 类型和 Haskell 中的 unit type (单元类型) \code{()}。这是一个只有一个值的类型——在 C++ 中是隐式的,在 Haskell 中是显式的,用 \code{()} 表示。我们也已经确定了从任何类型到单元类型存在且仅存在一个纯函数:

\src{snippet02}
因此,终对象的所有条件都满足了。

注意,在这个例子中,唯一性条件至关重要,因为存在其他集合(实际上,除了空集之外的所有集合)也具有来自每个集合的入态射。例如,对于每种类型都定义了一个布尔值函数(一个 predicate (谓词)):

\src{snippet03}
但是 \code{Bool} 不是终对象。对于每种类型(除了 \code{Void},对于 \code{Void} 两个函数都等于 \code{absurd}),至少还存在另一个 \code{Bool} 值函数:

\src{snippet04}
坚持唯一性给了我们恰到好处的精确度,将终对象的定义缩小到仅限于一种类型。

\section{Duality}

你不能不注意到我们定义始对象和终对象的方式之间的对称性。两者之间唯一的区别是态射的方向。事实证明,对于任何范畴 $\cat{C}$,我们可以通过简单地反转所有箭头来定义 \newterm{opposite category} (对偶范畴) $\cat{C}^\mathit{op}$。只要我们同时重新定义组合,对偶范畴就自动满足范畴的所有要求。如果原始态射 $f \Colon a \to b$ 和 $g \Colon b \to c$ 组合成 $h \Colon a \to c$,其中 $h = g \circ f$,那么反转后的态射 $f^\mathit{op} \Colon b \to a$ 和 $g^\mathit{op} \Colon c \to b$ 将组合成 $h^\mathit{op} \Colon c \to a$,其中 $h^\mathit{op} = f^\mathit{op} \circ g^\mathit{op}$。而反转恒等箭头是一个 (双关语警报!no-op,即无操作。) (no-op)。

\newterm{Duality} (对偶性) 是范畴的一个非常重要的属性,因为它使每个在范畴论工作的数学家的生产力加倍。对于你提出的每一个构造,都有其对偶;对于你证明的每一个定理,你都能免费得到另一个。对偶范畴中的构造通常带有前缀“co”,所以你有 products (积) 和 coproducts (余积),monads (单子) 和 comonads (余单子),cones (锥) 和 cocones (余锥),limits (极限) 和 colimits (余极限),等等。不过没有 cocomonads (余余单子),因为反转箭头两次会让我们回到原始状态。

由此可见,终对象就是对偶范畴中的始对象。

\section{Isomorphisms}

作为程序员,我们很清楚定义 equality (相等性) 是一项不平凡的任务。两个对象相等意味着什么?它们必须占据内存中的相同位置(指针相等)吗?还是它们所有组件的值相等就足够了?如果一个复数用实部和虚部表示,另一个用模和角表示,它们相等吗?你可能认为数学家们已经弄清楚了相等的含义,但他们没有。他们面临着同样的问题:多种相互竞争的相等性定义。有 propositional equality (命题相等性),intensional equality (内涵相等性),extensional equality (外延相等性),以及 homotopy type theory (同伦类型论) 中作为路径的相等性。然后还有较弱的同构概念,以及更弱的 equivalence (等价) 概念。

直觉上,isomorphic (同构的) 对象看起来是一样的——它们具有相同的形状。这意味着一个对象的每个部分都通过一一映射对应于另一个对象的某个部分。就我们的工具所能分辨的而言,这两个对象是彼此的完美副本。在数学上,这意味着存在一个从对象 $a$ 到对象 $b$ 的映射,并且存在一个从对象 $b$ 回到对象 $a$ 的映射,并且它们互为 inverse (逆)。在范畴论中,我们用态射代替映射。同构是一个 invertible (可逆的) 态射;或者是一对态射,其中一个是的另一个的逆。

我们根据 composition (组合) 和 identity (恒等) 来理解逆:如果态射 $g$ 和 $f$ 的组合是恒等态射,那么 $g$ 是 $f$ 的逆。这实际上是两个方程,因为组合两个态射有两种方式:

\src{snippet05}
当我说始对象(终对象)在同构意义下是唯一的时,我的意思是任何两个始对象(终对象)都是同构的。这实际上很容易看出。假设我们有两个始对象 $i_{1}$ 和 $i_{2}$。因为 $i_{1}$ 是始对象,所以存在一个从 $i_{1}$ 到 $i_{2}$ 的唯一态射 $f$。同理,因为 $i_{2}$ 是始对象,所以存在一个从 $i_{2}$ 到 $i_{1}$ 的唯一态射 $g$。这两个态射的组合是什么?

\begin{figure}[H]
  \centering
  \includegraphics[width=0.4\textwidth]{images/uniqueness.jpg}
  \caption{此图中的所有态射都是唯一的。}
\end{figure}

\noindent
组合 $g \circ f$ 必须是一个从 $i_{1}$ 到 $i_{1}$ 的态射。但是 $i_{1}$ 是始对象,所以只能有一个从 $i_{1}$ 到 $i_{1}$ 的态射。因为我们处于一个范畴中,我们知道存在一个从 $i_{1}$ 到 $i_{1}$ 的恒等态射,而且只有一个位置,所以它必须是恒等态射。因此 $g \circ f$ 等于恒等。类似地,$f \circ g$ 必须等于恒等,因为只能有一个从 $i_{2}$ 回到 $i_{2}$ 的态射。这证明了 $f$ 和 $g$ 必须互为逆。因此,任何两个始对象都是同构的。

注意,在这个证明中,我们使用了从始对象到其自身的态射的唯一性。没有这一点,我们就无法证明“同构意义下”的部分。但是为什么我们需要 $f$ 和 $g$ 的唯一性呢?因为始对象不仅在同构意义下是唯一的,它在\emph{唯一}同构意义下也是唯一的。原则上,两个对象之间可能存在多个同构,但这里情况并非如此。这种“在唯一同构意义下的唯一性”是所有泛构造的重要属性。

\section{Products}

下一个泛构造是 product (积)。我们知道两个集合的 Cartesian product (笛卡尔积) 是什么:它是 pair (对) 的集合。但是连接积集与其构成集合的模式是什么?如果我们能弄清楚这一点,我们将能够将其推广到其他范畴。

我们所能说的是,存在两个函数,即 projections (投影),从积到每个构成集合。在 Haskell 中,这两个函数被称为 \code{fst} 和 \code{snd},它们分别选择一个对的第一个和第二个组件:

\src{snippet06}

\src{snippet07}
在这里,函数是通过 pattern matching (模式匹配) 其参数来定义的:匹配任何对的模式是 \code{(x, y)},它将其组件提取到变量 \code{x} 和 \code{y} 中。

通过使用 wildcard (通配符),这些定义可以进一步简化:

\src{snippet08}
在 C++ 中,我们会使用模板函数,例如:

\begin{snip}{cpp}
template<class A, class B> A
fst(pair<A, B> const & p) {
    return p.first;
}
\end{snip}
凭借这看似非常有限的知识,让我们尝试在集合范畴中定义一个对象和态射的模式,这将引导我们构造两个集合 $a$ 和 $b$ 的积。这个模式包含一个对象 $c$ 和两个分别连接它到 $a$ 和 $b$ 的态射 $p$ 和 $q$:

\src{snippet09}

\begin{figure}[H]
  \centering
  \includegraphics[width=0.3\textwidth]{images/productpattern.jpg}
\end{figure}

\noindent
所有符合这个模式的 $c$ 都将被视为积的候选者。可能有很多这样的候选者。

\begin{figure}[H]
  \centering
  \includegraphics[width=0.4\textwidth]{images/productcandidates.jpg}
\end{figure}

\noindent
例如,让我们选择两个 Haskell 类型,\code{Int} 和 \code{Bool},作为我们的构成要素,并获取一些它们积的候选者的样本。

这里有一个:\code{Int}。 \code{Int} 能被认为是 \code{Int} 和 \code{Bool} 的积的候选者吗?是的,可以——这是它的投影:

\src{snippet10}
这很蹩脚,但它符合标准。

这里是另一个:\code{(Int, Int, Bool)}。它是一个包含三个元素的 tuple (元组),或者说一个三元组。以下是使其成为合法候选者的两个态射(我们对三元组使用模式匹配):

\src{snippet11}
你可能已经注意到,我们的第一个候选者太小了——它只覆盖了积的 \code{Int} 维度;第二个候选者太大了——它虚假地复制了 \code{Int} 维度。

但是我们还没有探讨泛构造的另一部分:排序。我们希望能够比较我们模式的两个实例。我们想要比较一个候选对象 $c$ 及其两个投影 $p$ 和 $q$ 与另一个候选对象 $c'$ 及其两个投影 $p'$ 和 $q'$。我们想说 $c$ 比 $c'$ “更好”,如果存在一个从 $c'$ 到 $c$ 的态射 $m$——但这太弱了。我们还希望它的投影比 $c'$ 的投影“更好”或“更普适”。这意味着投影 $p'$ 和 $q'$ 可以使用 $m$ 从 $p$ 和 $q$ 中重构出来:

\src{snippet12}

\begin{figure}[H]
  \centering
  \includegraphics[width=0.4\textwidth]{images/productranking.jpg}
\end{figure}

\noindent
看待这些等式的另一种方式是 $m$ \emph{factorizes} (因子分解) 了 $p'$ 和 $q'$。就假装这些等式是在自然数中,并且点是乘法:$m$ 是 $p'$ 和 $q'$ 共享的一个公因子。

为了建立一些直觉,让我向你展示对 \code{(Int, Bool)} 以及两个规范投影 \code{fst} 和 \code{snd} 确实比我之前提出的两个候选者\emph{更好}。

\begin{figure}[H]
  \centering
  \includegraphics[width=0.4\textwidth]{images/not-a-product.jpg}
\end{figure}

\noindent
第一个候选者的映射 \code{m} 是:

\src{snippet13}
确实,两个投影 \code{p} 和 \code{q} 可以被重构为:

\src{snippet14}
第二个例子的 \code{m} 同样是唯一确定的:

\src{snippet15}
我们能够证明 \code{(Int, Bool)} 比这两个候选者中的任何一个都好。让我们看看为什么反过来不成立。我们能找到某个 \code{m'} 来帮助我们从 \code{p} 和 \code{q} 重构 \code{fst} 和 \code{snd} 吗?

\src{snippet16}
在我们的第一个例子中,\code{q} 总是返回 \code{True},而我们知道存在第二个组件为 \code{False} 的对。我们不能从 \code{q} 重构 \code{snd}。

第二个例子不同:我们在运行 \code{p} 或 \code{q} 之后保留了足够的信息,但是因子分解 \code{fst} 和 \code{snd} 的方法不止一种。因为 \code{p} 和 \code{q} 都忽略了三元组的第二个组件,我们的 \code{m'} 可以在其中放入任何东西。我们可以有:

\src{snippet17}

或者
\src{snippet18}
等等。

总而言之,给定任何类型 \code{c} 及其两个投影 \code{p} 和 \code{q},存在一个从 \code{c} 到笛卡尔积 \code{(a, b)} 的唯一 \code{m} 来因子分解它们。实际上,它只是将 \code{p} 和 \code{q} 组合成一个对。

\src{snippet19}
这使得笛卡尔积 \code{(a, b)} 成为我们的最佳匹配,这意味着这个泛构造在集合范畴中是有效的。它选取了任意两个集合的积。

现在让我们忘记集合,并使用相同的泛构造来定义任何范畴中两个对象的积。这样的积并不总是存在,但当它存在时,它在唯一同构意义下是唯一的。

\begin{quote}
  两个对象 $a$ 和 $b$ 的\textbf{积}是对象 $c$,它配备了两个投影,使得对于任何其他配备了两个投影的对象 $c'$,都存在一个从 $c'$ 到 $c$ 的唯一态射 $m$,该态射因子分解了那些投影。
\end{quote}

\noindent
一个从两个候选者产生因子分解函数 \code{m} 的(高阶)函数有时被称为 \newterm{factorizer} (分解映射)。在我们的例子中,它将是函数:

\src{snippet20}

\section{Coproduct}

像范畴论中的每个构造一样,积有一个对偶,称为 coproduct (余积)。当我们在积模式中反转箭头时,我们最终得到一个对象 $c$,它配备了两个 \emph{injections} (单射;这里更自然的翻译是“注入”),即 \code{i} 和 \code{j}:从 $a$ 和 $b$ 到 $c$ 的态射。

\src{snippet21}

\begin{figure}[H]
  \centering
  \includegraphics[width=0.4\textwidth]{images/coproductpattern.jpg}
\end{figure}

\noindent
排序也被反转了:如果存在一个从 $c$ 到 $c'$ 的态射 $m$ 因子分解了注入 $i'$ 和 $j'$,那么对象 $c$ 比配备了注入 $i'$ 和 $j'$ 的对象 $c'$ “更好”:

\src{snippet22}

\begin{figure}[H]
  \centering
  \includegraphics[width=0.4\textwidth]{images/coproductranking.jpg}
\end{figure}

\noindent
“最好的”这样的对象,即具有连接它到任何其他模式的唯一态射的对象,被称为余积,并且如果它存在,它在唯一同构意义下是唯一的。

\begin{quote}
  两个对象 $a$ 和 $b$ 的\textbf{余积}是对象 $c$,它配备了两个注入,使得对于任何其他配备了两个注入的对象 $c'$,都存在一个从 $c$ 到 $c'$ 的唯一态射 $m$,该态射因子分解了那些注入。
\end{quote}

\noindent
在集合范畴中,余积是两个集合的 \emph{disjoint union} (不交并)。$a$ 和 $b$ 的不交并的一个元素要么是 $a$ 的元素,要么是 $b$ 的元素。如果两个集合有重叠,不交并包含公共部分的两份副本。你可以把不交并的一个元素想象成被标记了一个标识符,指明它的来源。

对于程序员来说,从类型的角度理解余积更容易:它是两种类型的 \newterm{tagged union} (带标签联合)。C++ 支持联合 (unions),但它们没有标签。这意味着在你的程序中,你必须以某种方式跟踪联合中的哪个成员是有效的。要创建一个带标签联合,你必须定义一个标签——一个 enumeration (枚举)——并将它与联合组合起来。例如,一个 \code{int} 和一个 \code{char const *} 的带标签联合可以实现为:

\begin{snip}{cpp}
struct Contact {
    enum { isPhone, isEmail } tag;
    union { int phoneNum; char const * emailAddr; };
};
\end{snip}
这两个注入既可以作为 constructors (构造函数) 实现,也可以作为函数实现。例如,这里是将第一个注入作为函数 \code{PhoneNum} 实现:

\begin{snip}{cpp}
Contact PhoneNum(int n) {
    Contact c;
    c.tag = isPhone;
    c.phoneNum = n;
    return c;
}
\end{snip}
它将一个整数注入到 \code{Contact} 中。

带标签联合也称为 \newterm{variant} (变体),在 boost 库中有一个非常通用的变体实现,即 \code{boost::variant}。

在 Haskell 中,你可以通过用竖线分隔数据构造函数来将任何数据类型组合成带标签联合。\code{Contact} 的例子转化为声明:

\src{snippet23}
在这里,\code{PhoneNum} 和 \code{EmailAddr} 既作为构造函数(注入),又作为模式匹配的标签(稍后会详细介绍)。例如,这是你如何使用电话号码构造一个联系人:

\src{snippet24}
与内置于 Haskell 作为基本对的积的规范实现不同,余积的规范实现是一个名为 \code{Either} 的数据类型,它在标准 Prelude 中定义如下:

\src{snippet25}
它由两个类型 \code{a} 和 \code{b} 参数化,并有两个构造函数:\code{Left} 接受一个类型 \code{a} 的值,\code{Right} 接受一个类型 \code{b} 的值。

就像我们为积定义了分解映射一样,我们也可以为余积定义一个。给定一个候选类型 \code{c} 和两个候选注入 \code{i} 和 \code{j},\code{Either} 的分解映射产生分解函数:

\src{snippet26}

\section{Asymmetry}

我们已经看到了两组对偶定义:终对象的定义可以通过反转箭头的方向从始对象的定义得到;类似地,余积的定义可以从积的定义得到。然而,在集合范畴中,始对象与终对象非常不同,余积与积也非常不同。我们稍后会看到,积的行为像乘法,终对象扮演着 1 的角色;而余积的行为更像加法,始对象扮演着 0 的角色。特别是,对于有限集,积的大小是各个集合大小的乘积,而余积的大小是各个集合大小的和。

这表明集合范畴关于箭头反转是不对称的。

注意,虽然空集到任何集合都有唯一的态射(\code{absurd} 函数),但它没有回来的态射。单例集有来自任何集合的唯一入态射,但它\emph{也}有到每个集合(除了空集)的出态射。正如我们之前看到的,这些从终对象出发的出态射在挑选其他集合的元素方面起着非常重要的作用(空集没有元素,所以没有什么可挑选的)。

正是单例集与积的关系使它区别于余积。考虑使用由单元类型 \code{()} 表示的单例集,作为积模式的另一个——远为逊色的——候选者。为它配备两个投影 \code{p} 和 \code{q}:从单例集到每个构成集合的函数。每个投影都从任一集合中选择一个具体的元素。因为积是泛的,所以也存在一个从我们的候选者,即单例集,到积的(唯一)态射 \code{m}。这个态射从积集中选择一个元素——它选择一个具体的对。它也因子分解了两个投影:

\src{snippet27}
当作用于单例值 \code{()},即单例集的唯一元素时,这两个方程变成:

\src{snippet28}
由于 \code{m ()} 是被 \code{m} 选中的积的元素,这些方程告诉我们,被 \code{p} 从第一个集合中选中的元素 \code{p ()} 是被 \code{m} 选中的对的第一个组件。类似地,\code{q ()} 等于第二个组件。这与我们对积的元素是来自构成集合的元素的对的理解完全一致。

对于余积没有这样简单的解释。我们可以尝试使用单例集作为余积的候选者,试图从中提取元素,但那样我们将有两个注入进入它,而不是两个投影从中出来。它们不会告诉我们任何关于其源的信息(事实上,我们已经看到它们忽略了输入参数)。从余积到我们的单例集的唯一态射也不会告诉我们什么。从始对象的方向看,集合范畴与从终对象的方向看,看起来非常不同。

这不是集合的内在属性,而是我们用作 $\Set$ 中态射的函数的属性。函数通常是 \newterm{asymmetric} (非对称的)。让我解释一下。

一个函数必须为其 domain set (定义域集合) 的每个元素定义(在编程中,我们称之为 \newterm{total function} (全函数)),但它不必覆盖整个 codomain (到达域)。我们已经看到了一些极端情况:来自单例集的函数——只在到达域中选择单个元素的函数。(实际上,来自空集的函数是真正的极端。)当定义域的大小远小于到达域的大小时,我们通常认为这样的函数是将定义域嵌入到到达域中。例如,我们可以认为来自单例集的函数将其单个元素嵌入到到达域中。我称它们为 \newterm{embedding functions} (嵌入函数),但数学家更喜欢给相反的情况命名:紧密填充其到达域的函数称为 \newterm{surjective} (满射) 或 \newterm{onto} (满射)。

另一个不对称的来源是函数允许将定义域集合的多个元素映射到到达域的一个元素。它们可以 collapse (塌缩) 它们。极端情况是将整个集合映射到单例集的函数。你已经看到了多态的 \code{unit} 函数就是这样做的。塌缩只能通过组合来加剧。两个塌缩函数的组合比单个函数更具塌缩性。数学家对非塌缩函数有一个名称:他们称之为 \newterm{injective} (单射) 或 \newterm{one-to-one} (单射)。

当然,也有一些函数既不是嵌入也不是塌缩。它们被称为 \newterm{bijections} (双射),它们是真正对称的,因为它们是可逆的。在集合范畴中,同构与双射是相同的。

\section{挑战}

\begin{enumerate}
  \tightlist
  \item
        证明终对象在唯一同构意义下是唯一的。
  \item
        在一个偏序集中,两个对象的积是什么?提示:使用泛构造。
  \item
        在一个偏序集中,两个对象的余积是什么?
  \item
        用你最喜欢的语言(除了 Haskell)实现与 Haskell \code{Either} 等价的泛型类型。
  \item
        证明 \code{Either} 是比配备了两个注入的 \code{int} “更好”的余积:

        \begin{snip}{cpp}
int i(int n) { return n; }
int j(bool b) { return b ? 0: 1; }
\end{snip}

        提示:定义一个函数

        \begin{snip}{cpp}
int m(Either const & e);
\end{snip}

        它因子分解了 \code{i} 和 \code{j}。
  \item
        继续前一个问题:你将如何论证带有两个注入 \code{i} 和 \code{j} 的 \code{int} 不可能比 \code{Either} “更好”?
  \item
        仍然继续:那么这些注入呢?

        \begin{snip}{cpp}
int i(int n) {
    if (n < 0) return n;
    return n + 2;
}

int j(bool b) { return b ? 0: 1; }
\end{snip}
  \item
        为 \code{int} 和 \code{bool} 的余积想出一个较差的候选者,它不能比 \code{Either} 更好,因为它允许多个可接受的从它到 \code{Either} 的态射。
\end{enumerate}

\section{Bibliography}

\begin{enumerate}
  \tightlist
  \item
        The Catsters,
        \urlref{https://www.youtube.com/watch?v=upCSDIO9pjc}{Products and
          Coproducts} 视频。
\end{enumerate}