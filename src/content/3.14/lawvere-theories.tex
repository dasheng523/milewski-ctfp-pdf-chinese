% !TEX root = ../../ctfp-print.tex

\lettrine[lhang=0.17]{我}{意识到我们可能}正在偏离编程,深入到硬核数学领域。但你永远不知道下一次编程领域的重大革命会带来什么,以及理解它可能需要哪种数学知识。现在有一些非常有趣的想法正在流传,比如具有连续时间 (continuous time) 的函数式反应式编程 (functional reactive programming),将 Haskell 的类型系统扩展到依赖类型 (dependent types),或者在编程中探索同伦类型论 (homotopy type theory)。

到目前为止,我一直随意地将类型 (types) 等同于值的 \emph{集合} (sets)。这并不严格正确,因为这种方法没有考虑到这样一个事实:在编程中,我们是 \emph{计算} (compute) 值的,而计算是一个需要时间的过程,在极端情况下,可能不会终止。发散计算 (Divergent computations) 是每个图灵完备 (Turing-complete) 语言的一部分。

还有一些基础性的原因,说明为什么集合论 (set theory) 可能不是计算机科学甚至数学本身的最佳基础。一个很好的类比是,集合论是绑定到特定架构 (architecture) 的汇编语言 (assembly language)。如果你想在不同的架构上运行你的数学,你必须使用更通用的工具。

一种可能性是用空间 (spaces) 代替集合。空间带有更多结构,并且可以在不借助集合的情况下定义。通常与空间相关的一件事是拓扑 (topology),这对于定义连续性 (continuity) 之类的事物是必需的。而处理拓扑的传统方法,你猜对了,是通过集合论。特别是,子集 (subset) 的概念对于拓扑至关重要。毫不奇怪,范畴论者将这个想法推广到了 $\Set$ 以外的范畴。具有恰好合适的属性以替代集合论的范畴类型称为 \newterm{topos}(复数:topoi),除其他外,它提供了一个广义的子集概念。

\section{泛代数}

在不同抽象层次上描述代数 (algebras) 有很多种方法。我们试图找到一种通用语言来描述像幺半群 (monoids)、群 (groups) 或环 (rings) 这样的事物。在最简单的层面上,所有这些构造都定义了集合元素上的 \emph{运算} (operations),以及这些运算必须满足的一些 \emph{定律} (laws)。例如,幺半群可以用一个结合的 (associative) 二元运算来定义。我们还有一个单位元 (unit element) 和单位元定律。但只要稍加想象,我们就可以将单位元变成一个零元运算 (nullary operation)——一个不接受任何参数并返回集合中一个特殊元素的运算。如果我们想讨论群,我们添加一个一元运算符 (unary operator),它接受一个元素并返回其逆元 (inverse)。与之相应地还有左逆元和右逆元定律。环定义了两个二元运算符以及更多定律。等等。

总的来说,代数是由一组具有不同 $n$ 值的 $n$ 元运算 (n-ary operations) 和一组等式恒等式 (equational identities) 定义的。这些恒等式都是全称量化的 (universally quantified)。结合律等式必须对三个元素的所有可能组合都满足,依此类推。

顺便说一句,这排除了域 (fields) 的考虑,原因很简单,零(加法的单位元)关于乘法没有逆元。域的逆元定律不能被全称量化。

如果我们将运算(函数)替换为态射 (morphisms),这个泛代数 (universal algebra) 的定义可以扩展到 $\Set$ 以外的范畴。我们选择一个对象 $a$(称为泛型对象 (generic object))而不是一个集合。一元运算只是 $a$ 的一个自同态 (endomorphism)。但其他元数 (arities)(\newterm{元数 (arity)} 是给定运算的参数数量)呢?二元运算(元数为 2)可以定义为从积 (product) $a\times{}a$ 回到 $a$ 的一个态射。一般的 $n$ 元运算是从 $a$ 的 $n$ 次幂到 $a$ 的一个态射:
\[\alpha_n \Colon a^n \to a\]
零元运算是从终对象 (terminal object)($a$ 的零次幂)出发的态射。因此,为了定义任何代数,我们只需要一个范畴,其对象是一个特殊对象 $a$ 的幂。具体的代数被编码在该范畴的 hom-集 (hom-sets) 中。这简而言之就是 Lawvere 理论。

Lawvere 理论的推导经过许多步骤,所以这里是路线图:

\begin{enumerate}
  \tightlist
  \item
        有限集范畴 (Category of finite sets) $\cat{FinSet}$。
  \item
        其骨架 (skeleton) $\cat{F}$。
  \item
        其对偶范畴 (opposite category) $\Fop$。
  \item
        Lawvere 理论 $\cat{L}$:范畴 $\cat{Law}$ 中的一个对象。
  \item
        Lawvere 范畴的模型 (Model) $M$:范畴\\
        $\cat{Mod}(\cat{L}, \Set)$ 中的一个对象。
\end{enumerate}

\begin{figure}[H]
  \centering
  \includegraphics[width=0.8\textwidth]{images/lawvere1.png}
\end{figure}

\section{Lawvere 理论}

所有 Lawvere 理论共享一个共同的骨架。Lawvere 理论中的所有对象都是仅使用积(实际上只是幂)从一个对象生成的。但是我们如何在一般范畴中定义这些积呢?事实证明,我们可以使用从一个更简单的范畴的映射来定义积。实际上,这个更简单的范畴可能定义的是余积 (coproducts) 而不是积,我们将使用一个 \emph{逆变} (contravariant) 函子将它们嵌入到我们的目标范畴中。逆变函子将余积变成积,将内射 (injections) 变成投影 (projections)。

Lawvere 范畴骨架的自然选择是有限集范畴 $\cat{FinSet}$。它包含空集 (empty set) $\varnothing$、单例集 (singleton set) $1$、二元集 $2$ 等等。这个范畴中的所有对象都可以使用余积从单例集生成(将空集视为零元余积的特例)。例如,二元集是两个单例的和,$2 = 1 + 1$,正如 Haskell 中表示的那样:

\src{snippet01}
然而,尽管很自然地认为只有一个空集,但可能存在许多不同的单例集。特别是,集合 $1 + \varnothing$ 不同于集合 $\varnothing + 1$,也不同于 $1$——即使它们都是同构的。集合范畴中的余积不是结合的。我们可以通过构建一个识别所有同构集合的范畴来弥补这种情况。这样的范畴称为 \newterm{骨架} (skeleton)。换句话说,任何 Lawvere 理论的骨架都是 $\cat{FinSet}$ 的骨架 $\cat{F}$。这个范畴中的对象可以与自然数(包括零)等同,这些自然数对应于 $\cat{FinSet}$ 中元素的计数。余积扮演加法的角色。$\cat{F}$ 中的态射对应于有限集之间的函数。例如,从 $\varnothing$ 到 $n$ 存在唯一的态射(空集是初始对象 (initial object)),没有从 $n$ 到 $\varnothing$ 的态射(除了 $\varnothing \to \varnothing$),从 $1$ 到 $n$ 有 $n$ 个态射(内射),从 $n$ 到 $1$ 有一个态射,等等。这里,$n$ 表示 $\cat{F}$ 中的一个对象,对应于 $\cat{FinSet}$ 中所有通过同构被识别的 $n$ 元集。

使用范畴 $\cat{F}$,我们可以正式地将 \newterm{Lawvere 理论} (Lawvere theory) 定义为一个范畴 $\cat{L}$,配备一个特殊的函子:
\[I_{\cat{L}} \Colon \Fop \to \cat{L}\]
这个函子必须在对象上是双射的 (bijection),并且必须保持有限积 (finite products)($\Fop$ 中的积与 $\cat{F}$ 中的余积相同):
\[I_{\cat{L}} (m\times{}n) = I_{\cat{L}} m\times{}I_{\cat{L}} n\]
你有时可能会看到这个函子被描述为对象上的恒等 (identity-on-objects),这意味着 $\cat{F}$ 和 $\cat{L}$ 中的对象是相同的。因此我们将使用相同的名称来称呼它们——我们将用自然数来表示它们。但请记住,$\cat{F}$ 中的对象与集合不同(它们是同构集合的类)。

$\cat{L}$ 中的 hom-集通常比 $\Fop$ 中的更丰富。它们可能包含除了对应于 $\cat{FinSet}$ 中函数的态射之外的其他态射(后者有时称为 \newterm{基本积运算 (basic product operations)})。Lawvere 理论的等式定律被编码在那些态射中。

关键的观察是,$\cat{F}$ 中的单例集 $1$ 被映射到 $\cat{L}$ 中我们也称为 $1$ 的某个对象,而 $\cat{L}$ 中的所有其他对象自动成为该对象的幂。例如,$\cat{F}$ 中的二元集 $2$ 是余积 $1 + 1$,因此它必须被映射到 $\cat{L}$ 中的积 $1 \times 1$(或 $1^2$)。从这个意义上说,范畴 $\cat{F}$ 的行为类似于 $\cat{L}$ 的对数。

在 $\cat{L}$ 的态射中,我们有那些由函子 $I_{\cat{L}}$ 从 $\cat{F}$ 转移过来的。它们在 $\cat{L}$ 中扮演结构性角色。特别是余积内射 $i_k$ 变成了积投影 $p_k$。一个有用的直觉是将投影:
\[p_k \Colon 1^n \to 1\]
想象成一个 $n$ 变量函数的原型,该函数忽略除第 $k$ 个变量之外的所有变量。相反,$\cat{F}$ 中的常态射 $n \to 1$ 变成了 $\cat{L}$ 中的对角态射 (diagonal morphisms) $1 \to 1^n$。它们对应于变量的复制。

$\cat{L}$ 中有趣的态射是那些定义了除投影之外的 $n$ 元运算的态射。正是这些态射区分了不同的 Lawvere 理论。它们是定义代数的乘法、加法、单位元选择等等。但是为了使 $\cat{L}$ 成为一个完整的范畴,我们还需要复合运算 $n \to m$(或等价地,$1^n \to 1^m$)。由于范畴的简单结构,它们结果是类型为 $n \to 1$ 的更简单态射的积。这是对“返回积的函数是函数的积”这一陈述的推广(或者,正如我们之前看到的,hom-函子是连续的)。

\begin{figure}[H]
  \centering
  \includegraphics[width=0.8\textwidth]{images/lawvere1.png}
  \caption{Lawvere 理论 $\cat{L}$ 基于 $\Fop$,它从中继承了定义积的“无聊”态射。它添加了描述 $n$ 元运算的“有趣”态射(虚线箭头)。}
\end{figure}

Lawvere 理论构成一个范畴 $\cat{Law}$,其中的态射是保持有限积并与函子 $I$ 交换的函子。给定两个这样的理论 $(\cat{L}, I_{\cat{L}})$ 和 $(\cat{L'}, I'_{\cat{L'}})$,它们之间的态射是一个函子 $F \Colon \cat{L} \to \cat{L'}$ 使得:
\begin{gather*}
  F (m \times n) = F m \times F n \\
  F \circ I_{\cat{L}} = I'_{\cat{L'}}
\end{gather*}
Lawvere 理论之间的态射封装了一个理论在另一个理论内部解释的思想。例如,如果我们忽略逆元,群乘法可以解释为幺半群乘法。

Lawvere 范畴最简单的平凡例子是 $\Fop$ 本身(对应于为 $I_{\cat{L}}$ 选择恒等函子)。这个没有任何运算或定律的 Lawvere 理论恰好是 $\cat{Law}$ 中的初始对象。

此时展示一个非平凡的 Lawvere 理论的例子会非常有帮助,但在首先理解模型是什么之前很难解释清楚。

\section{Lawvere 理论的模型}

理解 Lawvere 理论的关键是认识到这样一个理论概括了许多共享相同结构的个体代数。例如,幺半群的 Lawvere 理论描述了作为幺半群的本质。它必须对所有幺半群都有效。一个特定的幺半群成为这样一个理论的模型 (model)。模型被定义为从 Lawvere 理论 $\cat{L}$ 到集合范畴 $\Set$ 的一个函子。(存在将 Lawvere 理论推广到使用其他范畴作为模型的理论,但这里我将只关注 $\Set$。)由于 $\cat{L}$ 的结构在很大程度上依赖于积,我们要求这样的函子保持有限积。因此,$\cat{L}$ 的一个模型,也称为 Lawvere 理论 $\cat{L}$ 上的代数,由一个函子定义:
\begin{gather*}
  M \Colon \cat{L} \to \Set \\
  M (a \times b) \cong M a \times M b
\end{gather*}
注意,我们要求积的保持仅需 \emph{直到同构}。这非常重要,因为严格保持积会排除大多数有趣的理论。

模型对积的保持意味着 $M$ 在 $\Set$ 中的像是由集合 $M 1$($\cat{L}$ 中对象 $1$ 的像)的幂生成的一系列集合。让我们称这个集合为 $a$。(这个集合有时被称为一个 \emph{型 (sort)},这样的代数被称为 \newterm{单型的} (single-sorted)。存在将 Lawvere 理论推广到多型代数 (multi-sorted algebras) 的理论。)特别是,$\cat{L}$ 中的二元运算被映射到函数:
\[a \times a \to a\]
与任何函子一样,$\cat{L}$ 中的多个态射可能会被压缩到 $\Set$ 中的同一个函数。

顺便说一句,所有定律都是全称量化的等式这一事实意味着每个 Lawvere 理论都有一个平凡模型:一个将所有对象映射到单例集,并将所有态射映射到其上恒等函数的常函子。

$\cat{L}$ 中形式为 $m \to n$ 的一般态射被映射到函数:
\[a^m \to a^n\]
如果我们有两个不同的模型 $M$ 和 $N$,它们之间的自然变换是由 $n$ 索引的一族函数:
\[\mu_n \Colon M n \to N n\]
或等价地:
\[\mu_n \Colon a^n \to b^n\]
其中 $b = N 1$。

注意,自然性条件保证了 $n$ 元运算的保持:
\[N f \circ \mu_n = \mu_1 \circ M f\]
其中 $f \Colon n \to 1$ 是 $\cat{L}$ 中的一个 $n$ 元运算。

定义模型的函子构成一个模型范畴 $\cat{Mod}(\cat{L}, \Set)$,其态射为自然变换。

考虑平凡 Lawvere 范畴 $\Fop$ 的一个模型。这样的模型完全由其在 $1$ 处的值 $M 1$ 确定。由于 $M 1$ 可以是任何集合,这些模型的数量与 $\Set$ 中的集合数量一样多。此外,$\cat{Mod}(\Fop, \Set)$ 中的每个态射(函子 $M$ 和 $N$ 之间的自然变换)由其在 $M 1$ 处的分量唯一确定。反之,每个函数 $M 1 \to N 1$ 都导出一个两个模型 $M$ 和 $N$ 之间的自然变换。因此 $\cat{Mod}(\Fop, \Set)$ 等价于 $\Set$。

\section{幺半群理论}

最简单的非平凡 Lawvere 理论例子描述了幺半群的结构。它是一个单一的理论,提炼了所有可能幺半群的结构,因为该理论的模型涵盖了整个幺半群范畴 $\cat{Mon}$。我们已经看到一个 \hyperref[free-monoids]{泛构造},它表明每个幺半群都可以通过识别态射的一个子集从相应的自由幺半群 (free monoid) 获得。因此,单个自由幺半群已经概括了大量的幺半群。然而,存在无限多个自由幺半群。幺半群的 Lawvere 理论 $\cat{L}_{\cat{Mon}}$ 将它们全部组合在一个优雅的构造中。

每个幺半群必须有一个单位元,所以我们必须在 $\cat{L}_{\cat{Mon}}$ 中有一个从 $0$ 到 $1$ 的特殊态射 $\eta$。注意,在 $\cat{F}$ 中不可能有相应的态射。这样的态射会走向相反的方向,从 $1$ 到 $0$,这在 $\cat{FinSet}$ 中将是一个从单例集到空集的函数。不存在这样的函数。

接下来,考虑态射 $2 \to 1$,即 $\cat{L}_{\cat{Mon}}(2, 1)$ 的成员,它必须包含所有二元运算的原型。在 $\cat{Mod}(\cat{L}_{\cat{Mon}}, \Set)$ 中构造模型时,这些态射将被映射到从笛卡尔积 $M 1 \times M 1$ 到 $M 1$ 的函数。换句话说,是两个参数的函数。

问题是:仅使用幺半群运算符可以实现多少个双参数函数?我们称这两个参数为 $a$ 和 $b$。有一个函数忽略两个参数并返回幺半群单位元。然后有两个投影分别返回 $a$ 和 $b$。接下来是返回 $ab$、$ba$、$aa$、$bb$、$aab$ 等等的函数…… 事实上,具有两个参数 $a$ 和 $b$ 的此类函数的数量与具有生成元 $a$ 和 $b$ 的自由幺半群中的元素数量一样多。注意 $\cat{L}_{\cat{Mon}}(2, 1)$ 必须包含所有这些态射,因为其中一个模型是自由幺半群。在自由幺半群中,它们对应于不同的函数。其他模型可能会将 $\cat{L}_{\cat{Mon}}(2, 1)$ 中的多个态射压缩为单个函数,但自由幺半群不会。

如果我们将具有 $n$ 个生成元的自由幺半群表示为 $n^*$,我们可以将 hom-集 $\cat{L}(2, 1)$ 与幺半群范畴 $\cat{Mon}$ 中的 hom-集 $\cat{Mon}(1^*, 2^*)$ 等同起来。通常,我们选择 $\cat{L}_{\cat{Mon}}(m, n)$ 为 $\cat{Mon}(n^*, m^*)$。换句话说,范畴 $\cat{L}_{\cat{Mon}}$ 是自由幺半群范畴的对偶范畴。

幺半群 Lawvere 理论的 \emph{模型} 范畴 $\cat{Mod}(\cat{L}_{\cat{Mon}}, \Set)$ 等价于所有幺半群的范畴 $\cat{Mon}$。

\section{Lawvere 理论与 Monad}

你可能记得,代数理论可以用 monad (单子) 来描述——特别是 \hyperref[algebras-for-monads]{monad 的代数}。因此,Lawvere 理论和 monad 之间存在联系也就不足为奇了。

首先,让我们看看 Lawvere 理论如何导出一个 monad。它是通过遗忘函子 (forgetful functor) 和自由函子 (free functor) 之间的 \hyperref[free-forgetful-adjunctions]{伴随关系} 来实现的。遗忘函子 $U$ 为每个模型分配一个集合。这个集合是通过在 $\cat{L}$ 中的对象 $1$ 处求值来自 $\cat{Mod}(\cat{L}, \Set)$ 的函子 $M$ 得到的。

推导 $U$ 的另一种方法是利用 $\Fop$ 是 $\cat{Law}$ 中的初始对象这一事实。这意味着,对于任何 Lawvere 理论 $\cat{L}$,都存在唯一的函子 $\Fop \to \cat{L}$。这个函子在模型上导出相反方向的函子(因为模型是 \emph{从} 理论到集合的函子):
\[\cat{Mod}(\cat{L}, \Set) \to \cat{Mod}(\Fop, \Set)\]
但是,正如我们讨论过的,$\Fop$ 的模型范畴等价于 $\Set$,所以我们得到遗忘函子:
\[U \Colon \cat{Mod}(\cat{L}, \Set) \to \Set\]
可以证明,如此定义的 $U$ 总是有左伴随,即自由函子 $F$。

这对于有限集很容易看出。自由函子 $F$ 产生自由代数 (free algebras)。自由代数是 $\cat{Mod}(\cat{L}, \Set)$ 中的一个特定模型,它由有限生成元集合 $n$ 生成。我们可以将 $F$ 实现为可表示函子 (representable functor):
\[\cat{L}(n, -) \Colon \cat{L} \to \Set\]
为了证明它确实是自由的,我们只需证明它是遗忘函子的左伴随:
\[\cat{Mod}(\cat{L}, \Set)(\cat{L}(n, -), M) \cong \Set(n, U(M))\]
让我们简化右侧:
\[\Set(n, U(M)) \cong \Set(n, M 1) \cong (M 1)^n \cong M n\]
(我使用了态射集合同构于指数对象的事实,在这种情况下,它只是迭代积。)伴随关系是 Yoneda 引理的结果:
\[[\cat{L}, \Set](\cat{L}(n, -), M) \cong M n\]
遗忘函子和自由函子共同定义了 $\Set$ 上的一个 \hyperref[monads-categorically]{monad} $T = U \circ F$。因此,每个 Lawvere 理论都生成一个 monad。

事实证明,\hyperref[algebras-for-monads]{这个 monad 的代数} 范畴等价于模型范畴。

你可能记得 monad 代数定义了求值使用 monad 形成的表达式的方法。Lawvere 理论定义了可用于生成表达式的 n 元运算。模型提供了求值这些表达式的方法。

然而,monad 和 Lawvere 理论之间的联系并非双向的。只有有限性 monad (finitary monads) 才能导出 Lawvere 理论。有限性 monad 基于有限性函子 (finitary functor)。$\Set$ 上的有限性函子完全由其在有限集上的作用确定。它在任意集合 $a$ 上的作用可以使用以下 coend (余终) 求值:
\[F a = \int^n a^n \times (F n)\]
由于 coend 推广了余积或和,这个公式是幂级数展开 (power series expansion) 的推广。或者我们可以使用函子是广义容器的直觉。在这种情况下,$a$ 的有限性容器可以描述为形状 (shapes) 和内容 (contents) 的和。这里,$F n$ 是存储 $n$ 个元素的形状集合,内容是一个 $n$ 元组元素,它本身是 $a^n$ 的一个元素。例如,列表(作为函子)是有限性的,每个元数对应一个形状。树每个元数有更多形状,等等。

首先,所有从 Lawvere 理论生成的 monad 都是有限性的,并且可以表示为 coend:
\[T_{\cat{L}} a = \int^n a^n \times \cat{L}(n, 1)\]
反之,给定 $\Set$ 上的任何有限性 monad $T$,我们可以构造一个 Lawvere 理论。我们从为 $T$ 构造一个 Kleisli 范畴 (Kleisli category) 开始。你可能记得,Kleisli 范畴中从 $a$ 到 $b$ 的态射由底层范畴中的一个态射给出:
\[a \to T b\]
当限制在有限集时,这变成:
\[m \to T n\]
这个 Kleisli 范畴的对偶范畴 $\cat{Kl}^\mathit{op}_{T}$,限制在有限集上,就是所讨论的 Lawvere 理论。特别是,描述 $\cat{L}$ 中 n 元运算的 hom-集 $\cat{L}(n, 1)$ 由 hom-集 $\cat{Kl}_{T}(1, n)$ 给出。

事实证明,我们在编程中遇到的大多数 monad 都是有限性的,一个显著的例外是续延 monad (continuation monad)。将 Lawvere 理论的概念扩展到有限性运算之外是可能的。

\section{作为 Coend 的 Monad}

让我们更详细地探讨 coend 公式。
\[T_{\cat{L}} a = \int^n a^n \times \cat{L}(n, 1)\]
首先,这个 coend 是在 $\cat{F}$ 中定义的 profunctor $P$ 上取的:
\[P n m = a^n \times \cat{L}(m, 1)\]
这个 profunctor 在第一个参数 $n$ 上是逆变的。考虑它如何提升态射。$\cat{FinSet}$ 中的态射是有限集的映射 $f \Colon m \to n$。这样的映射描述了从一个 $n$ 元集中选择 $m$ 个元素(允许重复)。它可以被提升到 $a$ 的幂的映射,即(注意方向):
\[a^n \to a^m\]
提升只是从一个 $n$ 元组 $(a_1, a_2, \ldots{}, a_n)$ 中选择 $m$ 个元素(可能带有重复)。

\begin{figure}[H]
  \centering
  \includegraphics[width=0.5\textwidth]{images/liftpower.png}
\end{figure}

\noindent
例如,让我们取 $f_k \Colon 1 \to n$——从一个 $n$ 元集中选择第 $k$ 个元素。它提升为一个函数,该函数接受一个 $a$ 的 $n$ 元组并返回第 $k$ 个元素。

或者让我们取 $f \Colon m \to 1$——一个将所有 $m$ 个元素映射到一个元素的常函数。它的提升是一个函数,该函数接受 $a$ 的单个元素并将其复制 $m$ 次:
\[\lambda{}x \to (\underbrace{x, x, \ldots{}, x}_{m})\]
你可能会注意到,所讨论的 profunctor 在第二个参数上是协变的这一点并不明显。hom-函子 $\cat{L}(m, 1)$ 实际上在 $m$ 上是逆变的。然而,我们取的 coend 不是在范畴 $\cat{L}$ 中,而是在范畴 $\cat{F}$ 中。coend 变量 $n$ 遍历有限集(或其骨架)。范畴 $\cat{L}$ 包含 $\cat{F}$ 的对偶,因此 $\cat{F}$ 中的态射 $m \to n$ 是 $\cat{L}$ 中 $\cat{L}(n, m)$ 的成员(嵌入由函子 $I_{\cat{L}}$ 给出)。

让我们检查 $\cat{L}(m, 1)$ 作为从 $\cat{F}$ 到 $\Set$ 的函子的函子性。我们想要提升一个函数 $f \Colon m \to n$,所以我们的目标是实现一个从 $\cat{L}(m, 1)$ 到 $\cat{L}(n, 1)$ 的函数。对应于函数 $f$,在 $\cat{L}$ 中存在一个从 $n$ 到 $m$ 的态射(注意方向)。将这个态射与 $\cat{L}(m, 1)$ 前复合 (Precomposing) 得到 $\cat{L}(n, 1)$ 的一个子集。

\begin{figure}[H]
  \centering
  \begin{tikzcd}[column sep=large]
    \cat{L}(m, 1) \arrow[r] & \cat{L}(n, 1)\\
    {}^m \bullet \arrow[r, "f"'] & \bullet^n
  \end{tikzcd}
\end{figure}

\noindent
注意,通过提升函数 $1 \to n$,我们可以从 $\cat{L}(1, 1)$ 到达 $\cat{L}(n, 1)$。我们稍后会用到这个事实。

逆变函子 $a^n$ 和协变函子 $\cat{L}(m, 1)$ 的积是一个 profunctor $\Fop \times \cat{F} \to \Set$。记住 coend 可以定义为 profunctor 所有对角成员的余积(不相交和),其中某些元素被识别。识别对应于 cowedge (余楔) 条件。

这里,coend 开始时是集合 $a^n \times \cat{L}(n, 1)$ 在所有 $n$ 上的不相交和。识别可以通过将 \hyperref[ends-and-coends]{coend 表示为余等化子 (coequalizer)} 来生成。我们从一个非对角项 $a^n \times \cat{L}(m, 1)$ 开始。为了到达对角线,我们可以将态射 $f \Colon m \to n$ 应用于积的第一个或第二个分量。然后将这两个结果识别。

\begin{figure}[H]
  \centering
  \begin{tikzcd}
    & a^n \times \cat{L}(m, 1)
    \arrow[dl, "\langle f {,} \id \rangle"']
    \arrow[dr, "\langle \id {,} f \rangle"]
    & \\
    a^m \times \cat{L}(m, 1)
    & \scalebox{2.5}[1]{\sim}
    & a^n \times \cat{L}(n, 1) \\
    & f \Colon m \to n &
  \end{tikzcd}
\end{figure}

\noindent
我之前已经展示过,$f \Colon 1 \to n$ 的提升导致以下两个变换:
\[a^n \to a\]
和:
\[\cat{L}(1, 1) \to \cat{L}(n, 1)\]
因此,从 $a^n \times \cat{L}(1, 1)$ 开始,我们可以到达:
\[a \times \cat{L}(1, 1)\]
当我们提升 $\langle f, \id \rangle$ 时,以及:
\[a^n \times \cat{L}(n, 1)\]
当我们提升 $\langle \id, f \rangle$ 时。然而,这并不意味着 $a^n \times \cat{L}(n, 1)$ 的所有元素都可以与 $a \times \cat{L}(1, 1)$ 识别。这是因为并非 $\cat{L}(n, 1)$ 的所有元素都可以从 $\cat{L}(1, 1)$ 到达。记住,我们只能提升来自 $\cat{F}$ 的态射。$\cat{L}$ 中的非平凡 $n$ 元运算不能通过提升态射 $f \Colon 1 \to n$ 来构造。

换句话说,我们只能识别 coend 公式中那些 $\cat{L}(n, 1)$ 可以通过应用基本态射从 $\cat{L}(1, 1)$ 到达的加数 (addends)。它们都等价于 $a \times \cat{L}(1, 1)$。基本态射是那些作为 $\cat{F}$ 中态射的像的态射。

让我们看看 Lawvere 理论最简单的情况,即 $\Fop$ 本身。在这样的理论中,每个 $\cat{L}(n, 1)$ 都可以从 $\cat{L}(1, 1)$ 到达。这是因为 $\cat{L}(1, 1)$ 是一个只包含恒等态射的单例集,而 $\cat{L}(n, 1)$ 只包含对应于 $\cat{F}$ 中内射 $1 \to n$ 的态射,它们 \emph{是} 基本态射。因此,余积中的所有加数都是等价的,我们得到:
\[T a = a \times \cat{L}(1, 1) = a\]
这就是恒等 monad (identity monad)。

\section{副作用的 Lawvere 理论}

由于 monad 和 Lawvere 理论之间存在如此强的联系,很自然会问 Lawvere 理论是否可以用作编程中 monad 的替代方案。monad 的主要问题是它们不能很好地组合。没有通用的方法来构建 monad 变换器 (monad transformers)。Lawvere 理论在这方面具有优势:它们可以使用余积和张量积进行组合。另一方面,只有有限性 monad 可以容易地转换为 Lawvere 理论。这里的例外是续延 monad。这个领域正在进行研究(见参考文献)。

为了让你体验一下 Lawvere 理论如何用于描述副作用 (side effects),我将讨论传统上使用 \code{Maybe} monad 实现的异常 (exceptions) 的简单情况。

\code{Maybe} monad 由具有单个零元运算 $0 \to 1$ 的 Lawvere 理论生成。该理论的一个模型是一个函子,它将 $1$ 映射到某个集合 $a$,并将零元运算映射到一个函数:

\src{snippet02}
我们可以使用 coend 公式恢复 \code{Maybe} monad。让我们考虑添加零元运算对 hom-集 $\cat{L}(n, 1)$ 的影响。除了创建一个新的 $\cat{L}(0, 1)$(这在 $\Fop$ 中不存在),它还向 $\cat{L}(n, 1)$ 添加了新的态射。这些是将类型为 $n \to 0$ 的态射与我们的 $0 \to 1$ 复合的结果。在 coend 公式中,所有这些贡献都被识别为 $a^0 \times \cat{L}(0, 1)$,因为它们可以从:
\[a^n \times \cat{L}(0, 1)\]
通过以两种不同方式提升 $0 \to n$ 得到。

\begin{figure}[H]
  \centering
  \begin{tikzcd}
    & a^n \times \cat{L}(0, 1)
    \arrow[dl, "\langle f {,} \id \rangle"']
    \arrow[dr, "\langle \id {,} f \rangle"]
    & \\
    a^0 \times \cat{L}(0, 1)
    & \scalebox{2.5}[1]{\sim}
    & a^n \times \cat{L}(n, 1) \\
    & f \Colon 0 \to n &
  \end{tikzcd}
\end{figure}

\noindent
Coend 简化为:
\[T_{\cat{L}} a = a^0 + a^1\]
或者,使用 Haskell 表示法:

\src{snippet03}
这等价于:

\src{snippet04}
注意,这个 Lawvere 理论只支持抛出 (raising) 异常,而不支持处理 (handling) 异常。

\section{挑战}

\begin{enumerate}
  \tightlist
  \item
        枚举 $\cat{F}$($\cat{FinSet}$ 的骨架)中 $2$ 和 $3$ 之间的所有态射。
  \item
        证明幺半群 Lawvere 理论的模型范畴等价于列表 monad 的 monad 代数范畴。
  \item
        幺半群的 Lawvere 理论生成列表 monad。证明其二元运算可以使用相应的 Kleisli 箭头生成。
  \item
        \textbf{FinSet} 是 $\Set$ 的一个子范畴,并且存在一个将其嵌入 $\Set$ 的函子。$\Set$ 上的任何函子都可以限制在 $\cat{FinSet}$ 上。证明有限性函子是其自身限制的左 Kan 扩展。
\end{enumerate}

\section{进一步阅读}
\begin{enumerate}
  \tightlist
  \item
        \urlref{http://www.tac.mta.ca/tac/reprints/articles/5/tr5.pdf}{Functorial Semantics of Algebraic Theories} (代数理论的函子语义学),F. William Lawvere
  \item
        \urlref{http://homepages.inf.ed.ac.uk/gdp/publications/Comp_Eff_Monads.pdf}{Notions of computation determine monads} (计算的概念决定了 monad),Gordon Plotkin 和 John Power
\end{enumerate}