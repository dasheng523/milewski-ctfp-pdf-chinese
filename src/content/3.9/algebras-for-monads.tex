% !TEX root = ../../ctfp-print.tex

\lettrine[lhang=0.17]{如}{果我们将} endofunctors (自函子) 解释为定义 expressions (表达式) 的方式,那么 algebras (代数) 让我们能够对它们进行求值,而 monads (单子) 则让我们能够形成和操作它们。通过将代数与单子结合,我们不仅获得了大量的功能,而且还能回答一些有趣的问题。

其中一个问题涉及单子和 adjunctions (伴随) 之间的关系。正如我们所见,每个伴随 \hyperref[monads-categorically]{都定义了一个单子}(以及一个 comonad (余单子))。问题是:每个单子(余单子)都能从伴随导出吗?答案是肯定的。存在一整族产生给定单子的伴随。我将向你展示两个这样的伴随。

\begin{figure}[H]
  \centering
  \includegraphics[width=0.25\textwidth]{images/pigalg.png}
\end{figure}

\noindent
让我们回顾一下定义。Monad 是一个自函子 $m$,配备了两个满足某些 coherence conditions (相干条件) 的 natural transformations (自然变换)。这些变换在 $a$ 上的分量是:
\begin{align*}
  \eta_a & \Colon a \to m\ a         \\
  \mu_a  & \Colon m\ (m\ a) \to m\ a
\end{align*}
同一自函子的代数是选择一个特定的对象——载体 (carrier) $a$——以及态射 (morphism):
\[\mathit{alg} \Colon m\ a \to a\]
首先要注意的是,代数的方向与 $\eta_a$ 相反。直觉上,$\eta_a$ 从类型 $a$ 的值创建一个平凡的表达式。使得代数与 monad 兼容的第一个相干条件确保,使用载体为 $a$ 的代数对该表达式求值,会得到原始值:
\[\mathit{alg} \circ \eta_a = \id_a\]
第二个条件源于这样一个事实:有两种方法可以求值双重嵌套的表达式 $m\ (m\ a)$。我们可以首先应用 $\mu_a$ 来扁平化表达式,然后使用代数的求值器;或者我们可以应用提升后的求值器来求值内部表达式,然后将求值器应用于结果。我们希望这两种策略是等价的:
\[\mathit{alg} \circ \mu_a = \mathit{alg} \circ m\ \mathit{alg}\]
这里,\code{m alg} 是使用函子 (functor) $m$ 提升 $\mathit{alg}$ 所得到的态射。以下交换图描述了这两个条件(我预期后续内容将 $m$ 替换为 $T$):

\begin{figure}[H]
  \centering
  \begin{subfigure}
    \centering
    \begin{tikzcd}[column sep=large, row sep=large]
      a \arrow[rd, equal] \arrow[r, "\eta_a"]
      & Ta \arrow[d, "\mathit{alg}"] \\
      & a
    \end{tikzcd}
  \end{subfigure}
  \hspace{1cm}
  \begin{subfigure}
    \centering
    \begin{tikzcd}[column sep=large, row sep=large]
      T(Ta) \arrow[r, "T\ \mathit{alg}"] \arrow[d, "\mu_a"]
      & Ta \arrow[d, "\mathit{alg}"] \\
      Ta \arrow[r, "\mathit{alg}"]
      & a
    \end{tikzcd}
  \end{subfigure}
\end{figure}

\noindent
我们也可以用 Haskell 表达这些条件:

\src{snippet01}
让我们看一个小例子。列表自函子的代数由某个类型 \code{a} 和一个从 \code{a} 的列表生成 \code{a} 的函数组成。我们可以通过选择元素类型和 accumulator (累加器) 类型都等于 \code{a},使用 \code{foldr} 来表示这个函数:

\src{snippet02}
这个特定的代数由一个双参数函数(我们称之为 \code{f})和一个值 \code{z} 指定。列表函子恰好也是一个 monad,其 \code{return} 将值转换为单元素列表。代数(这里是 \code{foldr f z})在 \code{return} 之后的组合将 \code{x} 映射为:

\src{snippet03}
其中 \code{f} 的作用以中缀表示法写出。如果对于每个 \code{x},以下相干条件都满足,则该代数与 monad 兼容:

\src{snippet04}
如果我们将 \code{f} 视为二元运算符 (binary operator),这个条件告诉我们 \code{z} 是右单位元 (right unit)。

第二个相干条件作用于列表的列表。\code{join} 的作用是连接各个列表。然后我们可以折叠 (fold) 结果列表。另一方面,我们可以先折叠各个列表,然后折叠结果列表。再次,如果我们将 \code{f} 解释为二元运算符,这个条件告诉我们这个二元运算是 associative (结合的)。当 \code{(a, f, z)} 是一个幺半群 (monoid) 时,这些条件肯定满足。

\section{T-代数}

由于数学家更喜欢将他们的 monad 称为 $T$,他们将与 monad 兼容的代数称为 T-代数 (T-algebras)。给定范畴 $\cat{C}$ 中的 monad $T$ 的 T-代数构成一个 category (范畴),称为 Eilenberg-Moore 范畴 (Eilenberg-Moore category),通常表示为 $\cat{C}^T$。该范畴中的态射是代数同态 (homomorphisms of algebras)。这些与我们之前为 F-代数 (F-algebras) 定义的同态 (homomorphisms) 相同。

T-代数是一个由载体对象和求值器组成的对 $(a, f)$。存在一个明显从 $\cat{C}^T$ 到 $\cat{C}$ 的遗忘函子 (forgetful functor) $U^T$,它将 $(a, f)$ 映射到 $a$。它还将 T-代数的同态映射到 $\cat{C}$ 中载体对象之间的相应态射。你可能还记得我们讨论伴随时提到,遗忘函子的左伴随 (left adjoint) 称为自由函子 (free functor)。

$U^T$ 的左伴随称为 $F^T$。它将 $\cat{C}$ 中的对象 $a$ 映射到 $\cat{C}^T$ 中的自由代数 (free algebra)。这个自由代数的载体是 $T a$。它的求值器是从 $T (T a)$ 回到 $T a$ 的态射。由于 $T$ 是一个 monad,我们可以使用 monad 的 $\mu_a$(Haskell 中的 \code{join})作为求值器。

我们仍然需要证明这是一个 T-代数。为此,必须满足两个相干条件:
\begin{align*}
  \mathit{alg} & \circ \eta_{Ta} = \id_{Ta}     \\
  \mathit{alg} & \circ \mu_a = \mathit{alg} \circ T\ \mathit{alg}
\end{align*}
但如果你将 $\mu$ 代入代数,这些正是 monad 定律。

你可能记得,每个伴随都定义了一个 monad。事实证明,$F^T$ 和 $U^T$ 之间的伴随恰好定义了用于构造 Eilenberg-Moore 范畴的那个 monad $T$。由于我们可以对每个 monad 执行此构造,我们得出结论:每个 monad 都可以从伴随生成。稍后我将向你展示还有另一个伴随生成相同的 monad。

计划如下:首先,我将证明 $F^T$ 确实是 $U^T$ 的左伴随。我将通过定义此伴随的单位 (unit) 和余单位 (counit) 并证明相应的三角恒等式 (triangular identities) 满足来实现。然后,我将证明由此伴随生成的 monad 确实是我们最初的 monad。

伴随的单位是自然变换:
\[\eta \Colon I \to U^T \circ F^T\]
让我们计算这个变换的 $a$ 分量。恒等函子 (identity functor) 给我们 $a$。自由函子产生自由代数 $(T a, \mu_a)$,遗忘函子将其简化为 $T a$。总的来说,我们得到一个从 $a$ 到 $T a$ 的映射。我们将简单地使用 monad $T$ 的单位作为这个伴随的单位。

让我们看看余单位:
\[\varepsilon \Colon F^T \circ U^T \to I\]
让我们计算它在某个 T-代数 $(a, f)$ 上的分量。遗忘函子忘记了 $f$,自由函子产生了对 $(T a, \mu_a)$。因此,为了定义余单位 $\varepsilon$ 在 $(a, f)$ 上的分量,我们需要 Eilenberg-Moore 范畴中的正确态射,或者说 T-代数的同态:
\[(T a, \mu_a) \to (a, f)\]
这样的同态应将载体 $T a$ 映射到 $a$。让我们直接恢复被遗忘的求值器 $f$。这次我们将它用作 T-代数的同态。实际上,使 $f$ 成为 T-代数的那个交换图可以重新解释为表明它是 T-代数的同态:

\begin{figure}[H]
  \centering
  \begin{tikzcd}[column sep=large, row sep=large]
    T(Ta) \arrow[r, "T f"] \arrow[d, "\mu_a"]
    & Ta \arrow[d, "f"] \\
    Ta \arrow[r, "f"]
    & a
  \end{tikzcd}
\end{figure}

\noindent
因此,我们将余单位自然变换 $\varepsilon$ 在 $(a, f)$(T-代数范畴中的一个对象)上的分量定义为 $f$。

为了完成伴随,我们还需要证明单位和余单位满足三角恒等式。它们是:

\begin{figure}[H]
  \centering
  \begin{subfigure}
    \centering
    \begin{tikzcd}[column sep=large, row sep=large]
      Ta \arrow[rd, equal] \arrow[r, "T \eta_a"]
      & T(Ta) \arrow[d, "\mu_a"] \\
      & Ta
    \end{tikzcd}
  \end{subfigure}%
  \hspace{1cm}
  \begin{subfigure}
    \centering
    \begin{tikzcd}[column sep=large, row sep=large]
      a \arrow[rd, equal] \arrow[r, "\eta_a"]
      & Ta \arrow[d, "f"] \\
      & a
    \end{tikzcd}
  \end{subfigure}
\end{figure}

\noindent
第一个因为 monad $T$ 的单位元定律 (unit law) 而成立。第二个只是 T-代数 $(a, f)$ 的定律。

我们已经确定这两个函子构成一个伴随:
\[F^T \dashv U^T\]
每个伴随都会产生一个 monad。往返过程
\[U^T \circ F^T\]
是 $\cat{C}$ 中的自函子,它产生了相应的 monad。让我们看看它对对象 $a$ 的作用。由 $F^T$ 创建的自由代数是 $(T a, \mu_a)$。遗忘函子 $U^T$ 丢弃了求值器。所以,确实,我们有:
\[U^T \circ F^T = T\]
正如预期的那样,伴随的单位就是 monad $T$ 的单位。

你可能记得伴随的余单位通过以下公式产生 monad 乘法 (monadic multiplication):
\[\mu = R \circ \varepsilon \circ L\]
这是三个自然变换的水平组合 (horizontal composition),其中两个是恒等自然变换,分别将 $L$ 映射到 $L$ 和 $R$ 映射到 $R$。中间那个,余单位,是一个自然变换,其在代数 $(a, f)$ 上的分量是 $f$。

让我们计算分量 $\mu_a$。我们首先在 $F^T$ 之后水平组合 $\varepsilon$,这导致 $\varepsilon$ 在 $F^T a$ 处的分量。由于 $F^T$ 将 $a$ 映到代数 $(T a, \mu_a)$,而 $\varepsilon$ 选取求值器,我们最终得到 $\mu_a$。左侧与 $U^T$ 的水平组合不改变任何东西,因为 $U^T$ 对态射的作用是平凡的。所以,确实,从伴随得到的 $\mu$ 与原始 monad $T$ 的 $\mu$ 相同。

\section{Kleisli 范畴}

我们之前见过 Kleisli 范畴 (Kleisli category)。它是从另一个范畴 $\cat{C}$ 和一个 monad $T$ 构建的范畴。我们称这个范畴为 $\cat{C}_T$。Kleisli 范畴 $\cat{C}_T$ 中的对象是 $\cat{C}$ 的对象,但态射不同。Kleisli 范畴中从 $a$ 到 $b$ 的态射 $f_{\cat{K}}$ 对应于原始范畴中从 $a$ 到 $T b$ 的态射 $f$。我们称这个态射为从 $a$ 到 $b$ 的 Kleisli 箭头 (Kleisli arrow)。

Kleisli 范畴中态射的组合是根据 Kleisli 箭头的 monad 组合 (monadic composition) 定义的。例如,让我们组合 $g_{\cat{K}}$ 在 $f_{\cat{K}}$ 之后。在 Kleisli 范畴中我们有:
\begin{gather*}
  f_{\cat{K}} \Colon a \to b \\
  g_{\cat{K}} \Colon b \to c
\end{gather*}
这在范畴 $\cat{C}$ 中对应于:
\begin{gather*}
  f \Colon a \to T b \\
  g \Colon b \to T c
\end{gather*}
我们定义组合:
\[h_{\cat{K}} = g_{\cat{K}} \circ f_{\cat{K}}\]
作为 $\cat{C}$ 中的 Kleisli 箭头
\begin{align*}
  h & \Colon a \to T c          \\
  h & = \mu \circ (T g) \circ f
\end{align*}
在 Haskell 中我们会这样写:

\src{snippet05}
存在一个从 $\cat{C}$ 到 $\cat{C}_T$ 的函子 $F$,它对对象的作用是平凡的。在态射上,它通过创建一个修饰 $f$ 返回值的 Kleisli 箭头,将 $\cat{C}$ 中的 $f$ 映射到 $\cat{C}_T$ 中的态射。给定一个态射:
\[f \Colon a \to b\]
它在 $\cat{C}_T$ 中创建一个具有相应 Kleisli 箭头的态射:
\[\eta \circ f\]
在 Haskell 中我们会这样写:

\src{snippet06}
我们也可以定义一个从 $\cat{C}_T$ 回到 $\cat{C}$ 的函子 $G$。它接受 Kleisli 范畴中的对象 $a$ 并将其映射到 $\cat{C}$ 中的对象 $T a$。它对对应于 Kleisli 箭头:
\[f \Colon a \to T b\]
的态射 $f_{\cat{K}}$ 的作用是 $\cat{C}$ 中的一个态射:
\[T a \to T b\]
其由先提升 $f$ 然后应用 $\mu$ 给出:
\[\mu_{T b} \circ T f\]
在 Haskell 表示法中,这会读作:

\begin{snipv}
G f\textsubscript{T} = join . fmap f
\end{snipv}
你可能认出这是根据 \code{join} 定义的 monad 绑定 (monadic bind)。

很容易看出这两个函子构成一个伴随:
\[F \dashv G\]
并且它们的组合 $G \circ F$ 再现了原始的 monad $T$。

所以这是产生相同 monad 的第二个伴随。事实上,存在一整个伴随范畴 $\cat{Adj}(\cat{C}, T)$,它们都导致 $\cat{C}$ 上的相同 monad $T$。我们刚刚看到的 Kleisli 伴随是这个范畴中的初始对象 (initial object),而 Eilenberg-Moore 伴随是终对象 (terminal object)。

\section{Comonad 的余代数}

类似地,可以为任何 \hyperref[comonads]{comonad (余单子)} $W$ 进行类似的构造。我们可以定义与 comonad 兼容的余代数 (coalgebras) 范畴。它们使得以下图表交换:

\begin{figure}[H]
  \centering
  \begin{subfigure}
    \centering
    \begin{tikzcd}[column sep=large, row sep=large]
      a \arrow[rd, equal]
      & Wa \arrow[l, "\varepsilon_a"'] \\
      & a \arrow[u, "\mathit{coa}"']
    \end{tikzcd}
  \end{subfigure}%
  \hspace{1cm}
  \begin{subfigure}
    \centering
    \begin{tikzcd}[column sep=large, row sep=large]
      W(Wa)
      & Wa \arrow[l, "W\ \mathit{coa}"'] \\
      Wa \arrow[u, "\delta_a"]
      & a \arrow[u, "\mathit{coa}"] \arrow[l, "\mathit{coa}"']
    \end{tikzcd}
  \end{subfigure}
\end{figure}

\noindent
其中 $\mathit{coa}$ 是载体为 $a$ 的余代数的共求值态射 (coevaluation morphism):
\[\mathit{coa} \Colon a \to W a\]
而 $\varepsilon$ 和 $\delta$ 是定义 comonad 的两个自然变换(在 Haskell 中,它们的分量称为 \code{extract} 和 \code{duplicate})。

存在一个明显的从这些余代数的范畴到 $\cat{C}$ 的遗忘函子 $U^W$。它只是忘记了共求值。我们将考虑它的右伴随 (right adjoint) $F^W$。
\[U^W \dashv F^W\]
遗忘函子的右伴随称为余自由函子 (cofree functor)。$F^W$ 生成余自由余代数 (cofree coalgebras)。它将 $\cat{C}$ 中的对象 $a$ 赋给余代数 $(W a, \delta_a)$。该伴随将原始 comonad 再现为复合 $U^W \circ F^W$。

类似地,我们可以构造一个具有余 Kleisli 箭头的余 Kleisli 范畴 (co-Kleisli category),并从相应的伴随重新生成 comonad。

\section{透镜 (Lenses)}

让我们回到关于透镜 (lenses) 的讨论。透镜可以写成一个余代数:
\[\mathit{coalg}_s \Colon a \to \mathit{Store}\ s\ a\]
其函子为 $\mathit{Store}\ s$:

\src{snippet07}
这个余代数也可以表示为一对函数:
\begin{align*}
  \mathit{set} & \Colon a \to s \to a \\
  \mathit{get} & \Colon a \to s
\end{align*}
(将 $a$ 看作“全部 (all)”,$s$ 看作其中的“小 (small)”部分。) 用这对函数表示,我们有:
\[\mathit{coalg}_s\ a = \mathit{Store}\ (\mathit{set}\ a)\ (\mathit{get}\ a)\]
这里,$a$ 是类型 $a$ 的一个值。注意部分应用的 \code{set} 是一个函数 $s \to a$。

我们也知道 $\mathit{Store}\ s$ 是一个 comonad:

\src{snippet08}
问题是:在什么条件下,透镜是这个 comonad 的余代数?第一个相干条件:
\[\varepsilon_a \circ \mathit{coalg} = \idarrow[a]\]
转化为:
\[\mathit{set}\ a\ (\mathit{get}\ a) = a\]
这是透镜定律 (lens law),表达了这样一个事实:如果你将结构 $a$ 的一个字段设置为其先前的值,则没有任何改变。

第二个条件:
\[\mathit{fmap}\ \mathit{coalg} \circ \mathit{coalg} = \delta_a \circ \mathit{coalg}\]
需要更多的工作。首先,回忆 \code{Store} 函子的 \code{fmap} 定义:

\src{snippet09}
将 \code{fmap coalg} 应用于 \code{coalg} 的结果得到:

\src{snippet10}
另一方面,将 \code{duplicate} 应用于 \code{coalg} 的结果产生:

\src{snippet11}
为了使这两个表达式相等,\code{Store} 下的两个函数在作用于任意 \code{s} 时必须相等:

\src{snippet12}
展开 \code{coalg},我们得到:

\src{snippet13}
这等价于剩下的两个透镜定律。第一个:

\src{snippet14}
告诉我们设置字段值两次与设置一次相同。第二个定律:

\src{snippet15}
告诉我们获取设置为 $s$ 的字段的值会得到 $s$。

换句话说,一个行为良好 (well-behaved) 的透镜确实是 \code{Store} 函子的 comonad 余代数。

\section{挑战}

\begin{enumerate}
  \tightlist
  \item
        自由函子 $F \Colon C \to C^T$ 对态射的作用是什么?提示:使用 monad $\mu$ 的自然性条件 (naturality condition)。
  \item
        定义伴随:
        \[U^W \dashv F^W\]
  \item
        证明上述伴随再现了原始的 comonad。
\end{enumerate}