% !TEX root = ../../ctfp-print.tex

\lettrine[lhang=0.17]{范}{畴论中的}大多数构造都是对数学其他更具体领域结果的推广。像积 (products)、余积 (coproducts)、幺半群 (monoids)、指数 (exponentials) 等概念,在范畴论出现之前很久就已经为人所知了。它们在数学的不同分支中可能以不同的名称为人所知。集合论中的笛卡尔积 (Cartesian product),序理论 (order theory) 中的交 (meet),逻辑学 (logic) 中的合取 (conjunction)——它们都是范畴积 (categorical product) 这个抽象概念的具体例子。

Yoneda 引理 (Yoneda lemma) 在这方面脱颖而出,它是一个关于一般范畴的宏大陈述,在数学的其他分支中几乎没有先例。有人说,它最接近的类比是群论 (group theory) 中的 Cayley 定理 (Cayley's theorem)(每个群都同构于某个集合的置换群 (permutation group))。

Yoneda 引理的设定是一个任意范畴 $\cat{C}$ 以及一个从 $\cat{C}$ 到 $\Set$ 的函子 $F$。我们在上一节中看到,一些以 $\Set$ 为值的函子是可表示的 (representable),即与某个 hom-函子 (hom-functor) 同构。Yoneda 引理告诉我们,所有以 $\Set$ 为值的函子都可以通过自然变换 (natural transformations) 从 hom-函子得到,并且它明确地列举了所有这样的变换。

当我谈到自然变换时,我提到自然性条件 (naturality condition) 可能相当严格。当你在一个对象上定义自然变换的一个分量 (component) 时,自然性可能强大到足以将这个分量“传输”到通过态射与之相连的另一个对象。源范畴和目标范畴中对象之间的箭头越多,传输自然变换分量的约束就越多。$\Set$ 恰好是一个箭头非常丰富的范畴。

Yoneda 引理告诉我们,一个 hom-函子和任何其他函子 $F$ 之间的自然变换,完全由其在仅仅一个点上的单个分量的值确定!自然变换的其余部分仅仅由自然性条件导出。

那么让我们回顾一下 Yoneda 引理所涉及的两个函子之间的自然性条件。第一个函子是 hom-函子。它将 $\cat{C}$ 中的任何对象 $x$ 映射到态射集合 $\cat{C}(a, x)$——其中 $a$ 是 $\cat{C}$ 中的一个固定对象。我们还看到它将任何从 $x \to y$ 的态射 $f$ 映射到 $\cat{C}(a, f)$。

第二个函子是一个任意的以 $\Set$ 为值的函子 $F$。

让我们称这两个函子之间的自然变换为 $\alpha$。因为我们在 $\Set$ 中操作,自然变换的分量,如 $\alpha_x$ 或 $\alpha_y$,只是集合之间的常规函数:
\begin{gather*}
  \alpha_x \Colon \cat{C}(a, x) \to F x \\
  \alpha_y \Colon \cat{C}(a, y) \to F y
\end{gather*}

\begin{figure}[H]
  \centering
  \includegraphics[width=0.4\textwidth]{images/yoneda1.png}
\end{figure}

\noindent
并且因为这些只是函数,我们可以查看它们在特定点上的值。但是集合 $\cat{C}(a, x)$ 中的一个点是什么?这里的关键观察是:集合 $\cat{C}(a, x)$ 中的每个点也是一个从 $a$ 到 $x$ 的态射 $h$。

所以 $\alpha$ 的自然性方块:
\[\alpha_y \circ \cat{C}(a, f) = F f \circ \alpha_x\]
逐点地作用于 $h$ 时,变成:
\[\alpha_y (\cat{C}(a, f) h) = (F f) (\alpha_x h)\]
你可能还记得上一节中,hom-函子 $\cat{C}(a,-)$ 作用于态射 $f$ 被定义为前复合 (precomposition):
\[\cat{C}(a, f) h = f \circ h\]
这导致:
\[\alpha_y (f \circ h) = (F f) (\alpha_x h)\]
这个条件有多强,可以通过将其特殊化到 $x = a$ 的情况来看出。

\begin{figure}[H]
  \centering
  \includegraphics[width=0.4\textwidth]{images/yoneda2.png}
\end{figure}

\noindent
在这种情况下,$h$ 变成一个从 $a$ 到 $a$ 的态射。我们知道至少存在一个这样的态射,$h = \id_a$。让我们把它代入:
\[\alpha_y f = (F f) (\alpha_a \id_a)\]
注意刚刚发生了什么:左边是 $\alpha_y$ 作用于 $\cat{C}(a, y)$ 的任意元素 $f$。而它完全由 $\alpha_a$ 在 $\id_a$ 处的单个值确定。我们可以选择任何这样的值,它将生成一个自然变换。由于 $\alpha_a$ 的值在集合 $F a$ 中,因此 $F a$ 中的任何点都将定义某个 $\alpha$。

反过来,给定任何从 $\cat{C}(a, -)$ 到 $F$ 的自然变换 $\alpha$,你可以在 $\id_a$ 处对其求值,得到 $F a$ 中的一个点。

我们刚刚证明了 Yoneda 引理:

\begin{quote}
  从 $\cat{C}(a, -)$ 到 $F$ 的自然变换与 $F a$ 的元素之间存在一一对应关系。
\end{quote}
换句话说,
\[\mathit{Nat}(\cat{C}(a, -), F) \cong F a\]
或者,如果我们使用 $[\cat{C}, \Set]$ 表示 $\cat{C}$ 和 $\Set$ 之间的函子范畴 (functor category),那么自然变换的集合只是该范畴中的一个 hom-集,我们可以写成:
\[[\cat{C}, \Set](\cat{C}(a, -), F) \cong F a\]
我稍后会解释这种对应关系实际上是如何成为一个自然同构 (natural isomorphism) 的。

现在让我们试着对这个结果获得一些直觉。最令人惊奇的是,整个自然变换仅从一个种子点 (nucleation site) 结晶出来:即我们赋给它在 $\id_a$ 处的值。它从那个点开始,遵循自然性条件扩散开来。它淹没了 $\cat{C}$ 在 $\Set$ 中的像。所以让我们首先考虑 $\cat{C}$ 在 $\cat{C}(a, -)$ 下的像是什么。

让我们从 $a$ 本身的像开始。在 hom-函子 $\cat{C}(a, -)$ 下,$a$ 被映射到集合 $\cat{C}(a, a)$。另一方面,在函子 $F$ 下,它被映射到集合 $F a$。自然变换 $\alpha_a$ 的分量是从 $\cat{C}(a, a)$ 到 $F a$ 的某个函数。让我们只关注集合 $\cat{C}(a, a)$ 中的一个点,即对应于态射 $\id_a$ 的点。为了强调它只是集合中的一个点,让我们称它为 $p$。分量 $\alpha_a$ 应该将 $p$ 映射到 $F a$ 中的某个点 $q$。我将向你展示,任何 $q$ 的选择都会导致一个唯一的自然变换。

\begin{figure}[H]
  \centering
  \includegraphics[width=0.3\textwidth]{images/yoneda3.png}
\end{figure}

\noindent
第一个论断是,一个点 $q$ 的选择唯一地决定了函数 $\alpha_a$ 的其余部分。确实,让我们选择 $\cat{C}(a, a)$ 中的任何其他点 $p'$,对应于某个从 $a$ 到 $a$ 的态射 $g$。这就是 Yoneda 引理的魔力所在:$g$ 可以被视为集合 $\cat{C}(a, a)$ 中的一个点 $p'$。同时,它选择了两个集合之间的 \emph{函数}。确实,在 hom-函子下,态射 $g$ 被映射到函数 $\cat{C}(a, g)$;在 $F$ 下,它被映射到 $F g$。

\begin{figure}[H]
  \centering
  \includegraphics[width=0.4\textwidth]{images/yoneda4.png}
\end{figure}

\noindent
现在让我们考虑 $\cat{C}(a, g)$ 对我们原始的 $p$(你还记得,它对应于 $\id_a$)的作用。它被定义为前复合,$g \circ \id_a$,等于 $g$,对应于我们的点 $p'$。所以态射 $g$ 被映射到一个函数,该函数作用于 $p$ 时产生 $p'$,也就是 $g$。我们绕了一圈又回来了!

现在考虑 $F g$ 对 $q$ 的作用。它是某个 $q'$,是 $F a$ 中的一个点。为了完成自然性方块,$p'$ 必须在 $\alpha_a$ 下被映射到 $q'$。我们选择了一个任意的 $p'$(一个任意的 $g$)并推导出了它在 $\alpha_a$ 下的映射。因此,函数 $\alpha_a$ 被完全确定了。

第二个论断是,对于 $\cat{C}$ 中任何与 $a$ 相连的对象 $x$,$\alpha_x$ 被唯一确定。推理是类似的,只是现在我们多了两个集合,$\cat{C}(a, x)$ 和 $F x$,以及从 $a$ 到 $x$ 的态射 $g$ 在 hom-函子下被映射到:
\[\cat{C}(a, g) \Colon \cat{C}(a, a) \to \cat{C}(a, x)\]
在 $F$ 下被映射到:
\[F g \Colon F a \to F x\]
同样,$\cat{C}(a, g)$ 作用于我们的 $p$ 由前复合给出:$g \circ \id_a$,它对应于 $\cat{C}(a, x)$ 中的一个点 $p'$。自然性决定了 $\alpha_x$ 作用于 $p'$ 的值为:
\[q' = (F g) q\]
由于 $p'$ 是任意的,整个函数 $\alpha_x$ 因此被确定。

\begin{figure}[H]
  \centering
  \includegraphics[width=0.4\textwidth]{images/yoneda5.png}
\end{figure}

\noindent
如果 $\cat{C}$ 中存在与 $a$ 没有连接的对象呢?它们在 $\cat{C}(a, -)$ 下都被映射到同一个集合——空集 (empty set)。回想一下,空集是集合范畴中的初始对象 (initial object)。这意味着从这个集合到任何其他集合都存在唯一的函数。我们称这个函数为 \code{absurd}。所以在这里,我们同样对自然变换的分量没有选择:它只能是 \code{absurd}。

理解 Yoneda 引理的一种方式是认识到,以 $\Set$ 为值的函子之间的自然变换只是一族函数,而函数通常是有损的。函数可能会压缩信息,并且可能只覆盖其陪域 (codomain) 的一部分。唯一不损失信息的函数是那些可逆的函数——即同构 (isomorphisms)。由此可知,保持结构的最好的以 $\Set$ 为值的函子是那些可表示的函子。它们要么是 hom-函子,要么是与 hom-函子自然同构的函子。任何其他函子 $F$ 都是通过有损变换从 hom-函子获得的。这样的变换不仅可能丢失信息,而且还可能只覆盖函子 $F$ 在 $\Set$ 中像的一小部分。

\section{Haskell 中的 Yoneda}

我们已经在 Haskell 中以 reader 函子的形式遇到过 hom-函子:

\src{snippet01}
reader 通过前复合来映射态射(这里是函数):

\src{snippet02}
Yoneda 引理告诉我们,reader 函子可以自然地映射到任何其他函子。

自然变换是一个多态函数 (polymorphic function)。因此,给定一个函子 \code{F},我们有一个从 reader 函子到它的映射:

\src{snippet03}
像往常一样,\code{forall} 是可选的,但我喜欢明确写出它来强调自然变换的参数多态 (parametric polymorphism)。

Yoneda 引理告诉我们,这些自然变换与 \code{F a} 的元素一一对应:

\begin{snipv}
forall x . (a -> x) -> F x \ensuremath{\cong} F a
\end{snipv}
这个等式的右边是我们通常会认为是数据结构的东西。还记得将函子解释为广义容器吗?\code{F a} 是 \code{a} 的容器。但左边是一个以函数作为参数的多态函数。Yoneda 引理告诉我们这两种表示是等价的——它们包含相同的信息。

另一种说法是:给我一个类型为:

\src{snippet04}
的多态函数,我将产生一个 \code{a} 的容器。诀窍就是我们在 Yoneda 引理证明中使用的那个:我们用 \code{id} 调用这个函数来获得 \code{F a} 的一个元素:

\src{snippet05}
反之亦然:给定一个类型为 \code{F a} 的值:

\src{snippet06}
可以定义一个多态函数:

\src{snippet07}
具有正确的类型。你可以轻松地在这两种表示之间来回转换。

拥有多种表示的优点在于,其中一种可能比另一种更容易组合,或者在某些应用中比另一种更高效。

这个原理最简单的例证是编译器构造中经常使用的代码转换:续延传递风格 (continuation passing style, CPS) (\acronym{CPS})。这是 Yoneda 引理应用于恒等函子 (identity functor) 的最简单应用。用恒等替换 \code{F} 产生:

\begin{snipv}
forall r . (a -> r) -> r \ensuremath{\cong} a
\end{snipv}
这个公式的解释是,任何类型 \code{a} 都可以被一个接受 \code{a} 的“处理器 (handler)” 的函数所替代。处理器是一个接受 \code{a} 并执行计算其余部分——即续延 (continuation)——的函数。(类型 \code{r} 通常封装某种状态码。)

这种编程风格在用户界面 (UI)、异步系统 (asynchronous systems) 和并发编程 (concurrent programming) 中非常常见。\acronym{CPS} 的缺点是它涉及控制反转 (inversion of control)。代码被分割在生产者 (producers) 和消费者 (consumers)(处理器)之间,并且不容易组合。任何做过相当数量非平凡 web 编程的人都熟悉由交互的有状态处理器 (stateful handlers) 产生的意大利面条式代码 (spaghetti code) 的噩梦。正如我们稍后将看到的,明智地使用函子和单子 (monads) 可以恢复 \acronym{CPS} 的某些组合性质 (compositional properties)。

\section{Co-Yoneda}

像往常一样,通过反转箭头的方向,我们得到了一个附赠的构造。Yoneda 引理可以应用于对偶范畴 (opposite category) $\cat{C}^\mathit{op}$,从而为我们提供逆变函子 (contravariant functors) 之间的映射。

等价地,我们可以通过固定 hom-函子的目标对象而不是源对象来推导 co-Yoneda 引理 (co-Yoneda lemma)。我们得到从 $\cat{C}$ 到 $\Set$ 的逆变 hom-函子:$\cat{C}(-, a)$。Yoneda 引理的逆变版本建立了从该函子到任何其他逆变函子 $F$ 的自然变换与集合 $F a$ 的元素之间的一一对应关系:
\[\mathit{Nat}(\cat{C}(-, a), F) \cong F a\]
这是 co-Yoneda 引理的 Haskell 版本:

\begin{snipv}
forall x . (x -> a) -> F x \ensuremath{\cong} F a
\end{snipv}
请注意,在某些文献中,是逆变版本被称为 Yoneda 引理。

\section{挑战}

\begin{enumerate}
  \tightlist
  \item
        证明在 Haskell 中构成 Yoneda 同构的两个函数 \code{phi} 和 \code{psi} 互为逆运算。

        \begin{snip}{haskell}
phi :: (forall x . (a -> x) -> F x) -> F a
phi alpha = alpha id

psi :: F a -> (forall x . (a -> x) -> F x)
psi fa h = fmap h fa
\end{snip}
  \item
        离散范畴 (discrete category) 是指只有对象而除恒等态射外没有其他态射的范畴。对于从此类范畴出发的函子,Yoneda 引理是如何工作的?
  \item
        一个单位列表 \code{{[}(){]}} 除了其长度之外不包含其他信息。因此,作为一种数据类型,它可以被视为整数的一种编码。空列表编码零,单元列表 \code{{[}(){]}}(一个值,不是一个类型)编码一,依此类推。使用列表函子的 Yoneda 引理为这个数据类型构造另一种表示。
\end{enumerate}

\section{参考文献}

\begin{enumerate}
  \tightlist
  \item
        \urlref{https://www.youtube.com/watch?v=TLMxHB19khE}{Catsters} 视频。
\end{enumerate}