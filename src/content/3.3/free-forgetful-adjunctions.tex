% !TEX root = ../../ctfp-print.tex

\lettrine[lhang=0.17]{自}{由构造 (Free constructions) 是}伴随 (adjunctions) 的一个强大应用。一个 \newterm{自由函子 (free functor)} 被定义为一个 \newterm{遗忘函子 (forgetful functor)} 的左伴随 (left adjoint)。遗忘函子通常是一个相当简单的函子,它遗忘某些结构。例如,许多有趣的范畴是建立在集合之上的。但是范畴对象,它们抽象了这些集合,没有内部结构——它们没有元素。尽管如此,这些对象通常带有集合的记忆,因为存在一个映射——一个函子——从给定的范畴 $\cat{C}$ 到 $\Set$。$\cat{C}$ 中某个对象对应的集合被称为其 \newterm{底层集合 (underlying set)}。

幺半群 (Monoids) 就是这种具有底层集合——元素集合——的对象。存在一个从幺半群范畴 (category of monoids) $\cat{Mon}$ 到集合范畴 $\Set$ 的遗忘函子 $U$,它将幺半群映射到它们的底层集合。它也将幺半群态射(同态 (homomorphisms))映射到集合之间的函数。

我喜欢将 $\cat{Mon}$ 看作具有分裂的人格。一方面,它是一堆带有乘法和单位元的集合。另一方面,它是一个范畴,其对象没有特征,唯一的结构编码在它们之间的态射中。每个保持乘法和单位元的集合函数都产生 $\cat{Mon}$ 中的一个态射。\\
\newline
需要记住的事情:

\begin{itemize}
  \tightlist
  \item
        可能存在许多幺半群映射到同一个集合,并且
  \item
        幺半群态射的数量少于(或最多等于)其底层集合之间的函数数量。
\end{itemize}

\noindent
作为遗忘函子 $U$ 的左伴随的函子 $F$ 是从它们的生成元集合 (generator sets) 构建自由幺半群 (free monoids) 的自由函子。这个伴随源于我们之前讨论过的自由幺半群泛构造。\footnote{参见关于 \hyperref[free-monoids]{自由幺半群} 的第 13 章。}

\begin{figure}[H]
  \centering
  \includegraphics[width=0.6\textwidth]{images/forgetful.jpg}
  \caption{幺半群 $m_1$ 和 $m_2$ 具有相同的底层集合。在 $m_2$ 和 $m_3$ 的底层集合之间,函数的数量多于它们之间的态射数量。}
\end{figure}

\noindent
用 hom-集来说,我们可以将这个伴随写成:
\[\cat{Mon}(F x, m) \cong \Set(x, U m)\]
这个(在 $x$ 和 $m$ 上自然的)同构 (isomorphism) 告诉我们:

\begin{itemize}
  \tightlist
  \item
        对于由 $x$ 生成的自由幺半群 $F x$ 和任意幺半群 $m$ 之间的每一个幺半群同态,都存在一个唯一的函数,将生成元集合 $x$ 嵌入 (embeds) 到 $m$ 的底层集合中。这是一个 $\Set(x, U m)$ 中的函数。
  \item
        对于每一个将 $x$ 嵌入到某个 $m$ 的底层集合中的函数,都存在一个唯一的幺半群态射,介于由 $x$ 生成的自由幺半群和幺半群 $m$ 之间。(这是我们在泛构造中称为 $h$ 的态射。)
\end{itemize}

\begin{figure}[H]
  \centering
  \includegraphics[width=0.8\textwidth]{images/freemonadjunction.jpg}
\end{figure}

\noindent
直觉是 $F x$ 是可以在 $x$ 的基础上构建的“最大”幺半群。如果我们能看到幺半群的内部,我们会发现任何属于 $\cat{Mon}(F x, m)$ 的态射都将这个自由幺半群 \emph{嵌入} 到某个其他幺半群 $m$ 中。它通过可能地等同化 (identifying) 一些元素来做到这一点。特别是,它将 $F x$ 的生成元(即 $x$ 的元素)嵌入到 $m$ 中。伴随表明,$x$ 的嵌入,由右侧 $\Set(x, U m)$ 中的函数给出,唯一地决定了左侧幺半群的嵌入,反之亦然。

在 Haskell 中,列表数据结构是一个自由幺半群(有一些注意事项:参见 \urlref{http://comonad.com/reader/2015/free-monoids-in-haskell/}{Dan Doel 的博客文章})。列表类型 \code{{[}a{]}} 是一个自由幺半群,其中类型 \code{a} 表示生成元集合。例如,类型 \code{{[}Char{]}} 包含单位元——空列表 \code{{[}{]}}——以及像 \code{{[}'a'{]}}、\code{{[}'b'{]}} 这样的单元列表 (singletons)——自由幺半群的生成元。其余的通过应用“积”来生成。这里,两个列表的积只是将一个附加到另一个后面。附加运算是结合的 (associative) 并且有单位元(即存在一个中性元素——这里是空列表)。由 \code{Char} 生成的自由幺半群只不过是来自 \code{Char} 的所有字符组成的字符串集合。它在 Haskell 中被称为 \code{String}:

\src{snippet01}
(\code{type} 定义了一个类型别名 (type synonym)——一个现有类型的不同名称)。

另一个有趣的例子是由单个生成元构建的自由幺半群。它是单位列表的类型,\code{{[}(){]}}。它的元素是 \code{{[}{]}}、\code{{[}(){]}}、\code{{[}(), (){]}} 等。每个这样的列表都可以用一个自然数 (natural number)——它的长度——来描述。单位列表中没有编码更多的信息。将两个这样的列表附加在一起会产生一个新列表,其长度是其组成部分长度的总和。很容易看出类型 \code{{[}(){]}} 与自然数(带零)的加法幺半群 (additive monoid) 是同构的。这里有两个互为逆的函数,证明了这种同构:

\src{snippet02}
为简单起见,我使用了类型 \code{Int} 而不是 \code{Natural},但思想是相同的。函数 \code{replicate} 创建一个长度为 \code{n} 的列表,预先填充了给定的值——这里是单位元。

\section{一些直觉}

接下来是一些粗略的论证。这类论证远非严谨,但有助于形成直觉。

为了对自由/遗忘伴随获得一些直觉,记住函子和函数本质上是有损耗的 (lossy) 是有帮助的。函子可能会合并多个对象和态射,函数可能会将集合中的多个元素捆绑在一起。而且,它们的像可能只覆盖其陪域 (codomain) 的一部分。

$\Set$ 中的一个“平均”hom-集将包含从损耗最小的函数(例如,单射 (injections),或者可能是同构)到将整个域 (domain) 压缩到单个元素的常数函数(如果存在的话)的整个谱系。

我倾向于认为任意范畴中的态射也是有损耗的。这只是一个心智模型,但它很有用,尤其是在思考伴随——特别是其中一个范畴是 $\Set$ 的伴随时。

形式上,我们只能谈论可逆的态射(同构)或不可逆的态射。正是后者可以被认为是损耗的。还有单态射 (mono-morphisms) 和满态射 (epi-morphisms) 的概念,它们推广了单射(不压缩)和满射 (surjective)(覆盖整个陪域)函数的思想,但可能存在既是单态射又是满态射,但仍然不可逆的态射。

在自由 ⊣ 遗忘 (Free ⊣ Forgetful) 伴随中,我们在左侧有一个更受约束的范畴 $\cat{C}$,在右侧有一个约束较少的范畴 $\cat{D}$。$\cat{C}$ 中的态射“较少”,因为它们必须保留一些额外的结构。在 $\cat{Mon}$ 的情况下,它们必须保留乘法和单位元。$\cat{D}$ 中的态射不必保留那么多结构,所以它们“更多”。

当我们对 $\cat{C}$ 中的对象 $c$ 应用遗忘函子 $U$ 时,我们将其视为揭示 $c$ 的“内部结构”。事实上,如果 $\cat{D}$ 是 $\Set$,我们认为 $U$ \emph{定义} 了 $c$ 的内部结构——它的底层集合。(在任意范畴中,除了通过与其他对象的连接之外,我们无法谈论对象的内部,但这里我们只是在进行粗略的论证。)

如果我们使用 $U$ 映射两个对象 $c'$ 和 $c$,我们期望,一般而言,hom-集 $\cat{C}(c', c)$ 的映射将只覆盖 $\cat{D}(U c', U c)$ 的一个子集。这是因为 $\cat{C}(c', c)$ 中的态射必须保留额外的结构,而 $\cat{D}(U c', U c)$ 中的态射则不必。

\begin{figure}[H]
  \centering
  \includegraphics[width=0.45\textwidth]{images/forgettingmorphisms.jpg}
\end{figure}

\noindent
但是由于伴随被定义为特定 hom-集的 \newterm{同构},我们在选择 $c'$ 时必须非常挑剔。在伴随中,$c'$ 不是从 $\cat{C}$ 中的任何地方选取的,而是从自由函子 $F$ 的(大概更小的)像中选取的:
\[\cat{C}(F d, c) \cong \cat{D}(d, U c)\]
因此,$F$ 的像必须由具有大量指向任意 $c$ 的态射的对象组成。事实上,从 $F d$ 到 $c$ 的保持结构的态射必须与从 $d$ 到 $U c$ 的不保持结构的态射一样多。这意味着 $F$ 的像必须由基本上没有结构的对象组成(这样态射就没有结构需要保留)。这种“无结构”的对象被称为自由对象 (free objects)。

\begin{figure}[H]
  \centering
  \includegraphics[width=0.45\textwidth]{images/freeimage.jpg}
\end{figure}

\noindent
在幺半群的例子中,自由幺半群除了由单位元和结合律生成的结构外,没有其他结构。除此之外,所有的乘法都会产生全新的元素。

在自由幺半群中,$2 * 3$ 不是 $6$——它是一个新元素 ${[}2, 3{]}$。由于 ${[}2, 3{]}$ 和 $6$ 之间没有等同化,从这个自由幺半群到任何其他幺半群 $m$ 的态射允许将它们分开映射。但它也可以将 ${[}2, 3{]}$ 和 $6$(它们的积)都映射到 $m$ 的同一个元素。或者在加法幺半群中将 ${[}2, 3{]}$ 和 $5$(它们的和)等同起来,等等。不同的等同化给你不同的幺半群。

这引出了另一个有趣的直觉:自由幺半群不是执行幺半群运算,而是累积传递给它的参数。它们不是将 $2$ 和 $3$ 相乘,而是在列表中记住 $2$ 和 $3$。这种方案的优点是我们不必指定将使用哪种幺半群运算。我们可以继续累积参数,只在最后将一个运算符应用于结果。然后我们可以选择应用哪个运算符。我们可以将数字相加,或相乘,或执行模 2 加法,等等。自由幺半群将表达式的创建与其求值分开。我们在讨论代数 (algebras) 时会再次看到这个想法。

这种直觉可以推广到其他更复杂的自由构造。例如,我们可以在求值之前累积整个表达式树 (expression trees)。这种方法的优点是我们可以转换这些树以使求值更快或消耗更少的内存。例如,这在实现矩阵演算时会用到,其中急切求值 (eager evaluation) 会导致大量分配临时数组来存储中间结果。

\section{挑战}

\begin{enumerate}
  \tightlist
  \item
        考虑由单元集 (singleton set) 作为其生成元构建的自由幺半群。证明从这个自由幺半群到任何幺半群 $m$ 的态射与从单元集到 $m$ 的底层集合的函数之间存在一一对应关系。
\end{enumerate}