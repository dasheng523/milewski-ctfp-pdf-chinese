% !TEX root = ../../ctfp-print.tex

\begin{quote}
  一段时间以来,我一直在构思写一本关于 category theory (范畴论) 的书,这本书将面向程序员 (programmers)。请注意,不是计算机科学家 (computer scientists),而是程序员 —— 是工程师 (engineers) 而非科学家 (scientists)。我知道这听起来很疯狂,我也确实感到惶恐。我无法否认科学与工程之间存在巨大的鸿沟,因为我曾在鸿沟的两边都工作过。但我总是感到一种强烈的冲动要去解释事物。我非常钦佩理查德·费曼 (Richard Feynman),他是简单解释的大师。我知道我不是费曼,但我会尽力而为。我从发布这篇序言开始 —— 它旨在激励读者学习范畴论 —— 希望能以此引发讨论并征求反馈。\footnote{
    你也可以在 \href{https://goo.gl/GT2UWU}{https://goo.gl/GT2UWU} (或在 YouTube 上搜索 “bartosz milewski category theory”) 观看我向现场观众讲授这些内容。}
\end{quote}

\lettrine[lhang=0.17]{我}{将尝试}, 在寥寥数段之内,说服你这本书正是为你而写,并且让你相信,无论你可能对在 “充裕的业余时间” 学习数学中最抽象的分支之一有何疑虑,这些疑虑都毫无根据。

我的乐观基于几点观察。首先,范畴论是极其有用的编程思想的宝库。Haskell (Haskell) 程序员长期以来一直在利用这一资源,这些思想正缓慢渗透到其他语言中,但这个过程太慢了。我们需要加快这个进程。

其次,数学有很多不同种类,它们吸引不同的受众。你可能对微积分 (calculus) 或代数 (algebra) 过敏,但这并不意味着你不会喜欢范畴论。我甚至敢说,范畴论是那种特别适合程序员思维方式的数学。这是因为范畴论——与其说处理具体细节——不如说处理结构 (structure)。它处理的是那种能让程序可组合 (composable) 的结构。

组合 (Composition) 是范畴论的核心——它是范畴本身定义的一部分。而且我将极力论证,组合是编程的本质。我们一直在组合事物,远在某个伟大的工程师提出子例程 (subroutine) 的想法之前。不久前,结构化编程 (structured programming) 的原则革新了编程,因为它们使得代码块可组合。接着出现了面向对象编程 (object oriented programming),其核心就是组合对象。函数式编程 (Functional programming) 不仅仅是关于组合函数 (function) 和代数数据结构 (algebraic data structures)——它还使得并发 (concurrency) 可组合——这在其他编程范式 (programming paradigms) 中几乎是不可能的。

第三,我有一个秘密武器,一把屠夫的刀,我将用它来 ‘肢解’ 数学,使其对程序员更易接受。当你是一名专业数学家时,你必须非常小心地厘清所有假设,恰当地限定每一个陈述,并严格地构建所有证明。这使得数学论文和书籍对于局外人来说极其难读。我受过物理学训练,在物理学中,我们通过非形式推理 (informal reasoning) 取得了惊人的进展。数学家们曾嘲笑狄拉克 δ 函数 (Dirac delta function),那是伟大的物理学家 P. A. M. 狄拉克为了解决某些微分方程而当场发明的。当他们发现了一个称为分布理论 (distribution theory) 的全新微积分分支,将狄拉克的洞见形式化之后,他们便不再嘲笑了。

当然,在使用 ‘挥手示意’ 式论证 (hand-waving arguments) 时,你可能会冒着说出明显错误的风险,所以我将尽力确保本书中非形式论证的背后有坚实的数学理论支撑。我的床头柜上确实放着一本翻旧了的 Saunders Mac Lane 的 \emph{Categories for the Working Mathematician}。

由于这是 \emph{写给程序员} 的范畴论,我将使用计算机代码来阐释所有主要概念。你可能知道,函数式语言 (functional languages) 比更流行的命令式语言 (imperative languages) 更接近数学。它们也提供了更强的抽象能力 (abstracting power)。因此,一个自然的诱惑是说:你必须先学习 Haskell,然后范畴论的宝藏才会向你敞开。但这将暗示范畴论在函数式编程之外没有应用,而这根本不是真的。所以我将提供大量 C++ (C++) 示例。诚然,你将不得不克服一些丑陋的语法,模式可能无法从冗长的背景中脱颖而出,而且你可能被迫进行一些复制粘贴来代替更高层次的抽象,但这只是 C++ 程序员的宿命。

但在 Haskell 方面,你并不能完全置身事外。你不必成为 Haskell 程序员,但你需要它作为一种语言,来勾勒和记录那些要在 C++ 中实现的想法。我当初开始接触 Haskell 正是如此。我发现它简洁的语法和强大的类型系统 (type system) 在理解和实现 C++ 模板 (templates)、数据结构 (data structures) 和算法 (algorithms) 方面有很大帮助。但因为我不能期望读者已经了解 Haskell,我会慢慢介绍它,并边讲边解释一切。

如果你是一位经验丰富的程序员,你可能会问自己:我编程这么久,从没担心过范畴论或函数式方法,那么现在有什么变化呢?你肯定注意到了,新的函数式特性正源源不断地侵入命令式语言。即使是面向对象编程的堡垒 Java,也引入了 lambda 表达式 (lambdas)。C++ 近来以疯狂的速度发展——每隔几年就出一个新标准——试图跟上这个变化的世界。所有这些活动都是在为一个颠覆性的变化做准备,或者,正如我们物理学家所称的,一个相变 (phase transition)。如果你持续给水加热,它最终会沸腾。我们现在就像一只青蛙,必须决定是继续在越来越热的水中游泳,还是开始寻找其他选择。

\begin{figure}[H]
  \centering
  \includegraphics[width=0.5\textwidth]{images/img_1299.jpg}
\end{figure}

\noindent
推动这一巨大变化的驱动力之一是多核革命 (multicore revolution)。流行的编程范式,面向对象编程,在并发和并行 (parallelism) 领域毫无优势,反而鼓励了危险且易出错的设计。数据隐藏 (Data hiding),面向对象的基本前提,当与共享 (sharing) 和可变性 (mutation) 结合时,就成了数据竞争 (data races) 的温床。将互斥锁 (mutex) 与其保护的数据结合起来的想法很好,但不幸的是,锁 (locks) 不可组合,而且隐藏锁使得死锁 (deadlocks) 更有可能发生且更难调试。

但即使没有并发,软件系统日益增长的复杂性也在考验命令式范式的可伸缩性 (scalability) 的极限。简单来说,副作用 (side effects) 正在失控。诚然,具有副作用的函数通常很方便且易于编写。原则上,它们的效果可以通过函数名和注释来体现。名为 SetPassword 或 WriteFile 的函数显然在改变某些状态并产生副作用,我们已经习惯于处理这种情况。只有当我们开始将具有副作用的函数与同样具有副作用的其他函数层层组合时,事情才开始变得棘手。并非副作用本身天生就是坏的——而是它们被隐藏起来,这使得它们在大规模下无法管理。副作用无法扩展,而命令式编程的核心就是副作用。

硬件的变化和软件日益增长的复杂性正迫使我们重新思考编程的基础。就像欧洲宏伟的哥特式大教堂的建造者一样,我们一直在将我们的技艺磨练到材料和结构的极限。在法国博韦 (Beauvais),有一座未完工的哥特式 \urlref{http://en.wikipedia.org/wiki/Beauvais_Cathedral}{大教堂},它见证了人类在局限性面前的深刻挣扎。它旨在打破先前所有的高度和轻盈度记录,但却经历了一系列坍塌。像铁棍和木支架这样的临时措施 (Ad hoc measures) 使其免于解体,但显然很多地方都出了问题。从现代角度看,如此多的哥特式建筑能够在没有现代材料科学 (material science)、计算机建模 (computer modelling)、有限元分析 (finite element analysis) 以及通用数学和物理学帮助的情况下成功完成,简直是个奇迹。我希望后代能够同样钦佩我们在构建复杂操作系统、Web 服务器和互联网基础设施方面所展示的编程技能。而且,坦率地说,他们应该钦佩,因为我们是基于非常薄弱的理论基础完成所有这一切的。如果我们想要前进,就必须修复这些基础。

\begin{figure}
  \centering
  \includegraphics[totalheight=0.5\textheight]{images/beauvais_interior_supports.jpg}
  \caption{防止博韦大教堂坍塌的临时措施。}
\end{figure}