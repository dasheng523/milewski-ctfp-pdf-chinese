% !TEX root = ../../ctfp-print.tex

\lettrine[lhang=0.17]{如}{果我还没有}说服你范畴论 (category theory) 完全是关于态射 (morphisms) 的,那我就没有尽到我的职责。由于下一个主题是伴随 (adjunctions),它是根据 hom-集 (hom-sets) 的同构 (isomorphisms) 来定义的,因此回顾一下我们对 hom-集构件的直觉是有意义的。此外,你将会看到伴随提供了一种更通用的语言来描述我们之前研究过的许多构造,所以回顾它们可能也会有所帮助。

\section{函子}

首先,你真的应该把函子 (functors) 看作是态射的映射 (mappings of morphisms)——Haskell 中 \code{Functor} 类型类 (typeclass) 的定义就强调了这一点,它围绕着 \code{fmap}。当然,函子也映射对象——态射的端点——否则我们就无法谈论保持复合 (preserving composition)。对象告诉我们哪些态射对是可复合的。一个态射的目标必须等于另一个态射的源——如果它们要复合的话。因此,如果我们希望态射的复合被映射到 \newterm{提升 (lifted)} 态射的复合,那么它们端点的映射基本上就被确定了。

\section{可换图}

态射的许多性质都是用可换图 (commuting diagrams) 来表达的。如果某个特定的态射可以用多种方式描述为其他态射的复合,那么我们就得到了一个可换图。

特别是,可换图构成了几乎所有泛构造 (universal constructions) 的基础(除了初始对象 (initial object) 和终结对象 (terminal object) 这两个显著的例外)。我们在积 (products)、余积 (coproducts)、各种其他(余)极限 ((co-)limits)、指数对象 (exponential objects)、自由幺半群 (free monoids) 等的定义中都看到了这一点。

积是泛构造的一个简单例子。我们选择两个对象 $a$ 和 $b$,然后看是否存在一个对象 $c$,以及一对态射 $p$ 和 $q$,具有作为它们积的泛性质 (universal property)。

\begin{figure}[H]
  \centering
  \includegraphics[width=0.3\textwidth]{images/productranking.jpg}
\end{figure}

\noindent
积是极限的一个特例。极限是根据锥 (cones) 定义的。一般的锥是由可换图构建的。这些图的可换性可以用函子映射的适当自然性条件 (naturality condition) 来替代。这样,可换性就被简化为更高级语言——自然变换 (natural transformations)——的汇编语言的角色。

\section{自然变换}

总的来说,每当我们需要一个从态射到可换方块 (commuting squares) 的映射时,自然变换都非常方便。自然性方块 (naturality square) 的两个相对边是某个态射 $f$ 在两个函子 $F$ 和 $G$ 下的映射。另外两边是自然变换的分量 (components)(它们也是态射)。

\begin{figure}[H]
  \centering
  \includegraphics[width=0.35\textwidth]{images/3_naturality.jpg}
\end{figure}

\noindent
自然性意味着当你移动到“相邻”分量时(相邻是指通过态射连接),你不会违背范畴或函子的结构。无论你是先使用自然变换的分量来弥合对象之间的差距,然后使用函子跳转到它的邻居;还是反过来,都没有关系。这两个方向是正交的。可以说,自然变换让你左右移动,而函子让你上下或前后移动。你可以将函子的 \emph{像 (image)} 想象成目标范畴中的一张“薄片 (sheet)”。自然变换将对应于 $F$ 的这样一张薄片映射到对应于 $G$ 的另一张薄片。

\begin{figure}[H]
  \centering
  \includegraphics[width=0.35\textwidth]{images/sheets.png}
\end{figure}

\noindent
我们在 Haskell 中已经看到了这种正交性的例子。在那里,函子的作用修改了容器的内容而不改变其形状,而自然变换则将未触及的内容重新包装到不同的容器中。这些操作的顺序无关紧要。

我们已经看到极限定义中的锥被自然变换所取代。自然性确保了每个锥的边都是可换的。尽管如此,极限是根据锥 \emph{之间} 的映射来定义的。这些映射也必须满足可换性条件。(例如,积定义中的三角形必须可换。)

这些条件也可以用自然性来替代。你可能还记得,\emph{泛} 锥,或者说极限,被定义为(逆变)hom-函子:
\[F \Colon c \to \cat{C}(c, \Lim[D])\]
和(同样是逆变的)将 \emph{C} 中的对象映射到锥(锥本身是自然变换)的函子之间的自然变换:
\[G \Colon c \to \mathit{Nat}(\Delta_c, D)\]
这里,$\Delta_c$ 是常数函子 (constant functor),而 $D$ 是在 $\cat{C}$ 中定义图表的函子。函子 $F$ 和 $G$ 在 $\cat{C}$ 中的态射上都有明确定义的作用。碰巧的是,这个 $F$ 和 $G$ 之间的特定自然变换是一个 \emph{同构 (isomorphism)}。

\section{自然同构}

自然同构 (natural isomorphism)——一个其每个分量都可逆的自然变换——是范畴论表达“两个事物相同”的方式。这种变换的分量必须是对象之间的同构——一个具有逆 (inverse) 的态射。如果你将函子的像想象成薄片,那么自然同构就是这些薄片之间的一一对应的可逆映射。

\section{Hom-集}

但是态射是什么?它们确实比对象有更多的结构:与对象不同,态射有两个端点。但是如果你固定了源对象和目标对象,两者之间的态射就形成了一个无聊的集合(至少对于局部小范畴 (locally small categories) 来说)。我们可以给这个集合中的元素命名,比如 $f$ 或 $g$,以区分彼此——但究竟是什么使它们不同呢?

给定 hom-集中态射之间的本质区别在于它们与其他(来自相邻 hom-集)态射复合 (composition) 的方式。如果存在一个态射 $h$,其与 $f$ 的(前或后)复合不同于与 $g$ 的复合,例如:
\[h \circ f \neq h \circ g\]
那么我们就可以直接“观察”到 $f$ 和 $g$ 之间的差异。但即使差异不能直接观察到,我们也可以使用函子来放大观察 hom-集。一个函子 $F$ 可能将这两个态射映射到不同的态射:
\[F f \neq F g\]
在一个更丰富的范畴中,相邻的 hom-集提供了更高的分辨率,例如,
\[h' \circ F f \neq h' \circ F g\]
其中 $h'$ 不在 $F$ 的像中。

\section{Hom-集同构}

许多范畴构造都依赖于 hom-集之间的同构。但由于 hom-集只是集合,它们之间的普通同构并不能告诉你太多信息。对于有限集,同构只是说它们有相同数量的元素。如果集合是无限的,它们的基数 (cardinality) 必须相同。但是任何有意义的 hom-集同构都必须考虑复合。而复合涉及不止一个 hom-集。我们需要定义跨越整个 hom-集集合的同构,并且需要施加一些与复合相互作用的兼容性条件。而 \newterm{自然 (natural)} 同构恰好满足这些要求。

但是 hom-集的自然同构是什么?自然性是函子之间映射的属性,而不是集合的属性。所以我们实际上是在谈论以 hom-集为值的函子 (hom-set-valued functors) 之间的自然同构。这些函子不仅仅是以集合为值的函子。它们对态射的作用是由相应的 hom-函子诱导的。态射通过 hom-函子使用前复合或后复合(取决于函子的协变性)进行规范映射。

Yoneda 嵌入 (Yoneda embedding) 就是这种同构的一个例子。它将 $\cat{C}$ 中的 hom-集映射到函子范畴中的 hom-集;并且它是自然的。Yoneda 嵌入中的一个函子是 $\cat{C}$ 中的 hom-函子,另一个函子将对象映射到 hom-集之间自然变换的集合。

极限的定义也是 hom-集之间的自然同构(同样,第二个是在函子范畴中):
\[\cat{C}(c, \Lim[D]) \simeq \mathit{Nat}(\Delta_c, D)\]
事实证明,我们构造的指数对象或自由幺半群也可以重写为 hom-集之间的自然同构。

这并非巧合——我们接下来将看到,这些只是伴随的不同例子,而伴随正是定义为 hom-集之间的自然同构。

\section{Hom-集的不对称性}

还有一个观察将有助于我们理解伴随。Hom-集通常是不对称的。一个 hom-集 $\cat{C}(a, b)$ 通常与 hom-集 $\cat{C}(b, a)$ 非常不同。这种不对称性的最终体现是视为范畴的偏序 (partial order)。在偏序中,从 $a$ 到 $b$ 的态射存在当且仅当 $a$ 小于或等于 $b$。如果 $a$ 和 $b$ 不同,那么就不可能有反向的态射,即从 $b$ 到 $a$。因此,如果 hom-集 $\cat{C}(a, b)$ 非空,在这种情况下意味着它是一个单元集 (singleton set),那么 $\cat{C}(b, a)$ 必须为空,除非 $a = b$。这个范畴中的箭头有一个明确的单向流。

预序 (preorder) 基于一种不一定是反对称的 (antisymmetric) 关系,它也是“基本上”定向的,除了偶尔的循环。将任意范畴视为预序的推广是很方便的。

预序是一个瘦范畴 (thin category)——所有的 hom-集要么是单元集,要么是空集。我们可以将一般范畴想象成一个“厚的”预序。

\section{挑战}

\begin{enumerate}
  \tightlist
  \item
        考虑自然性条件的一些退化情况,并绘制相应的图表。例如,如果函子 $F$ 或 $G$ 将对象 $a$ 和 $b$(态射 $f \Colon a \to b$ 的两端)都映射到同一个对象,例如 $F a = F b$ 或 $G a = G b$,会发生什么?(注意,这样你会得到一个锥或余锥。)然后考虑 $F a = G a$ 或 $F b = G b$ 的情况。最后,如果你从一个自循环的态射开始——$f \Colon a \to a$,会怎样?
\end{enumerate}