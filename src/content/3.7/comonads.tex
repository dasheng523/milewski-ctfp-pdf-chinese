% !TEX root = ../../ctfp-print.tex

\lettrine[lhang=0.17]{现}{在我们已经讨论了} monads (单子),我们可以利用 duality (对偶性) 的好处,通过反转箭头并在 opposite category (对偶范畴) 中工作,免费获得 comonads (余单子)。

回想一下,在最基本的层面上,monad 是关于组合 Kleisli arrows (Kleisli 箭头) 的:

\src{snippet01}
其中 \code{m} 是一个作为 monad 的 functor (函子)。如果我们用字母 \code{w}(\code{m} 的倒置)表示 comonad,我们可以将 co-Kleisli arrows (余 Kleisli 箭头) 定义为类型为以下形式的 morphism (态射):

\src{snippet02}
用于余 Kleisli 箭头的鱼形运算符的类似物定义为:

\src{snippet03}
为了让余 Kleisli 箭头构成一个 category (范畴),我们还需要有一个 identity co-Kleisli arrow (单位余 Kleisli 箭头),称为 \code{extract} (提取):

\src{snippet04}
这是 \code{return} (返回) 的对偶。我们还必须强制施加 associativity (结合律) 以及 left- and right-identity (左单位元和右单位元) 定律。综合起来,我们可以在 Haskell 中这样定义一个 comonad:

\src{snippet05}
在实践中,我们使用稍微不同的原语,稍后我们会看到。

问题是,comonad 在编程中有什么用处?

\section{使用 Comonad 编程}

让我们比较一下 monad 和 comonad。Monad 提供了一种使用 \code{return} 将值放入 container (容器) 的方法。它不提供对存储在内部的一个或多个值的访问。当然,实现 monad 的数据结构可能会提供对其内容的访问,但这被认为是额外的功能。没有通用的接口用于从 monad 中提取值。我们已经看到了 \code{IO} monad (IO 单子) 的例子,它以从不暴露其内容而自豪。

另一方面,Comonad 提供了从中提取单个值的方法。它不提供插入值的方法。所以如果你想把 comonad 看作一个容器,它总是预先填充好内容,并且允许你查看它。

就像 Kleisli 箭头接受一个值并产生一些经过修饰的结果——它用 context (上下文) 来修饰它——余 Kleisli 箭头接受一个值连同整个上下文,并产生一个结果。它是 \newterm{contextual computation (上下文计算)} 的体现。

\section{The Product Comonad (积余单子)}

还记得 reader monad (Reader 单子) 吗?我们引入它是为了解决实现需要访问某些只读 environment (环境) \code{e} 的计算问题。这样的计算可以表示为以下形式的 pure functions (纯函数):

\src{snippet06}
我们使用 currying (柯里化) 将它们转换为 Kleisli 箭头:

\src{snippet07}
但请注意,这些函数已经具有余 Kleisli 箭头的形式。让我们将它们的参数整理成更方便的函子形式:

\src{snippet08}
我们可以通过使相同的环境对我们正在组合的箭头可用,来轻松定义组合运算符:

\src{snippet09}
\code{extract} 的实现只是忽略了环境:

\src{snippet10}
毫不奇怪,product comonad (积余单子) 可以用来执行与 Reader monad 完全相同的计算。在某种程度上,环境的 comonadic (余单子) 实现更自然——它遵循了“上下文中的计算”的精神。另一方面,monad 带有方便的 \code{do} notation (`do` 表示法) 的 syntactic sugar (语法糖)。

Reader monad 和 product comonad 之间的联系更深层,与 Reader functor (Reader 函子) 是 product functor (积函子) 的 right adjoint (右伴随) 这一事实有关。不过,总的来说,comonad 涵盖了与 monad 不同的计算概念。我们稍后会看到更多例子。

很容易将 \code{Product} comonad 推广到任意的 product types (积类型),包括 tuples (元组) 和 records (记录)。

\section{剖析组合}

继续 dualization (对偶化) 的过程,我们可以继续对偶化 monadic bind (单子绑定) 和 join (连接)。或者,我们可以重复我们用于 monad 的过程,即研究鱼形运算符的结构。这种方法似乎更具启发性。

出发点是认识到组合运算符必须产生一个接受 \code{w a} 并产生 \code{c} 的余 Kleisli 箭头。产生 \code{c} 的唯一方法是将第二个函数应用于类型为 \code{w b} 的参数:

\src{snippet11}
但是我们如何产生一个可以馈送给 \code{g} 的 \code{w b} 类型的值呢?我们手头有类型为 \code{w a} 的参数和函数 \code{f :: w a -> b}。解决方案是定义 bind 的对偶,称为 extend (扩展):

\src{snippet12}
使用 \code{extend} 我们可以实现组合:

\src{snippet13}
接下来我们能剖析 \code{extend} 吗?你可能会想说,为什么不直接将函数 \code{w a -> b} 应用于参数 \code{w a},但你很快就会意识到你无法将结果 \code{b} 转换为 \code{w b}。记住,comonad 不提供提升值的方法。在这一点上,对于 monad 的类似构造,我们使用了 \code{fmap} (映射)。这里我们能使用 \code{fmap} 的唯一方法是,如果我们手头有类型为 \code{w (w a)} 的东西。如果我们能把 \code{w a} 变成 \code{w (w a)} 就好了。而且,方便的是,这恰好是 \code{join} 的对偶。我们称之为 \code{duplicate} (复制):

\src{snippet14}
所以,就像 monad 的定义一样,我们有三种等价的 comonad 定义:使用余 Kleisli 箭头、\code{extend} 或 \code{duplicate}。以下是直接取自 \code{Control.Comonad} 库的 Haskell 定义:

\src{snippet15}
提供了 \code{extend} 根据 \code{duplicate} 实现的默认实现,反之亦然,所以你只需要覆盖其中一个。

这些函数背后的直觉基于这样的想法:通常,可以将 comonad 视为填充了 \code{a} 类型值的容器(积余单子只是一个特殊情况,只有一个值)。存在一个“当前”值的概念,可以通过 \code{extract} 轻松访问。余 Kleisli 箭头执行一些计算,该计算聚焦于当前值,但它可以访问所有周围的值。想想 Conway's game of life (康威生命游戏)。每个 cell (单元) 包含一个值(通常只是 \code{True} 或 \code{False})。对应于生命游戏的 comonad 将是一个聚焦于“当前”单元的 cell grid (单元格网格)。

那么 \code{duplicate} 是做什么的呢?它接受一个 comonadic container (余单子容器) \code{w a} 并产生一个容器的容器 \code{w (w a)}。其思想是,这些容器中的每一个都聚焦于 \code{w a} 内部不同的 \code{a}。在生命游戏中,你会得到一个网格的网格,外部网格的每个单元包含一个内部网格,该内部网格聚焦于不同的单元。

现在看看 \code{extend}。它接受一个余 Kleisli 箭头和一个填充了 \code{a} 的余单子容器 \code{w a}。它将计算应用于所有这些 \code{a},用 \code{b} 替换它们。结果是一个填充了 \code{b} 的余单子容器。\code{extend} 通过将 focus (焦点) 从一个 \code{a} 转移到另一个,并依次将余 Kleisli 箭头应用于它们中的每一个来实现这一点。在生命游戏中,余 Kleisli 箭头将计算当前单元的新状态。为此,它会查看其上下文——大概是其最近的邻居。 \code{extend} 的默认实现说明了这个过程。首先我们调用 \code{duplicate} 来产生所有可能的焦点,然后我们将 \code{f} 应用于它们中的每一个。

\section{The Stream Comonad (流余单子)}

这种将焦点从容器的一个元素转移到另一个元素的过程,最好用 infinite stream (无限流) 的例子来说明。这种流就像一个列表,只是它没有空构造器:

\src{snippet16}
它显然是一个 \code{Functor} (函子):

\src{snippet17}
流的焦点是它的第一个元素,所以这里是 \code{extract} 的实现:

\src{snippet18}
\code{duplicate} 产生一个流的流,每个流聚焦于不同的元素。

\src{snippet19}
第一个元素是原始流,第二个元素是原始流的尾部,第三个元素是它的尾部,依此类推,无限进行。

这是完整的实例:

\src{snippet20}
这是一种看待流的非常函数式的方式。在命令式语言中,我们可能会从一个将流移动一个位置的方法 \code{advance} (前进) 开始。在这里,\code{duplicate} 一举产生了所有移动后的流。Haskell 的 laziness (惰性) 使这成为可能,甚至是可取的。当然,为了使 \code{Stream} 实用,我们也会实现 \code{advance} 的类似物:

\src{snippet21}
但它从来不是 comonadic 接口的一部分。

如果你有任何 digital signal processing (数字信号处理) 的经验,你会立刻看出流的余 Kleisli 箭头就是一个 digital filter (数字滤波器),而 \code{extend} 产生一个过滤后的流。

作为一个简单的例子,让我们实现 moving average filter (移动平均滤波器)。这是一个对流的 \code{n} 个元素求和的函数:

\src{snippet22}
这是一个计算流的前 \code{n} 个元素平均值的函数:

\src{snippet23}
部分应用的 \code{average n} 是一个余 Kleisli 箭头,所以我们可以将其 \code{extend} 到整个流上:

\src{snippet24}
结果是 running averages (移动平均) 的流。

流是 unidirectional (单向)、one-dimensional (一维) comonad 的一个例子。它可以很容易地变成 bidirectional (双向) 或扩展到二维或更多维度。

\section{Comonad Categorically (范畴论中的 Comonad)}

在范畴论中定义 comonad 是对偶性的直接实践。与 monad 一样,我们从一个 endofunctor (自函子) \code{T} 开始。定义 monad 的两个 natural transformations (自然变换) $\eta$ 和 $\mu$,对于 comonad 只是简单地反转:
\begin{align*}
  \varepsilon & \Colon T \to I   \\
  \delta      & \Colon T \to T^2
\end{align*}
这些变换的分量对应于 \code{extract} 和 \code{duplicate}。Comonad laws (余单子定律) 是 monad laws (单子定律) 的镜像。这里没什么大惊喜。

然后是从 adjunction (伴随) 推导 monad。对偶性反转了伴随:left adjoint (左伴随) 变成 right adjoint (右伴随),反之亦然。并且,由于组合 $R \circ L$ 定义了一个 monad,那么 $L \circ R$ 必须定义一个 comonad。伴随的 counit (余单位):
\[\varepsilon \Colon L \circ R \to I\]
确实与我们在 comonad 定义中看到的 $\varepsilon$ 相同——或者,在分量中,就是 Haskell 的 \code{extract}。我们也可以使用伴随的 unit (单位):
\[\eta \Colon I \to R \circ L\]
在 $L \circ R$ 的中间插入一个 $R \circ L$ 并产生 $L \circ R \circ L \circ R$。从 $T$ 构造出 $T^2$ 定义了 $\delta$,这就完成了 comonad 的定义。

我们还看到 monad 是一个 monoid (幺半群)。这个陈述的对偶将需要使用 comonoid (余幺半群),那么什么是 comonoid 呢?将幺半群原始定义为单对象范畴并不能对偶化成任何有趣的东西。当你反转所有自同态的方向时,你得到另一个幺半群。然而,回想一下,在我们处理 monad 的方法中,我们使用了幺半群的一个更一般的定义,即作为 monoidal category (幺半范畴) 中的一个对象。该构造基于两个态射:
\begin{align*}
  \mu  & \Colon m \otimes m \to m \\
  \eta & \Colon i \to m
\end{align*}
这些态射的反转产生了幺半范畴中的 comonoid:
\begin{align*}
  \delta      & \Colon m \to m \otimes m \\
  \varepsilon & \Colon m \to i
\end{align*}
我们可以在 Haskell 中写一个 comonoid 的定义:

\src{snippet25}
但它相当微不足道。显然 \code{destroy} 忽略了它的参数。

\src{snippet26}
\code{split} 只是一对函数:

\src{snippet27}
现在考虑与幺半群单位元定律对偶的 comonoid 定律。

\src{snippet28}
这里,\code{lambda} 和 \code{rho} 分别是左单位子和右单位子(参见 \hyperref[monads-categorically]{幺半范畴} 的定义)。代入定义,我们得到:

\src{snippet29}
这证明了 \code{g = id}。类似地,第二条定律展开为 \code{f = id}。总之:

\src{snippet30}
这表明在 Haskell 中(并且,一般地,在范畴 $\Set$ 中)每个对象都是一个平凡的 comonoid。

幸运的是,还有其他更有趣的幺半范畴可以在其中定义 comonoid。其中之一是自函子范畴。事实证明,就像 monad 是自函子范畴中的幺半群一样,

\begin{quote}
  Comonad 是自函子范畴中的 comonoid。
\end{quote}

\section{The Store Comonad (存储余单子)}

另一个重要的 comonad 例子是 state monad (状态单子) 的对偶。它被称为 costate comonad (余状态余单子) 或者,或者说,store comonad (存储余单子)。

我们之前已经看到状态单子是由定义 exponentials (指数对象) 的伴随生成的:
\begin{align*}
  L z & = z\times{}s      \\
  R a & = s \Rightarrow a
\end{align*}
我们将使用相同的伴随定义 costate comonad。Comonad 由组合 $L \circ R$ 定义:
\[L (R a) = (s \Rightarrow a)\times{}s\]
将其转换为 Haskell,我们从左边的 \code{Product} functor (积函子) 和右边的 \code{Reader} functor (Reader 函子) 之间的伴随开始。组合 \code{Product} 在 \code{Reader} 之后等价于以下定义:

\src{snippet31}
在对象 $a$ 处取伴随的余单位是态射:
\[\varepsilon_a \Colon ((s \Rightarrow a)\times{}s) \to a\]
或者,用 Haskell 表示法:

\src{snippet32}
这成为我们的 \code{extract}:

\src{snippet33}
伴随的单位:

\src{snippet34}
可以重写为部分应用的数据构造器:

\src{snippet35}
我们构造 $\delta$,或 \code{duplicate},作为水平组合:
\begin{align*}
  \delta & \Colon L \circ R \to L \circ R \circ L \circ R \\
  \delta & = L \circ \eta \circ R
\end{align*}
我们必须将 $\eta$ “塞”过最左边的 $L$,也就是 \code{Product} 函子。这意味着用 $\eta$,即 \code{Store f},作用于对的左分量(这就是 \code{Product} 的 \code{fmap} 会做的)。我们得到:

\src{snippet36}
(记住,在 $\delta$ 的公式中,$L$ 和 $R$ 代表其分量为恒等态射的恒等自然变换。)

这是 \code{Store} comonad 的完整定义:

\src{snippet37}
你可以将 \code{Store} 的 \code{Reader} 部分视为一个广义的 \code{a} 的容器,这些 \code{a} 使用 \code{s} 类型的值作为键。例如,如果 \code{s} 是 \code{Int},\code{Reader Int a} 就是一个无限的双向 \code{a} 流。\code{Store} 将这个容器与一个键类型的值配对。例如,\code{Reader Int a} 与一个 \code{Int} 配对。在这种情况下,\code{extract} 使用这个整数来索引到无限流中。你可以将 \code{Store} 的第二个分量视为当前位置。

继续这个例子,\code{duplicate} 创建了一个新的由 \code{Int} 索引的无限流。这个流包含流作为其元素。特别地,在当前位置,它包含原始流。但是如果你使用其他一些 \code{Int}(正数或负数)作为键,你会得到一个位于该新索引处的移动后的流。

总的来说,你可以说服自己,当 \code{extract} 作用于 \code{duplicate} 后的 \code{Store} 时,它会产生原始的 \code{Store}(事实上,comonad 的单位元定律指出 \code{extract . duplicate = id})。

\code{Store} comonad 作为 \code{Lens} library (Lens 库) 的理论基础扮演着重要角色。从概念上讲,\code{Store s a} comonad 封装了使用类型 \code{s} 作为 index (索引) 来“聚焦” (focusing) (像透镜一样)于数据类型 \code{a} 的特定 substructure (子结构) 的思想。特别是,类型为以下形式的函数:

\src{snippet38}
等价于一对函数:

\src{snippet39}
如果 \code{a} 是一个 product type (积类型),\code{set} 可以实现为设置 \code{a} 内部类型为 \code{s} 的字段,同时返回修改后的 \code{a} 版本。类似地,\code{get} 可以实现为从 \code{a} 中读取 \code{s} 字段的值。我们将在下一节更深入地探讨这些想法。

\section{挑战}

\begin{enumerate}
  \tightlist
  \item
        使用 \code{Store} comonad 实现 Conway's Game of Life (康威生命游戏)。提示:你为 \code{s} 选择什么类型?
\end{enumerate}