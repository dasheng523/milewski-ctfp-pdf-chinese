% !TEX root = ../../ctfp-print.tex

\lettrine[lhang=0.17]{既}{然你知道了}什么是 functor (函子),并且看到了一些例子,让我们看看如何用较小的函子构建较大的函子。特别有趣的是,看看哪些 type constructors (类型构造器)(它们对应范畴中 objects (对象) 之间的映射)可以扩展为函子(函子包含 morphisms (态射) 之间的映射)。

\section{双函子}

由于函子是 $\Cat$(范畴的范畴)中的态射,许多关于态射——特别是 functions (函数)——的直觉也适用于函子。例如,就像你可以有接受两个参数的函数一样,你也可以有接受两个参数的函子,或者一个 \newterm{bifunctor} (双函子)。在对象上,双函子将来自范畴 $\cat{C}$ 的一个对象和来自范畴 $\cat{D}$ 的一个对象的每一对,映射到范畴 $\cat{E}$ 中的一个对象。注意,这只是说它是从范畴的 \newterm{Cartesian product} (笛卡尔积) $\cat{C}\times{}\cat{D}$ 到 $\cat{E}$ 的一个映射。

\begin{figure}[H]
  \centering\includegraphics[width=0.3\textwidth]{images/bifunctor.jpg}
\end{figure}

\noindent
这相当直接。但是 functoriality (函子性) 意味着双函子也必须映射态射。不过,这次它必须将一对态射,一个来自 $\cat{C}$,一个来自 $\cat{D}$,映射到 $\cat{E}$ 中的一个态射。

再次强调,一对态射就是 product category (积范畴) $\cat{C}\times{}\cat{D}$ 中的单个态射。我们将范畴笛卡尔积中的态射定义为一对态射,它从一对对象映到另一对对象。这些态射对可以按显而易见的方式复合:
\[(f, g) \circ (f', g') = (f \circ f', g \circ g')\]
该复合是结合的,并且它有一个 identity (单位元)——一对 identity morphisms (单位态射) $(\id, \id)$。因此,范畴的笛卡尔积确实是一个范畴。

思考双函子的一种更简单的方式是,将它们视为在每个参数上分别都是函子。因此,与其将 functorial laws (函子定律)——结合律和单位元保持——从函子转换到双函子,不如分别对每个参数进行检查就足够了。然而,一般情况下,分别的函子性不足以证明 joint functoriality (联合函子性)。联合函子性失败的范畴被称为 \newterm{premonoidal} (前幺半群的)。

让我们在 Haskell 中定义一个双函子。在这种情况下,所有三个范畴都是相同的:Haskell 类型的范畴。双函子是接受两个类型参数的类型构造器。这是直接取自库 \code{Control.Bifunctor} 的 \code{Bifunctor} typeclass (类型类) 定义:

\src{snippet01}

\begin{figure}[H]
  \centering\includegraphics[width=0.3\textwidth]{images/bimap.jpg}
  \caption{bimap}
\end{figure}

类型变量 \code{f} 代表双函子。你可以看到,在所有类型签名中,它总是应用于两个类型参数。第一个类型签名定义了 \code{bimap}:同时映射两个函数。结果是一个 lifted function (提升后的函数),\code{(f a b -> f c d)},作用于由双函子的类型构造器生成的类型上。有一个基于 \code{first} 和 \code{second} 的 \code{bimap} 默认实现。(如前所述,这并不总是有效,因为这两个映射可能不可 commute (交换),即 \code{first g . second h} 可能与 \code{second h . first g} 不同。)


\noindent
另外两个类型签名,\code{first} 和 \code{second},是两个 \code{fmap},分别证明了 \code{f} 在第一个和第二个参数上的函子性。

\begin{figure}[H]
  \centering
  \begin{minipage}{0.45\textwidth}
    \centering
    \includegraphics[width=0.65\textwidth]{images/first.jpg} % first figure itself
    \caption{first}
  \end{minipage}\hfill
  \begin{minipage}{0.45\textwidth}
    \centering
    \includegraphics[width=0.6\textwidth]{images/second.jpg} % second figure itself
    \caption{second}
  \end{minipage}
\end{figure}

\noindent
类型类定义为它们两者都提供了基于 \code{bimap} 的默认实现。

在声明 \code{Bifunctor} 的 instance (实例) 时,你可以选择实现 \code{bimap} 并接受 \code{first} 和 \code{second} 的默认值,或者同时实现 \code{first} 和 \code{second} 并接受 \code{bimap} 的默认值(当然,你可以实现所有三个,但那样就需要你确保它们之间以这种方式相关联)。

\section{积和余积双函子}

双函子的一个重要例子是 categorical product (范畴积)——由 \hyperref[products-and-coproducts]{universal construction} (泛构造) 定义的两个对象的积。如果对于任何一对对象都存在积,那么从这些对象到积的映射是 bifunctorial (双函子性的)。这通常是正确的,特别是在 Haskell 中。以下是 pair constructor (序对构造器)——最简单的 product type (积类型)——的 \code{Bifunctor} 实例:

\src{snippet02}[b]
选择不多:\code{bimap} 简单地将第一个函数应用于序对的第一个分量,将第二个函数应用于第二个分量。给定类型,代码几乎是自明的:

\src{snippet03}
这里的双函子的作用是构造类型对,例如:

\begin{snip}{haskell}
(,) a b = (a, b)
\end{snip}
根据 duality (对偶性),如果在一个范畴中每个对象对都定义了 coproduct (余积),那么它也是一个双函子。在 Haskell 中,这通过 \code{Either} 类型构造器是 \code{Bifunctor} 的实例来体现:

\src{snippet04}[b]
这段代码也是自明的。

现在,还记得我们讨论 monoidal categories (幺半群范畴) 吗?幺半群范畴定义了一个作用于对象的 binary operator (二元运算符),以及一个 unit object (单位对象)。我提到过 $\Set$ 对于笛卡尔积是一个幺半群范畴,以 singleton set (单例集) 为单位元。并且它对于 disjoint union (不相交并) 也是一个幺半群范畴,以 empty set (空集) 为单位元。我没有提到的是,幺半群范畴的要求之一是二元运算符必须是一个双函子。这是一个非常重要的要求——我们希望 monoidal product (幺半群积) 与由态射定义的范畴结构兼容。我们现在离幺半群范畴的完整定义又近了一步(在到达那里之前,我们还需要学习 naturality (自然性))。

\section{函子性的代数数据类型}

我们已经看到了几个参数化数据类型的例子,它们结果是函子——我们能够为它们定义 \code{fmap}。复杂的数据类型由更简单的数据类型构建而成。特别是,algebraic data types (\acronym{ADT}, 代数数据类型) 是使用 sums (和) 与 products (积) 创建的。我们刚刚看到和与积是函子性的。我们也知道函子可以 compose (复合)。因此,如果我们能证明 \acronym{ADT} 的基本构建块是函子性的,我们就会知道参数化的 \acronym{ADT} 也是函子性的。

那么参数化代数数据类型的构建块是什么?首先,有些项不依赖于函子的类型参数,比如 \code{Maybe} 中的 \code{Nothing},或者 \code{List} 中的 \code{Nil}。它们等价于 \code{Const} 函子。记住,\code{Const} 函子忽略其类型参数(实际上,是 \emph{第二个} 类型参数,这是我们感兴趣的那个,第一个保持不变)。

然后,有些元素只是简单地 encapsulate (封装) 了类型参数本身,比如 \code{Maybe} 中的 \code{Just}。它们等价于 the identity functor (恒等函子)。我之前提到过恒等函子,作为 \emph{Cat} 中的单位态射,但没有给出它在 Haskell 中的定义。这里是:

\src{snippet05}

\src{snippet06}
你可以将 \code{Identity} 视为最简单的可能容器,它总是只存储一个类型为 \code{a} 的(不可变)值。

代数数据结构中的其他一切都是由这两个 primitives (原语) 使用积和和构造出来的。

有了这些新知识,让我们重新审视 \code{Maybe} 类型构造器:

\src{snippet07}
它是两种类型的和,我们现在知道和是函子性的。第一部分 \code{Nothing} 可以表示为作用于 \code{a} 的 \code{Const ()}(\code{Const} 的第一个类型参数设置为 unit (单位类型)——稍后我们会看到 \code{Const} 更有趣的用法)。第二部分只是恒等函子的不同名称。我们本可以将 \code{Maybe} 定义为(在 isomorphism (同构) 意义下):

\src{snippet08}
所以 \code{Maybe} 是双函子 \code{Either} 与两个函子 \code{Const ()} 和 \code{Identity} 的复合。(\code{Const} 实际上是一个双函子,但在这里我们总是使用它 partially applied (部分应用) 的形式。)

我们已经看到函子的复合是函子——我们很容易说服自己,对于双函子也是如此。我们只需要弄清楚一个双函子与两个函子的复合如何在态射上工作。给定两个态射,我们只需用一个函子提升一个态射,用另一个函子提升另一个。然后我们用双函子提升这对提升后的态射。

我们可以在 Haskell 中表达这种复合。让我们定义一个数据类型,它由一个双函子 \code{bf}(它是一个接受两个类型作为参数的类型构造器的类型变量)、两个函子 \code{fu} 和 \code{gu}(每个都接受一个类型变量的类型构造器)以及两个常规类型 \code{a} 和 \code{b} 参数化。我们将 \code{fu} 应用于 \code{a},将 \code{gu} 应用于 \code{b},然后将 \code{bf} 应用于得到的两个类型:

\src{snippet09}
这是在对象或类型上的复合。注意在 Haskell 中我们如何将类型构造器应用于类型,就像我们将函数应用于参数一样。语法是相同的。

如果你有点迷糊,尝试按此顺序将 \code{BiComp} 应用于 \code{Either}、\code{Const ()}、\code{Identity}、\code{a} 和 \code{b}。你将恢复我们 \code{Maybe b} 的 bare-bone (简化) 版本(\code{a} 被忽略了)。

新数据类型 \code{BiComp} 在 \code{a} 和 \code{b} 上是一个双函子,但前提是 \code{bf} 本身是一个 \code{Bifunctor} 并且 \code{fu} 和 \code{gu} 是 \code{Functor}s。编译器必须知道 \code{bf} 会有一个可用的 \code{bimap} 定义,并且 \code{fu} 和 \code{gu} 会有 \code{fmap} 的定义。在 Haskell 中,这在实例声明中表示为一个 precondition (前提条件):一组 class constraints (类约束) 后面跟着一个双箭头:

\src{snippet10}[b]
\code{BiComp} 的 \code{bimap} 实现是根据 \code{bf} 的 \code{bimap} 以及 \code{fu} 和 \code{gu} 的两个 \code{fmap} 给出的。当使用 \code{BiComp} 时,编译器会自动推断所有类型并选择正确的 overloaded (重载) 函数。

\code{bimap} 定义中的 \code{x} 具有以下类型:

\src{snippet11}
这相当冗长。外层的 \code{bimap} 突破了外层的 \code{bf} 层,而两个 \code{fmap} 分别深入到 \code{fu} 和 \code{gu} 之下。如果 \code{f1} 和 \code{f2} 的类型是:

\src{snippet12}
那么最终结果的类型是 \code{bf (fu a') (gu b')}:

\src{snippet13}[b]
如果你喜欢拼图游戏,这类类型操作可以提供数小时的娱乐。

所以事实证明,我们不必证明 \code{Maybe} 是一个函子——这个事实源于它被构造为两个函子性原语的和的方式。

敏锐的读者可能会问:如果代数数据类型的 \code{Functor} 实例的 derivation (派生) 如此机械化,难道不能由编译器自动化并执行吗?确实可以,而且正是如此。你需要通过在源文件顶部包含以下行来启用一个特定的 Haskell extension (扩展):

\begin{snip}{haskell}
{-# LANGUAGE DeriveFunctor #-}
\end{snip}
然后在你的数据结构中添加 \code{deriving Functor}:

\begin{snip}{haskell}
data Maybe a = Nothing | Just a deriving Functor
\end{snip}
相应的 \code{fmap} 就会为你实现。

代数数据结构的规律性使得不仅可以派生 \code{Functor} 的实例,还可以派生其他几个类型类的实例,包括我之前提到的 \code{Eq} 类型类。还有一个选项是教编译器为自定义的类型类派生实例,但这有点进阶。不过想法是相同的:你为基本构建块以及和与积提供行为,让编译器找出其余部分。

\section{C++ 中的函子}

如果你是 C++ programmer (程序员),那么在实现函子方面,你显然需要自己动手。然而,你应该能够在 C++ 中识别出某些类型的代数数据结构。如果这样的数据结构被做成了 generic template (泛型模板),你应该能够快速地为它实现 \code{fmap}。

让我们看一个 tree data structure (树数据结构),我们在 Haskell 中会将其定义为一个 recursive sum type (递归和类型):

\src{snippet14}
正如我之前提到的,在 C++ 中实现和类型的一种方法是通过 class hierarchies (类层次结构)。在 object-oriented language (面向对象的语言) 中,将 \code{fmap} 实现为基类 \code{Functor} 的 virtual function (虚函数),然后在所有子类中 override (覆盖) 它,是很自然的做法。不幸的是,这是不可能的,因为 \code{fmap} 是一个模板,不仅由它作用的对象类型(\code{this} 指针)参数化,还由应用于它的函数的返回类型参数化。虚函数在 C++ 中不能是 templatized (模板化的)。我们将 \code{fmap} 实现为一个泛型自由函数,并且我们将用 \code{dynamic\_cast} 替换 pattern matching (模式匹配)。

基类必须至少定义一个虚函数以支持动态转换,所以我们将 destructor (析构函数) 设为虚函数(无论如何这都是一个好主意):

\begin{snip}{cpp}
template<class T>
struct Tree {
    virtual ~Tree() {}
};
\end{snip}
\code{Leaf} 只是伪装的 \code{Identity} 函子:

\begin{snip}{cpp}
template<class T>
struct Leaf : public Tree<T> {
    T _label;
    Leaf(T l) : _label(l) {}
};
\end{snip}
\code{Node} 是一个积类型:

\begin{snip}{cpp}
template<class T>
struct Node : public Tree<T> {
    Tree<T> * _left;
    Tree<T> * _right;
    Node(Tree<T> * l, Tree<T> * r) : _left(l), _right(r) {}
};
\end{snip}
在实现 \code{fmap} 时,我们利用了对 \code{Tree} 类型的 dynamic dispatching (动态分派)。\code{Leaf} 的情况应用了 \code{Identity} 版本的 \code{fmap},而 \code{Node} 的情况则被视为一个双函子与 \code{Tree} 函子的两个副本的复合。作为 C++ 程序员,你可能不习惯用这些术语来分析代码,但这是 categorical thinking (范畴思维) 的一个很好的练习。

\begin{snip}{cpp}
template<class A, class B>
Tree<B> * fmap(std::function<B(A)> f, Tree<A> * t) {
    Leaf<A> * pl = dynamic_cast <Leaf<A>*>(t);
    if (pl)
        return new Leaf<B>(f (pl->_label));
    Node<A> * pn = dynamic_cast<Node<A>*>(t);
    if (pn)
        // 修正:明确模板参数以避免潜在的歧义
        return new Node<B>( fmap<A, B>(f, pn->_left) 
                          , fmap<A, B>(f, pn->_right)); 
    return nullptr;
}
\end{snip}
为简单起见,我决定忽略 memory (内存) 和 resource management (资源管理) 问题,但在 production code (生产代码) 中,你可能会使用 smart pointers (智能指针)(unique 或 shared,取决于你的策略)。

将其与 Haskell 的 \code{fmap} 实现进行比较:

\src{snippet15}
这个实现也可以由编译器自动派生。

\section{Writer 函子}

我承诺过我会回到我之前描述的 \hyperref[kleisli-categories]{Kleisli category} (Kleisli 范畴)。该范畴中的态射被表示为返回 \code{Writer} 数据结构的“embellished (装饰)”函数。

\src{snippet16}
我说过这种装饰与自函子有某种关联。确实,\code{Writer} 类型构造器在 \code{a} 上是函子性的。我们甚至不需要为它实现 \code{fmap},因为它只是一个简单的积类型。

但 Kleisli 范畴和函子之间的一般关系是什么?Kleisli 范畴作为一个范畴,定义了复合和单位元。让我提醒你,复合由 fish operator (鱼算子) 给出:

\src{snippet17}
单位态射由一个名为 \code{return} 的函数给出:

\src{snippet18}
事实证明,如果你足够长时间地观察这两个函数的类型(我的意思是,\emph{非常}长的时间),你可以找到一种组合它们的方法来产生一个具有正确类型签名以用作 \code{fmap} 的函数。像这样:

\src{snippet19}
在这里,鱼算子组合了两个函数:一个是熟悉的 \code{id},另一个是一个 lambda 表达式,它将 \code{return} 应用于 \code{f} 作用于 lambda 参数的结果。最难理解的部分可能是 \code{id} 的使用。鱼算子的参数不应该是一个接受“normal (普通)”类型并返回装饰类型的函数吗?嗯,不完全是。没人说 \code{a -> Writer b} 中的 \code{a} 必须是“普通”类型。它是一个 type variable (类型变量),所以它可以是任何东西,特别是它可以是一个装饰类型,比如 \code{Writer b}。

所以 \code{id} 将接受 \code{Writer a} 并将其转换为 \code{Writer a}。鱼算子将取出 \code{a} 的值并将其作为 \code{x} 传递给 lambda。在那里,\code{f} 会将其转换为 \code{b},而 \code{return} 会装饰它,使其成为 \code{Writer b}。总而言之,我们最终得到了一个接受 \code{Writer a} 并返回 \code{Writer b} 的函数,这正是 \code{fmap} 应该产生的。

注意这个论证非常通用:你可以用任何类型构造器替换 \code{Writer}。只要它支持鱼算子和 \code{return},你就可以定义 \code{fmap}。因此,Kleisli 范畴中的装饰总是一个函子。(不过,并非每个函子都会产生 Kleisli 范畴。)

你可能想知道我们刚刚定义的 \code{fmap} 是否与编译器用 \code{deriving Functor} 为我们派生的 \code{fmap} 相同。有趣的是,它们是相同的。这归因于 Haskell 实现 polymorphic functions (多态函数) 的方式。它被称为 \newterm{parametric polymorphism} (参数多态),并且是所谓的 \newterm{theorems for free} (免费定理) 的来源。其中一个定理说,如果对于给定的类型构造器存在一个 \code{fmap} 的实现,并且该实现保持单位元,那么它必须是唯一的。

\section{协变函子和逆变函子}

既然我们已经回顾了 writer 函子,让我们回到 reader 函子。它基于部分应用的函数箭头类型构造器:

\src{snippet20}
我们可以将其重写为 type synonym (类型别名):

\src{snippet21}
正如我们之前看到的,其 \code{Functor} 实例为:

\src{snippet22}
但就像 pair type constructor (序对类型构造器) 或 \code{Either} 类型构造器一样,函数类型构造器接受两个类型参数。序对和 \code{Either} 在两个参数上都是函子性的——它们是双函子。那么函数构造器也是双函子吗?

让我们尝试让它在第一个参数上是函子性的。我们将从一个类型别名开始——它就像 \code{Reader},但参数翻转了:

\src{snippet23}
这次我们固定返回类型 \code{r},并改变参数类型 \code{a}。让我们看看是否能以某种方式匹配类型来实现 \code{fmap},它将具有以下类型签名:

\src{snippet24}
只有两个分别接受 \code{a} 并返回 \code{b} 和 \code{r} 的函数,根本无法构建一个接受 \code{b} 并返回 \code{r} 的函数!如果我们能以某种方式 invert (反转) 第一个函数,使其接受 \code{b} 并返回 \code{a},情况就会不同。我们不能反转任意函数,但我们可以转到 opposite category (对偶范畴)。

简要回顾:对于每个范畴 $\cat{C}$,都有一个对偶范畴 $\cat{C}^\mathit{op}$。它与 $\cat{C}$ 具有相同的对象,但所有 arrows (箭头) 都反转了。

考虑一个在 $\cat{C}^\mathit{op}$ 和某个其他范畴 $\cat{D}$ 之间的函子:
\[F \Colon \cat{C}^\mathit{op} \to \cat{D}\]
这样的函子将 $\cat{C}^\mathit{op}$ 中的态射 $f^\mathit{op} \Colon a \to b$ 映射到 $\cat{D}$ 中的态射 $F f^\mathit{op} \Colon F a \to F b$。但是态射 $f^\mathit{op}$ 秘密地对应于原始范畴 $\cat{C}$ 中的某个态射 $f \Colon b \to a$。注意这个 inversion (反转)。

现在,$F$ 是一个 regular functor (常规函子),但是我们可以基于 $F$ 定义另一个映射,它不是函子——我们称之为 $G$。它是从 $\cat{C}$ 到 $\cat{D}$ 的映射。它以与 $F$ 相同的方式映射对象,但是当涉及到映射态射时,它会反转它们。它接受 $\cat{C}$ 中的一个态射 $f \Colon b \to a$,首先将其映射到对偶态射 $f^\mathit{op} \Colon a \to b$,然后对其使用函子 $F$,得到 $F f^\mathit{op} \Colon F a \to F b$。

考虑到 $F a$ 与 $G a$ 相同, $F b$ 与 $G b$ 相同,整个过程可以描述为:$G f \Colon (b \to a) \to (G a \to G b)$。这是一个“带扭转的函子”。以这种方式反转态射方向的范畴映射称为 \emph{contravariant functor} (逆变函子)。注意,逆变函子只是来自对偶范畴的常规函子。顺便说一下,我们迄今为止研究的那种常规函子,被称为 \emph{covariant functor} (协变函子)。

\begin{figure}[H]
  \centering
  \includegraphics[width=40mm]{images/contravariant.jpg}
\end{figure}

\noindent
这是在 Haskell 中定义逆变函子(实际上是逆变 \emph{自}函子)的类型类:

\src{snippet25}
我们的类型构造器 \code{Op} 是它的一个实例:

\src{snippet26}
注意函数 \code{f} 将自身插入到 \code{Op} 的内容——函数 \code{g}——的 \emph{before} (前面)(即右边)。

如果你注意到 \code{Op} 的 \code{contramap} 定义只是参数翻转了的 function composition operator (函数复合运算符),那么它可以变得更加 terser (简洁)。有一个用于翻转参数的特殊函数,叫做 \code{flip}:

\src{snippet27}
用它,我们得到:

\src{snippet28}

\section{Profunctor}

我们已经看到 function-arrow operator (函数箭头运算符) 在其第一个参数上是逆变的,在第二个参数上是协变的。这种东西有名字吗?事实证明,如果目标范畴是 $\Set$,这种东西被称为 \newterm{profunctor} (Profunctor)。因为逆变函子等价于来自对偶范畴的协变函子,所以 Profunctor 被定义为:
\[\cat{C}^\mathit{op} \times \cat{D} \to \Set\]
由于 Haskell 类型近似于集合 (sets),我们将名称 \code{Profunctor} 应用于接受两个参数的类型构造器 \code{p},它在第一个参数上是 contra-functorial (逆变函子性的),在第二个参数上是 functorial (函子性的)。这是从 \code{Data.Profunctor} 库中取出的相应类型类:

\src{snippet29}[b]
所有三个函数都带有默认实现。就像 \code{Bifunctor} 一样,在声明 \code{Profunctor} 的实例时,你可以选择实现 \code{dimap} 并接受 \code{lmap} 和 \code{rmap} 的默认值,或者同时实现 \code{lmap} 和 \code{rmap} 并接受 \code{dimap} 的默认值。

\begin{figure}[H]
  \centering
  \includegraphics[width=0.4\textwidth]{images/dimap.jpg}
  \caption{dimap}
\end{figure}

\noindent
现在我们可以断言函数箭头运算符是 \code{Profunctor} 的一个实例:

\src{snippet30}[b]
Profunctor 在 Haskell lens library (透镜库) 中有应用。当我们讨论 ends and coends (端和余端) 时,我们会再次看到它们。

\section{Hom 函子}

上述例子反映了一个更普遍的陈述,即取一对对象 $a$ 和 $b$ 并为其分配它们之间的态射集合,即 hom-set (态射集合) $\cat{C}(a, b)$,这个映射是一个函子。它是从积范畴 $\cat{C}^\mathit{op}\times{}\cat{C}$ 到集合范畴 $\Set$ 的一个函子。

让我们定义它在态射上的作用。$\cat{C}^\mathit{op}\times{}\cat{C}$ 中的一个态射是来自 $\cat{C}$ 的一对态射:
\begin{gather*}
  f \Colon a' \to a \\
  g \Colon b \to b'
\end{gather*}
这对态射的提升必须是(一个函数)从集合 $\cat{C}(a, b)$ 到集合 $\cat{C}(a', b')$ 的一个态射。只需选择 $\cat{C}(a, b)$ 的任何元素 $h$(它是一个从 $a$ 到 $b$ 的态射),并将其赋值为:
\[g \circ h \circ f\]
这是 $\cat{C}(a', b')$ 的一个元素。

如你所见,the hom-functor (Hom 函子) 是 Profunctor 的一个特例。

\section{挑战}

\begin{enumerate}
  \tightlist
  \item
        证明数据类型:

        \begin{snip}{haskell}
data Pair a b = Pair a b
\end{snip}

        是一个双函子。作为加分项,实现 \code{Bifunctor} 的所有三个方法,并使用 equational reasoning (等式推理) 证明,在可以应用默认实现的情况下,这些定义与默认实现兼容。
  \item
        证明 \code{Maybe} 的标准定义与这个 desugaring (脱糖) 后的版本之间的同构:

        \begin{snip}{haskell}
type Maybe' a = Either (Const () a) (Identity a)
\end{snip}

        提示:在两个实现之间定义两个映射。作为加分项,使用等式推理证明它们互为 inverse (逆映射)。
  \item
        让我们尝试另一个数据结构。我称之为 \code{PreList},因为它是 \code{List} 的 precursor (前身)。它用类型参数 \code{b} 替换了 recursion (递归)。

        \begin{snip}{haskell}
data PreList a b = Nil | Cons a b
\end{snip}

        你可以通过将 \code{PreList} 递归地应用于自身来恢复我们之前的 \code{List} 定义(当我们讨论 fixed points (不动点) 时会看到如何做到这一点)。

        证明 \code{PreList} 是 \code{Bifunctor} 的一个实例。
  \item
        证明以下数据类型在 \code{a} 和 \code{b} 中定义了双函子:

        \begin{snip}{haskell}
data K2 c a b = K2 c

data Fst a b = Fst a

data Snd a b = Snd b
\end{snip}

        作为加分项,对照 Conor McBride 的论文 \urlref{http://strictlypositive.org/CJ.pdf}{Clowns to the Left of me, Jokers to the Right} 检查你的解决方案。
  \item
        在 Haskell 以外的语言中定义一个双函子。在该语言中为 generic pair (泛型序对) 实现 \code{bimap}。
  \item
        \code{std::map} 在两个模板参数 \code{Key} 和 \code{T} 中,应该被视为双函子还是 Profunctor?你会如何 redesign (重新设计) 这个数据类型使其如此?
\end{enumerate}