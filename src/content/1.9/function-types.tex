% !TEX root = ../../ctfp-print.tex

\lettrine[lhang=0.17]{到}{目前为止},我一直对 function types (函数类型) 的含义含糊其辞。函数类型与其他类型不同。

以 \code{Integer} 为例:它只是一个整数集合。\code{Bool} 是一个包含两个元素的集合。但是函数类型 $a\to b$ 不仅仅如此:它是对象 $a$ 和 $b$ 之间的 morphisms (态射) 集合。在任何范畴中,两个对象之间的态射集合称为 hom-set (态射集合)。恰好在范畴 $\Set$ 中,每个态射集合本身也是同一个范畴中的对象——因为它毕竟是一个 \emph{set} (集合)。

\begin{figure}[H]
  \centering
  \includegraphics[width=0.35\textwidth]{images/set-hom-set.jpg}
  \caption{Set 中的 Hom-set 只是一个集合}
\end{figure}

\noindent
在其他范畴中,情况并非如此,它们的态射集合是范畴外部的。它们甚至被称为 \emph{external} hom-set (外部态射集合)。

\begin{figure}[H]
  \centering
  \includegraphics[width=0.35\textwidth]{images/hom-set.jpg}
  \caption{范畴 C 中的 Hom-set 是一个外部集合}
\end{figure}

\noindent
正是 $\Set$ 范畴的这种自指性质使得函数类型变得特殊。但是,至少在某些范畴中,有一种方法可以构造代表态射集合的对象。这样的对象被称为 \newterm{internal} hom-set (内部态射集合)。

\section{泛构造}

让我们暂时忘记函数类型是集合,尝试从头开始构造一个函数类型,或者更一般地,一个内部态射集合。像往常一样,我们将从 $\Set$ 范畴中获取线索,但小心翼翼地避免使用集合的任何属性,以便该构造能够自动适用于其他范畴。

函数类型可以被认为是一种复合类型,因为它与参数类型和结果类型有关。我们已经见过复合类型的构造——那些涉及对象之间关系的构造。我们使用 universal constructions (泛构造) 定义了 \hyperref[products-and-coproducts]{product} (积) 和 coproduct types (余积类型)。我们可以使用同样的技巧来定义函数类型。我们将需要一个涉及三个对象的模式:我们正在构造的函数类型、参数类型和结果类型。

连接这三种类型的明显模式称为 \newterm{function application} (函数应用) 或 \newterm{evaluation} (求值)。给定一个函数类型的候选者,我们称之为 $z$(注意,如果我们不在 $\Set$ 范畴中,这只是像任何其他对象一样的对象),以及参数类型 $a$(一个对象),应用将这对映射到结果类型 $b$(一个对象)。我们有三个对象,其中两个是固定的(代表参数类型和结果类型的对象)。

我们还有应用,这是一个映射。我们如何将这个映射纳入我们的模式?如果我们被允许查看对象内部,我们可以将一个函数 $f$($z$ 的一个元素)与一个参数 $x$($a$ 的一个元素)配对,并将其映射到 $f x$($f$ 对 $x$ 的应用,它是 $b$ 的一个元素)。

\begin{figure}[H]
  \centering\includegraphics[width=0.35\textwidth]{images/functionset.jpg}
  \caption{在 Set 中,我们可以从函数集合 $z$ 中选取一个函数 $f$,并从集合(类型)$a$ 中选取一个参数 $x$。我们在集合(类型)$b$ 中得到一个元素 $f x$。}
\end{figure}

\noindent
但是,与其处理单个对 $(f, x)$,我们不如讨论整个函数类型 $z$ 和参数类型 $a$ 的 \emph{product} (积)。积 $z\times{}a$ 是一个对象,我们可以选择一个从该对象到 $b$ 的箭头 $g$ 作为我们的应用态射。在 $\Set$ 中,$g$ 将是把每个对 $(f, x)$ 映射到 $f x$ 的函数。

所以这就是模式:两个对象 $z$ 和 $a$ 的积通过一个态射 $g$ 连接到另一个对象 $b$。

\begin{figure}[H]
  \centering
  \includegraphics[width=0.4\textwidth]{images/functionpattern.jpg}
  \caption{作为泛构造起点的对象和态射的模式}
\end{figure}

\noindent
这个模式是否足够具体,可以用泛构造来唯一确定函数类型?并非在每个范畴中都如此。但在我们感兴趣的范畴中是这样的。另一个问题是:是否有可能在不首先定义积的情况下定义函数对象?有些范畴中没有积,或者并非所有对象对都有积。答案是否定的:如果没有积类型,就没有函数类型。我们稍后讨论 exponentials (指数对象) 时会回到这一点。

让我们回顾一下泛构造。我们从一个对象和态射的模式开始。这是我们不精确的查询,通常会产生大量的匹配结果。特别是在 $\Set$ 中,几乎所有东西都与其他东西相连。我们可以取任何对象 $z$,将其与 $a$ 形成积,并且将会有一个从它到 $b$ 的函数(除非 $b$ 是空集)。

这时我们就要运用我们的秘密武器:ranking (排序)。这通常通过要求候选对象之间存在唯一的映射来完成——这个映射以某种方式 factorizes (因子分解) 我们的构造。在我们的例子中,我们将规定 $z$ 连同从 $z \times a$ 到 $b$ 的态射 $g$ 比某个带有其自身应用 $g'$ 的其他 $z'$ \emph{更好},当且仅当存在一个唯一的映射 $h$ 从 $z'$ 到 $z$,使得 $g'$ 的应用通过 $g$ 的应用分解。(提示:看着图读这句话。)

\begin{figure}[H]
  \centering
  \includegraphics[width=0.4\textwidth]{images/functionranking.jpg}
  \caption{在函数对象的候选者之间建立排序}
\end{figure}

\noindent
现在是棘手的部分,也是我直到现在才推迟这个特定泛构造的主要原因。给定态射 $h \Colon z'\to z$,我们想要闭合这个包含 $z'$ 与 $a$ 的积以及 $z$ 与 $a$ 的积的图。我们真正需要的是,给定从 $z'$ 到 $z$ 的映射 $h$,我们需要一个从 $z' \times a$ 到 $z \times a$ 的映射。现在,在讨论了 \hyperref[functoriality]{product's functoriality} (积的函子性) 之后,我们知道如何做到这一点。因为积本身是一个函子(更准确地说是一个自-双-函子),所以可以 lift pairs of morphisms (提升态射对)。换句话说,我们不仅可以定义对象的积,还可以定义态射的积。

由于我们没有触及积 $z' \times a$ 的第二个分量,我们将提升态射对 $(h, \id)$,其中 $\id$ 是 $a$ 上的 identity (单位态射)。

所以,这就是我们如何将一个应用 $g'$ 通过另一个应用 $g$ 分解出来:
\[g' = g \circ (h \times \id)\]
这里的关键是积在态射上的作用。

泛构造的第三部分是选择那个普遍最好的对象。让我们称这个对象为 $a \Rightarrow b$(把它看作一个对象的符号名称,不要与 Haskell 的类型类约束混淆——我稍后会讨论不同的命名方式)。这个对象带有它自己的应用——一个从 $(a \Rightarrow b) \times a$ 到 $b$ 的态射——我们称之为 $\mathit{eval}$。对象 $a \Rightarrow b$ 是最好的,如果任何其他函数对象的候选者都可以唯一地映射到它,使得其应用态射 $g$ 通过 $\mathit{eval}$ 分解。根据我们的排序,这个对象比任何其他对象都好。

\begin{figure}[H]
  \centering
  \includegraphics[width=0.4\textwidth]{images/universalfunctionobject.jpg}
  \caption{泛函数对象的定义。这与上图相同,但现在对象 $a \Rightarrow b$ 是 \emph{泛}的。}
\end{figure}

\noindent
形式化地:

\begin{longtable}[]{@{}l@{}}
  \toprule
  \begin{minipage}[t]{0.97\columnwidth}\raggedright\strut
    从 $a$ 到 $b$ 的 \emph{function object} (函数对象) 是一个对象 $a \Rightarrow b$ 连同一个态射
    \[\mathit{eval} \Colon ((a \Rightarrow b) \times a) \to b\]
    使得对于任何其他带有态射 $g \Colon z \times a \to b$ 的对象 $z$,存在一个唯一的态射
    \[h \Colon z \to (a \Rightarrow b)\]
    将 $g$ 通过 $\mathit{eval}$ 分解:
    \[g = \mathit{eval} \circ (h \times \id)\]
  \end{minipage}\tabularnewline
  \bottomrule
\end{longtable}

\noindent
当然,不能保证在给定范畴中对于任何一对对象 $a$ 和 $b$ 都存在这样一个对象 $a \Rightarrow b$。但它在 $\Set$ 中总是存在。此外,在 $\Set$ 中,这个对象与态射集合 $\Set(a, b)$ 同构。

这就是为什么在 Haskell 中,我们将函数类型 \code{a -> b} 解释为范畴的函数对象 $a \Rightarrow b$。

\section{柯里化}

让我们再看一下函数对象的所有候选者。然而,这次让我们将态射 $g$ 看作是两个变量 $z$ 和 $a$ 的函数。
\[g \Colon z \times a \to b\]
作为来自积的态射,这已经尽可能接近于成为两个变量的函数了。特别是在 $\Set$ 中,$g$ 是一个从值对(一个来自集合 $z$,一个来自集合 $a$)出发的函数。

另一方面,泛性质告诉我们,对于每个这样的 $g$,存在一个唯一的态射 $h$,它将 $z$ 映射到一个函数对象 $a \Rightarrow b$。
\[h \Colon z \to (a \Rightarrow b)\]
在 $\Set$ 中,这仅仅意味着 $h$ 是一个接受一个 $z$ 类型的变量并返回一个从 $a$ 到 $b$ 的函数的函数。这使得 $h$ 成为一个 higher order function (高阶函数)。因此,泛构造建立了一一对应关系,介于两个变量的函数和接受一个变量并返回函数的函数之间。这种对应关系称为 \newterm{currying} (柯里化),而 $h$ 被称为 $g$ 的柯里化版本。

这种对应关系是一一对应的,因为给定任何 $g$,都有唯一的 $h$,并且给定任何 $h$,你总是可以使用以下公式重新创建双参数函数 $g$:
\[g = \mathit{eval} \circ (h \times \id)\]
函数 $g$ 可以被称为 $h$ 的 \emph{uncurried} (非柯里化) 版本。

柯里化本质上内置于 Haskell 的语法中。一个返回函数的函数:

\src{snippet01}
通常被认为是两个变量的函数。这就是我们解读未加括号的签名的方式:

\src{snippet02}
这种解释在我们定义多参数函数的方式中显而易见。例如:

\src{snippet03}
同一个函数可以写成一个接受单参数并返回函数的函数——一个 lambda 表达式:

\src{snippet04}
这两个定义是等价的,并且两者都可以部分应用于仅一个参数,产生一个单参数函数,如:

\src{snippet05}
严格来说,一个双变量函数是接受一个序对(积类型)的函数:

\src{snippet06}
在这两种表示之间转换是微不足道的,完成转换的两个(高阶)函数毫不意外地被称为 \code{curry} 和 \code{uncurry}:

\src{snippet07}
和

\src{snippet08}
注意,\code{curry} 是函数对象泛构造的 \emph{factorizer} (分解算子)。如果写成这种形式,这一点尤其明显:

\src{snippet09}
(提醒:分解算子从候选者中产生分解函数。)

在非函数式语言中,如 C++,柯里化是可能的但并非易事。你可以将 C++ 中的多参数函数看作对应于接受 tuples (元组) 的 Haskell 函数(尽管更令人困惑的是,在 C++ 中你可以定义接受显式 \code{std::tuple} 的函数,以及 variadic functions (可变参数函数) 和接受 initializer lists (初始化列表) 的函数)。

你可以使用模板 \code{std::bind} 来部分应用 C++ 函数。例如,给定一个接受两个字符串的函数:

\begin{snip}{cpp}
std::string catstr(std::string s1, std::string s2) {
    return s1 + s2;
}
\end{snip}
你可以定义一个接受单个字符串的函数:

\begin{snip}{cpp}
using namespace std::placeholders;

auto greet = std::bind(catstr, "Hello ", _1);
std::cout << greet("Haskell Curry");
\end{snip}
Scala 比 C++ 或 Java 更具函数式特性,处于两者之间。如果你预期你正在定义的函数会被部分应用,你可以用多个参数列表来定义它:

\begin{snip}{cpp}
def catstr(s1: String)(s2: String) = s1 + s2
\end{snip}
当然,这需要库编写者具有一定程度的远见或预知能力。

\section{指数对象}

在数学文献中,函数对象,或两个对象 $a$ 和 $b$ 之间的 internal hom-object (内部态射对象),通常被称为 \newterm{exponential} (指数对象),并表示为 $b^{a}$。注意参数类型在指数位置。这种表示法起初可能看起来很奇怪,但如果你考虑到函数和积之间的关系,它就完全说得通了。我们已经看到,在内部态射对象的泛构造中必须使用积,但联系远不止于此。

当你考虑 finite types (有限类型)——具有有限数量值的类型,如 \code{Bool}、\code{Char},甚至 \code{Int} 或 \code{Double}——之间的函数时,这一点最为明显。这样的函数,至少在原则上,可以完全被 memoized (记忆化) 或转化为可供查找的 data structures (数据结构)。这就是作为态射的函数与作为对象的函数类型之间等价性的本质。

例如,一个来自 \code{Bool} 的(纯)函数完全由一对值指定:一个对应于 \code{False},一个对应于 \code{True}。所有可能的从 \code{Bool} 到(比如)\code{Int} 的函数的集合就是所有 \code{Int} 对的集合。这与积 \code{Int} \times{} \code{Int} 相同,或者稍微创新一点表示法,\code{Int}\textsuperscript{2}。

再举一个例子,让我们看看 C++ 类型 \code{char},它包含 256 个值(Haskell 的 \code{Char} 更大,因为 Haskell 使用 Unicode)。C++ 标准库中有几个函数通常使用查找来实现。像 \code{isupper} 或 \code{isspace} 这样的函数是使用表格实现的,这等价于 256 个布尔值的元组。元组是 product type (积类型),所以我们处理的是 256 个布尔值的积:\code{bool \times{} bool \times{} bool \times{} ... \times{} bool}。我们从算术中知道,迭代积定义了幂。如果你将 \code{bool} 自身“乘以” 256(或 \code{char})次,你会得到 \code{bool} 的 \code{char} 次幂,或 \code{bool}\textsuperscript{\code{char}}。

定义为 256 元组的 \code{bool} 类型中有多少个值?正好是 $2^{256}$。这也是从 \code{char} 到 \code{bool} 的不同函数的数量,每个函数对应一个唯一的 256 元组。你可以类似地计算出从 \code{bool} 到 \code{char} 的函数数量是 $256^{2}$,依此类推。在这些情况下,函数类型的指数表示法完全合理。

我们可能不想完全记忆化一个来自 \code{int} 或 \code{double} 的函数。但是函数和数据类型之间的等价性,即使不总是实用,也是存在的。还有无限类型,例如列表、字符串或树。对这些类型的函数进行及早记忆化将需要无限的存储空间。但 Haskell 是一种惰性语言,因此 lazily evaluated (惰性求值) 的(无限)数据结构和函数之间的界限是模糊的。这种 function vs. data duality (函数与数据的对偶性) 解释了 Haskell 函数类型与范畴指数对象(这更符合我们对 \emph{data} (数据) 的概念)的认同。

\section{笛卡尔闭范畴}

虽然我将继续使用集合范畴作为类型和函数的模型,但值得一提的是,存在一个更大的范畴族可以用于此目的。这些范畴被称为 \newterm{Cartesian closed} categories (笛卡尔闭范畴),而 $\Set$ 只是这类范畴的一个例子。

一个笛卡尔闭范畴必须包含:

\begin{enumerate}
  \tightlist
  \item
        The terminal object (终结对象),
  \item
        任何一对对象的 product (积),以及
  \item
        任何一对对象的 exponential (指数对象)。
\end{enumerate}
如果你将指数对象视为迭代积(可能无限多次),那么你可以认为笛卡尔闭范畴是支持任意元数积的范畴。特别是,终结对象可以被认为是零个对象的积——或者说是一个对象的零次幂。

从计算机科学的角度来看,笛卡尔闭范畴有趣的地方在于它们为简单类型 lambda 演算提供了模型,而简单类型 lambda 演算构成了所有类型化编程语言的基础。

终结对象和积有它们的对偶:the initial object (初始对象) 和 the coproduct (余积)。一个同时也支持这两者,并且积可以在余积上分配
\begin{gather*}
  a \times (b + c) = a \times b + a \times c \\
  (b + c) \times a = b \times a + c \times a
\end{gather*}
的笛卡尔闭范畴被称为 \newterm{bicartesian closed} category (双笛卡尔闭范畴)。我们将在下一节看到,双笛卡尔闭范畴,其中 $\Set$ 是一个典型的例子,具有一些有趣的属性。

\section{指数对象与代数数据类型}

将函数类型解释为指数对象非常符合 algebraic data types (\acronym{ADT}, 代数数据类型) 的体系。事实证明,所有高中代数中关于数字零和一、和、积、指数的基本恒等式,在任何双笛卡尔闭范畴中,对于初始对象和终结对象、余积、积和指数对象,几乎都保持不变。我们还没有工具来证明它们(例如 adjunctions (伴随) 或 the Yoneda lemma (米田引理)),但我还是会在这里列出它们,作为宝贵直觉的来源。

\subsection{零次幂}

\[a^{0} = 1\]
在范畴解释中,我们将 0 替换为初始对象,1 替换为终结对象,等号替换为 isomorphism (同构)。指数对象是内部态射对象。这个特定的指数对象表示从初始对象到任意对象 $a$ 的态射集合。根据初始对象的定义,恰好存在一个这样的态射,因此态射集合 $\cat{C}(0, a)$ 是一个 singleton set (单例集)。单例集是 $\Set$ 中的终结对象,所以这个恒等式在 $\Set$ 中是平凡成立的。我们想说的是,它在任何双笛卡尔闭范畴中都成立。

在 Haskell 中,我们将 0 替换为 \code{Void};1 替换为单位类型 \code{()};指数对象替换为函数类型。这个断言是说,从 \code{Void} 到任何类型 \code{a} 的函数集合等价于单位类型——这是一个单例。换句话说,只有一个函数 \code{Void -> a}。我们以前见过这个函数:它叫做 \code{absurd}。

这有点棘手,原因有二。一是在 Haskell 中我们并没有真正 uninhabited types (空类型)——每个类型都包含“永不结束的计算结果”,即 bottom (底值)。第二个原因是 \code{absurd} 的所有实现都是等价的,因为无论它们做什么,都没有人能够执行它们。没有值可以传递给 \code{absurd}。(如果你设法给它传递一个永不结束的计算,它将永远不会返回!)

\subsection{一的幂}

\[1^{a} = 1\]
这个恒等式,在 $\Set$ 中解释时,重申了终结对象的定义:从任何对象到终结对象存在唯一的态射。一般来说,从 $a$ 到终结对象的内部态射对象与终结对象本身同构。

在 Haskell 中,从任何类型 \code{a} 到单位类型只有一个函数。我们以前见过这个函数——它叫做 \code{unit}。你也可以把它看作是部分应用于 \code{()} 的函数 \code{const}。

\subsection{一次幂}

\[a^{1} = a\]
这是对以下观察的重述:从终结对象出发的态射可以用来挑选对象 \code{a} 的“元素”。这类态射的集合与对象本身同构。在 $\Set$ 中,以及在 Haskell 中,同构存在于集合 \code{a} 的元素与挑选这些元素的函数 \code{() -> a} 之间。

\subsection{和的指数}

\[a^{b+c} = a^{b} \times a^{c}\]
范畴上,这表示从两个对象的 coproduct (余积) 出发的指数对象同构于两个指数对象的 product (积)。在 Haskell 中,这个代数恒等式有一个非常实际的解释。它告诉我们,一个来自两个类型的和的函数等价于一对分别来自单个类型的函数。这正是我们在定义和类型上的函数时使用的 case analysis (情况分析)。我们通常不写一个带有 \code{case} 语句的函数定义,而是将其拆分为两个(或更多)分别处理每个类型构造器的函数。例如,考虑一个来自 sum type (和类型) \code{(Either Int Double)} 的函数:

\src{snippet10}
它可以被定义为一对分别来自 \code{Int} 和 \code{Double} 的函数:

\src{snippet11}
这里,\code{n} 是一个 \code{Int},\code{x} 是一个 \code{Double}。

\subsection{指数的指数}

\[(a^{b})^{c} = a^{b \times c}\]
这只是纯粹用指数对象来表达 currying (柯里化) 的一种方式。一个返回函数的函数等价于一个来自积的函数(一个双参数函数)。

\subsection{积上的指数}

\[(a \times b)^{c} = a^{c} \times b^{c}\]
在 Haskell 中:一个返回序对的函数等价于一对函数,每个函数产生序对的一个元素。

这些简单的高中代数恒等式能够被提升到范畴论并在函数式编程中得到实际应用,真是令人难以置信。

\section{柯里-霍华德同构}

我已经提到了 logic (逻辑) 和代数数据类型之间的对应关系。\code{Void} 类型和单位类型 \code{()} 对应于假和真。积类型和和类型对应于逻辑 conjunction (合取) $\wedge$ (AND) 和 disjunction (析取) $\vee$ (OR)。在这个体系中,我们刚刚定义的函数类型对应于逻辑 implication (蕴含) $\Rightarrow$。换句话说,类型 \code{a -> b} 可以读作“如果 a 则 b”。

根据 the Curry-Howard isomorphism (柯里-霍华德同构),每个类型都可以被解释为一个 proposition (命题)——一个可能为真或为假的陈述或判断。如果该类型是 inhabited (有居民的),则该命题被认为是真,如果不是,则为假。特别是,如果对应的函数类型是有居民的,则逻辑蕴含为真,这意味着存在该类型的函数。因此,一个函数的实现就是一个 theorem (定理) 的证明。编写程序等价于证明定理。让我们看几个例子。

让我们以我们在函数对象的定义中引入的函数 \code{eval} 为例。它的签名是:

\src{snippet12}
它接受一个由函数及其参数组成的序对,并产生适当类型的结果。它是态射的 Haskell 实现:

\[\mathit{eval} \Colon (a \Rightarrow b) \times a \to b\]
它定义了函数类型 $a \Rightarrow b$(或指数对象 $b^{a}$)。让我们使用柯里-霍华德同构将这个签名翻译成逻辑谓词:

\[((a \Rightarrow b) \wedge a) \Rightarrow b\]
你可以这样解读这个陈述:如果 $b$ 从 $a$ 推出为真,并且 $a$ 为真,那么 $b$ 必定为真。这在直觉上完全说得通,并且自古以来就以 \newterm{modus ponens} (肯定前件) 而闻名。我们可以通过实现以下函数来证明这个定理:

\src{snippet13}
如果你给我一个由函数 \code{f}(接受 \code{a} 返回 \code{b})和一个 \code{a} 类型的具体值 \code{x} 组成的序对,我可以通过简单地将函数 \code{f} 应用于 \code{x} 来产生一个 \code{b} 类型的具体值。通过实现这个函数,我刚刚证明了类型 \code{((a -> b), a) -> b} 是有居民的。因此,在我们的逻辑中,肯定前件是真。

那么一个明显错误的谓词呢?例如:如果 $a$ 或 $b$ 为真,那么 $a$ 必须为真。

\[a \vee b \Rightarrow a\]
这显然是错误的,因为你可以选择一个假的 $a$ 和一个真的 $b$,这就是一个反例。

使用柯里-霍华德同构将这个谓词映射到函数签名,我们得到:

\src{snippet14}
无论你怎么尝试,你都无法实现这个函数——如果你被调用时传入的是 \code{Right} 值,你就无法产生一个 \code{a} 类型的值。(记住,我们讨论的是 \emph{pure} (纯) 函数。)

最后,我们来看 \code{absurd} 函数的含义:

\src{snippet15}
考虑到 \code{Void} 翻译为假,我们得到:

\[\mathit{false} \Rightarrow a\]
Anything follows from falsehood (从假推导出任意命题,\emph{ex falso quodlibet})。以下是这个陈述(函数)在 Haskell 中的一个可能证明(实现):

\begin{snip}{haskell}
absurd (Void a) = absurd a
\end{snip}
其中 \code{Void} 定义为:

\begin{snip}{haskell}
newtype Void = Void Void
\end{snip}
像往常一样,\code{Void} 类型很棘手。这个定义使得无法构造一个值,因为为了构造一个值,你需要提供一个值。因此,函数 \code{absurd} 永远不会被调用。

这些都是有趣的例子,但柯里-霍华德同构有实际应用吗?可能在日常编程中没有。但是有一些编程语言,如 Agda 或 Coq,利用柯里-霍华德同构来证明定理。

计算机不仅在帮助数学家完成工作——它们正在彻底改变数学的基础本身。该领域最新的热门研究课题称为 Homotopy Type Theory (同伦类型论),是 type theory (类型论) 的一个分支。它充满了布尔值、整数、积和余积、函数类型等等。而且,仿佛是为了消除任何疑虑,该理论正在用 Coq 和 Agda 来构建。计算机正在以不止一种方式彻底改变世界。

\section{参考文献}

\begin{enumerate}
  \tightlist
  \item
        Ralf Hinze, Daniel W. H. James,
        \urlref{http://www.cs.ox.ac.uk/ralf.hinze/publications/WGP10.pdf}{Reason
          Isomorphically!}。这篇论文包含了本章提到的所有那些范畴论中的高中代数恒等式的证明。
\end{enumerate}