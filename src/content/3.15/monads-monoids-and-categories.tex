% !TEX root = ../../ctfp-print.tex

\lettrine[lhang=0.17]{没}{有哪个地方}是结束一本关于范畴论 (category theory) 的书的好时机。总有更多的东西要学。范畴论是一个庞大的学科。同时,很明显,相同的主题、概念和模式不断地重复出现。有种说法是所有概念都是 Kan 扩展 (Kan extensions),确实,你可以使用 Kan 扩展来推导极限 (limits)、余极限 (colimits)、伴随 (adjunctions)、单子 (monads)、Yoneda 引理等等。范畴本身的概念出现在所有抽象层次上,幺半群 (monoid) 和单子的概念也是如此。哪一个是最基础的呢?事实证明它们都是相互关联的,在一个永无止境的抽象循环中,一个引出另一个。我认为展示这些相互联系可能是结束本书的一个好方法。

\section{双范畴 (Bicategories)}

范畴论最困难的方面之一是视角的不断切换。以集合范畴为例。我们习惯于根据元素 (elements) 来定义集合 (sets)。空集没有元素。单例集有一个元素。两个集合的笛卡尔积 (Cartesian product) 是对的集合,等等。但是在讨论范畴 $\Set$ 时,我要求你忘记集合的内容,而专注于它们之间的态射 (morphisms)(箭头 (arrows))。你被允许时不时地揭开面纱,看看 $\Set$ 中某个特定的泛构造 (universal construction) 在元素方面描述了什么。终对象 (terminal object) 结果是一个只有一个元素的集合,等等。但这些只是健全性检查。

函子 (functor) 被定义为范畴的映射。将映射视为范畴中的态射是很自然的。函子结果是范畴之范畴(小范畴 (small categories),如果我们想避免关于规模的问题)中的态射。通过将函子视为箭头,我们放弃了关于其对范畴内部(其对象和态射)作用的信息,就像当我们将函数视为 $\Set$ 中的箭头时,我们放弃了关于其对集合元素作用的信息一样。但是任意两个范畴之间的函子也构成一个范畴。这次你被要求将一个范畴中的箭头视为另一个范畴中的对象。在函子范畴 (functor category) 中,函子是对象,自然变换 (natural transformations) 是态射。我们发现同一个事物在一个范畴中可以是箭头,在另一个范畴中可以是对象。将对象视为名词、箭头视为动词的朴素观点不成立。

与其在两种视图之间切换,我们可以尝试将它们合并为一个。这就是我们得到 2-范畴 ($\cat{2}$-category) 概念的方式,其中对象被称为 0-胞腔 (0-cells),态射被称为 1-胞腔 (1-cells),态射之间的态射被称为 2-胞腔 (2-cells)。

\begin{figure}[H]
  \centering
  \includegraphics[width=0.35\textwidth]{images/twocat.png}
  \caption{0-胞腔 $a, b$;1-胞腔 $f, g$;以及一个 2-胞腔 $\alpha$。}
\end{figure}

\noindent
范畴之范畴 $\Cat$ 是一个直接的例子。我们有范畴作为 0-胞腔,函子作为 1-胞腔,自然变换作为 2-胞腔。2-范畴的定律告诉我们,任意两个 0-胞腔之间的 1-胞腔构成一个范畴(换句话说,$\cat{C}(a, b)$ 是一个 hom-范畴 (hom-category) 而不是 hom-集 (hom-set))。这与我们之前断言的任意两个范畴之间的函子构成一个函子范畴非常吻合。

特别是,从任何 0-胞腔回到自身的 1-胞腔也构成一个范畴,即 hom-范畴 $\cat{C}(a, a)$;但这个范畴甚至有更多的结构。$\cat{C}(a, a)$ 的成员可以被视为 $\cat{C}$ 中的箭头,或者 $\cat{C}(a, a)$ 中的对象。作为箭头,它们可以相互复合 (compose)。但是当我们将它们视为对象时,复合变成了一个从一对对象到单个对象的映射。事实上,它看起来非常像一个积——准确地说,是一个张量积 (tensor product)。这个张量积有一个单位元 (unit):恒等 1-胞腔。事实证明,在任何 2-范畴中,hom-范畴 $\cat{C}(a, a)$ 自动成为一个幺半范畴 (monoidal category),其张量积定义为 1-胞腔的复合。结合律 (Associativity) 和单位元定律只是从相应的范畴定律中自然得出。

让我们看看这在我们 2-范畴 $\Cat$ 的规范例子中意味着什么。hom-范畴 $\Cat(a, a)$ 是 $a$ 上的自函子 (endofunctors) 范畴。自函子复合在其中扮演张量积的角色。恒等函子 (identity functor) 是关于这个积的单位元。我们之前已经看到自函子构成一个幺半范畴(我们在单子的定义中使用了这个事实),但现在我们看到这是一个更普遍的现象:任何 2-范畴中的自同 1-胞腔 (endo-1-cells) 都构成一个幺半范畴。我们稍后在推广单子时会回到这一点。

你可能记得,在一个一般的幺半范畴中,我们并不坚持幺半群定律 (monoid laws) 必须精确满足。单位元定律和结合律满足直到同构 (isomorphism) 通常就足够了。在一个 2-范畴中,$\cat{C}(a, a)$ 中的幺半定律源于 1-胞腔的复合定律。这些定律是严格的,所以我们总是得到一个严格幺半范畴 (strict monoidal category)。然而,也可以放宽这些定律。例如,我们可以说恒等 1-胞腔 $\idarrow[a]$ 与另一个 1-胞腔 $f \Colon a \to b$ 的复合是同构于 $f$,而不是等于 $f$。1-胞腔的同构是使用 2-胞腔定义的。换句话说,存在一个 2-胞腔:
\[\rho \Colon f \circ \idarrow[a] \to f\]
它有一个逆。

\begin{figure}[H]
  \centering
  \includegraphics[width=0.35\textwidth]{images/bicat.png}
  \caption{双范畴中的恒等定律成立直到同构(一个可逆的 2-胞腔 $\rho$)。}
\end{figure}

\noindent
我们可以对左恒等和结合律做同样的事情。这种放宽的 2-范畴称为双范畴 (bicategory)(还有一些额外的相干定律,我在此省略)。

不出所料,双范畴中的自同 1-胞腔构成一个具有非严格定律的一般幺半范畴。

双范畴的一个有趣例子是 span 范畴。两个对象 $a$ 和 $b$ 之间的 span 是一个对象 $x$ 和一对态射:
\begin{gather*}
  f \Colon x \to a \\
  g \Colon x \to b
\end{gather*}

\begin{figure}[H]
  \centering
  \includegraphics[width=0.35\textwidth]{images/span.png}
\end{figure}

\noindent
你可能记得我们在定义范畴积时使用了 span。在这里,我们想将 span 视为双范畴中的 1-胞腔。第一步是定义 span 的复合。假设我们有一个相邻的 span:
\begin{gather*}
  f' \Colon y \to b \\
  g' \Colon y \to c
\end{gather*}

\begin{figure}[H]
  \centering
  \includegraphics[width=0.5\textwidth]{images/compspan.png}
\end{figure}

\noindent
复合将是第三个 span,具有某个顶点 $z$。对其最自然的选择是 $g$ 沿着 $f'$ 的拉回 (pullback)。记住拉回是对象 $z$ 连同两个态射:
\begin{align*}
  h  & \Colon z \to x \\
  h' & \Colon z \to y
\end{align*}
使得:
\[g \circ h = f' \circ h'\]
并且在所有这样的对象中是泛的。

\begin{figure}[H]
  \centering
  \includegraphics[width=0.5\textwidth]{images/pullspan.png}
\end{figure}

\noindent
现在,让我们专注于集合范畴上的 span。在这种情况下,拉回只是笛卡尔积 $x \times y$ 中满足以下条件的对 $(p, q)$ 的集合:
\[g\ p = f'\ q\]
共享相同端点的两个 span 之间的态射被定义为其顶点之间的态射 $h$,使得适当的三角形交换。

\begin{figure}[H]
  \centering
  \includegraphics[width=0.4\textwidth]{images/morphspan.png}
  \caption{$\cat{Span}$ 中的一个 2-胞腔。}
\end{figure}

\noindent
总结一下,在双范畴 $\cat{Span}$ 中:0-胞腔是集合,1-胞腔是 span,2-胞腔是 span 态射。恒等 1-胞腔是一个退化的 span,其中所有三个对象都相同,两个态射都是恒等态射。

我们之前见过另一个双范畴的例子: \hyperref[ends-and-coends]{profunctor (profunctors)} 的双范畴 $\cat{Prof}$,其中 0-胞腔是范畴,1-胞腔是 profunctor,2-胞腔是自然变换。profunctor 的复合由 coend (余终) 给出。

\section{Monads}

到目前为止,你应该对将 monad 定义为自函子范畴中的幺半群相当熟悉了。让我们带着新的理解重新审视这个定义,即自函子范畴只是双范畴 $\Cat$ 中自同 1-胞腔的一个小小的 hom-范畴。我们知道它是一个幺半范畴:张量积来自自函子的复合。幺半群被定义为幺半范畴中的一个对象——这里它将是一个自函子 $T$——连同两个态射。自函子之间的态射是自然变换。一个态射将幺半单位元——恒等自函子——映射到 $T$:
\[\eta \Colon I \to T\]
第二个态射将 $T \otimes T$ 的张量积映射到 $T$。张量积由自函子复合给出,所以我们得到:
\[\mu \Colon T \circ T \to T\]

\begin{figure}[H]
  \centering
  \includegraphics[width=0.3\textwidth]{images/monad.png}
\end{figure}

\noindent
我们认出这些是定义 monad 的两个运算(在 Haskell 中它们被称为 \code{return} 和 \code{join}),并且我们知道幺半群定律变成了 monad 定律。

现在让我们从这个定义中移除所有关于自函子的提及。我们从一个双范畴 $\cat{C}$ 开始,并在其中选择一个 0-胞腔 $a$。正如我们之前看到的,hom-范畴 $\cat{C}(a, a)$ 是一个幺半范畴。因此,我们可以通过选择一个 1-胞腔 $T$ 和两个 2-胞腔来定义 $\cat{C}(a, a)$ 中的一个幺半群:
\begin{align*}
  \eta & \Colon I \to T         \\
  \mu  & \Colon T \circ T \to T
\end{align*}
满足幺半群定律。我们称 \emph{这个} 为 monad。

\begin{figure}[H]
  \centering
  \includegraphics[width=0.3\textwidth]{images/bimonad.png}
\end{figure}

\noindent
这是一个更通用的 monad 定义,仅使用 0-胞腔、1-胞腔和 2-胞腔。当应用于双范畴 $\Cat$ 时,它简化为通常的 monad。但让我们看看在其他双范畴中会发生什么。

让我们在 $\cat{Span}$ 中构造一个 monad。我们选择一个 0-胞腔,它是一个集合,出于稍后将明了的原因,我将称之为 $\mathit{Ob}$。接下来,我们选择一个自同 1-胞腔:一个从 $\mathit{Ob}$ 回到 $\mathit{Ob}$ 的 span。它在顶点有一个集合,我将称之为 $\mathit{Ar}$,配备两个函数:
\begin{align*}
  \mathit{dom} & \Colon \mathit{Ar} \to \mathit{Ob} \\
  \mathit{cod} & \Colon \mathit{Ar} \to \mathit{Ob}
\end{align*}

\begin{figure}[H]
  \centering
  \includegraphics[width=0.3\textwidth]{images/spanmonad.png}
\end{figure}

\noindent
让我们称集合 $\mathit{Ar}$ 的元素为“箭头 (arrows)”。如果我还告诉你称 $\mathit{Ob}$ 的元素为“对象 (objects)”,你可能会猜到这会导致什么。两个函数 $\mathit{dom}$ 和 $\mathit{cod}$ 为“箭头”分配定义域 (domain) 和上域 (codomain)。

为了使我们的 span 成为一个 monad,我们需要两个 2-胞腔 $\eta$ 和 $\mu$。在这种情况下,幺半单位元是从 $\mathit{Ob}$ 到 $\mathit{Ob}$ 的平凡 span,其顶点在 $\mathit{Ob}$ 处,两个函数都是恒等函数。2-胞腔 $\eta$ 是顶点 $\mathit{Ob}$ 和 $\mathit{Ar}$ 之间的函数。换句话说,$\eta$ 为每个“对象”分配一个“箭头”。$\cat{Span}$ 中的 2-胞腔必须满足交换条件——在这种情况下:
\begin{align*}
  \mathit{dom} & \circ \eta = \id \\
  \mathit{cod} & \circ \eta = \id
\end{align*}

\begin{figure}[H]
  \centering
  \includegraphics[width=0.4\textwidth]{images/spanunit.png}
\end{figure}

\noindent
在分量上,这变成:
\[\mathit{dom}\ (\eta\ \mathit{ob}) = \mathit{ob} = \mathit{cod}\ (\eta\ \mathit{ob})\]
其中 $\mathit{ob}$ 是 $\mathit{Ob}$ 中的一个“对象”。换句话说,$\eta$ 为每个“对象”分配一个“箭头”,其定义域和上域都是该“对象”。我们将称这个特殊的“箭头”为“恒等箭头 (identity arrow)”。

第二个 2-胞腔 $\mu$ 作用于 span $\mathit{Ar}$ 与自身的复合。复合被定义为拉回,因此其元素是来自 $\mathit{Ar}$ 的元素对——即“箭头”对 $(a_1, a_2)$。拉回条件是:
\[\mathit{cod}\ a_1 = \mathit{dom}\ a_2\]
我们说 $a_1$ 和 $a_2$ 是“可复合的 (composable)”,因为一个的上域是另一个的定义域。

\begin{figure}[H]
  \centering
  \includegraphics[width=0.5\textwidth]{images/spanmul.png}
\end{figure}

\noindent
2-胞腔 $\mu$ 是一个函数,它将一对可复合的箭头 $(a_1, a_2)$ 映射到来自 $\mathit{Ar}$ 的单个箭头 $a_3$。换句话说,$\mu$ 定义了箭头的复合。

很容易检查 monad 定律对应于箭头的恒等和结合律。我们刚刚定义了一个范畴(请注意,是一个小范畴,其中对象和箭头构成集合)。

所以,总而言之,范畴只是 span 双范畴中的一个 monad。

这个结果令人惊奇之处在于,它将范畴与其他代数结构如 monad 和幺半群置于同等地位。作为范畴并没有什么特别之处。它只是两个集合和四个函数。事实上,我们甚至不需要一个单独的对象集,因为对象可以与恒等箭头等同起来(它们是一一对应的)。所以它实际上只是一个集合和几个函数。考虑到范畴论在所有数学中扮演的关键角色,这是一个非常令人谦卑的认识。

\section{挑战}

\begin{enumerate}
  \tightlist
  \item
        推导在双范畴中定义为自同 1-胞腔复合的张量积的单位元和结合律。
  \item
        检查 $\cat{Span}$ 中 monad 的 monad 定律是否对应于结果范畴中的恒等和结合律。
  \item
        证明 $\cat{Prof}$ 中的 monad 是一个对象上恒等的函子 (identity-on-objects functor)。
  \item
        $\cat{Span}$ 中 monad 的 monad 代数 (monad algebra) 是什么?
\end{enumerate}

\section{参考文献}
\begin{enumerate}
  \tightlist
  \item
        \urlref{https://graphicallinearalgebra.net/2017/04/16/a-monoid-is-a-category-a-category-is-a-monad-a-monad-is-a-monoid/}{Paweł Sobociński 的博客}。
\end{enumerate}