% !TEX root = ../../ctfp-print.tex

\lettrine[lhang=0.17]{你}{已经看到了如何将} 类型和 pure functions (纯函数) 建模为一个 category (范畴)。我也提到了在范畴论中有一种方法可以对 side effects (副作用),或者说非纯函数,进行建模。让我们看一个这样的例子:记录或追踪其执行过程的函数。在 imperative language (命令式语言) 中,这很可能会通过改变某个 global mutable state (全局可变状态) 来实现,就像这样:

\begin{snip}{cpp}
string logger;

bool negate(bool b) {
    logger += "Not so! ";
    return !b;
}
\end{snip}
你知道这不是一个纯函数,因为它的 memoized (记忆化) 版本将无法产生日志。这个函数具有 \newterm{side effects} (副作用)。

在现代编程中,我们尽可能地避免使用全局可变状态——哪怕仅仅是因为 concurrency (并发) 带来的复杂性。而且你永远不会把这样的代码放到库里面。

对我们来说幸运的是,可以让这个函数变纯。你只需要显式地传入和传出日志。让我们通过添加一个 string 参数,并将常规输出与包含更新后日志的字符串配对来实现这一点:

\begin{snip}{cpp}
pair<bool, string> negate(bool b, string logger) {
    return make_pair(!b, logger + "Not so! ");
}
\end{snip}
这个函数是纯粹的,它没有副作用,每次使用相同的参数调用它时都会返回相同的 pair (对),并且如有必要可以进行记忆化。然而,考虑到日志的累积性,你必须记忆化所有可能导致特定调用的历史记录。对于以下情况,将会有单独的记忆条目:

\begin{snip}{cpp}
negate(true, "It was the best of times. ");
\end{snip}
和

\begin{snip}{cpp}
negate(true, "It was the worst of times. ");
\end{snip}
等等。

对于一个库函数来说,这也不是一个很好的接口。调用者可以自由地忽略返回类型中的字符串,所以这不算是一个巨大的负担;但是他们被迫传入一个字符串作为输入,这可能很不方便。

有没有一种侵入性更小的方法来做同样的事情?有没有一种方法可以分离关注点?在这个简单的例子中,函数\code{negate} 的主要目的是将一个布尔值转换为另一个布尔值。日志记录是次要的。诚然,记录的消息是特定于该函数的,但是将消息聚合成一个连续日志的任务是一个独立的关注点。我们仍然希望函数产生一个字符串,但是我们希望减轻它产生日志的负担。所以这里有一个折衷方案:

\begin{snip}{cpp}
pair<bool, string> negate(bool b) {
    return make_pair(!b, "Not so! ");
}
\end{snip}
其思想是日志将在函数调用\emph{之间}进行聚合。

为了看看这是如何完成的,让我们切换到一个稍微更现实的例子。我们有一个从 string 到 string 的函数,它将小写字符转换为大写:

\begin{snip}{cpp}
string toUpper(string s) {
    string result;
    int (*toupperp)(int) = &toupper; // toupper is overloaded
    transform(begin(s), end(s), back_inserter(result), toupperp);
    return result;
}
\end{snip}
还有另一个函数将一个字符串分割成一个字符串向量,按空白边界断开:

\begin{snip}{cpp}
vector<string> toWords(string s) {
    return words(s);
}
\end{snip}
实际工作是在辅助函数\code{words} 中完成的:

\begin{snip}{cpp}
vector<string> words(string s) {
    vector<string> result{""};
    for (auto i = begin(s); i != end(s); ++i)
    {
        if (isspace(*i))
            result.push_back("");
        else
            result.back() += *i;
    }
    return result;
}
\end{snip}
我们想要修改函数\code{toUpper}和\code{toWords},以便它们在常规返回值的基础上 piggyback (捎带) 一个消息字符串。

\begin{figure}[H]
  \centering
  \includegraphics[width=0.3\textwidth]{images/piggyback.jpg}
\end{figure}
\noindent
我们将“embellish (修饰)”这些函数的返回值。让我们以一种通用的方式来做这件事,定义一个模板\code{Writer},它封装了一个 pair (对),其第一个组件是任意类型\code{A} 的值,第二个组件是一个字符串:

\begin{snip}{cpp}
template<class A>
using Writer = pair<A, string>;
\end{snip}
以下是修饰后的函数:

\begin{snip}{cpp}
Writer<string> toUpper(string s) {
    string result;
    int (*toupperp)(int) = &toupper;
    transform(begin(s), end(s), back_inserter(result), toupperp);
    return make_pair(result, "toUpper ");
}

Writer<vector<string>> toWords(string s) {
    return make_pair(words(s), "toWords ");
}
\end{snip}
我们想要将这两个函数组合成另一个修饰过的函数,该函数将字符串大写并将其分割成单词,同时产生这些操作的日志。我们可以这样做:

\begin{snip}{cpp}
Writer<vector<string>> process(string s) {
    auto p1 = toUpper(s);
    auto p2 = toWords(p1.first);
    return make_pair(p2.first, p1.second + p2.second);
}
\end{snip}
我们已经实现了我们的目标:日志的聚合不再是单个函数的关注点。它们产生自己的消息,然后这些消息在外部被连接成一个更大的日志。

现在想象一下整个程序都是用这种风格编写的。这是一个充斥着重复、易错代码的噩梦。但我们是程序员。我们知道如何处理重复的代码:我们将其抽象出来!然而,这不是普通的抽象——我们必须抽象 \newterm{function composition} (函数组合) 本身。但组合是范畴论的本质,所以在我们编写更多代码之前,让我们从范畴论的角度来分析这个问题。

\section{Writer 范畴}

修饰一堆函数的返回类型以便捎带一些附加功能的想法被证明是非常有成果的。我们将看到更多这样的例子。起点是我们常规的类型和函数范畴。我们将保留类型作为对象,但将我们的态射重新定义为修饰过的函数。

例如,假设我们想要修饰从 \code{int} 到 \code{bool} 的函数 \code{isEven}。我们将其变成一个由修饰过的函数表示的态射。重要的一点是,这个态射仍然被认为是对象 \code{int} 和 \code{bool} 之间的一个箭头,即使修饰过的函数返回一个 pair (对):

\begin{snip}{cpp}
pair<bool, string> isEven(int n) {
    return make_pair(n % 2 == 0, "isEven ");
}
\end{snip}
根据范畴的定律,我们应该能够将这个态射与另一个从对象 \code{bool} 到任何其他对象的态射组合。特别是,我们应该能够将它与我们之前的 \code{negate} 组合:

\begin{snip}{cpp}
pair<bool, string> negate(bool b) {
    return make_pair(!b, "Not so! ");
}
\end{snip}
显然,我们不能像组合常规函数那样组合这两个态射,因为存在输入/输出不匹配。它们的组合应该更像这样:

\begin{snip}{cpp}
pair<bool, string> isOdd(int n) {
    pair<bool, string> p1 = isEven(n);
    pair<bool, string> p2 = negate(p1.first);
    return make_pair(p2.first, p1.second + p2.second);
}
\end{snip}
所以,在我们正在构建的这个新范畴中,组合两个态射的方法如下:

\begin{enumerate}
  \tightlist
  \item
        执行与第一个态射对应的修饰过的函数
  \item
        提取结果对的第一个组件,并将其传递给与第二个态射对应的修饰过的函数
  \item
        连接第一个结果的第二个组件(字符串)和第二个结果的第二个组件(字符串)
  \item
        返回一个新对,结合最终结果的第一个组件和连接后的字符串。
\end{enumerate}

如果我们想在 C++ 中将这种组合抽象为一个 higher order function (高阶函数),我们必须使用一个由我们范畴中三个对象对应的三个类型参数化的模板。它应该接受两个根据我们的规则可组合的修饰过的函数,并返回第三个修饰过的函数:

\begin{snip}{cpp}
template<class A, class B, class C>
function<Writer<C>(A)> compose(function<Writer<B>(A)> m1,
                               function<Writer<C>(B)> m2)
{
    return [m1, m2](A x) {
        auto p1 = m1(x);
        auto p2 = m2(p1.first);
        return make_pair(p2.first, p1.second + p2.second);
    };
}
\end{snip}
现在我们可以回到之前的例子,并使用这个新模板来实现 \code{toUpper} 和 \code{toWords} 的组合:

\begin{snip}{cpp}
Writer<vector<string>> process(string s) {
    return compose<string, string, vector<string>>(toUpper, toWords)(s);
}
\end{snip}
向\code{compose} 模板传递类型仍然有很多噪音。只要你有一个兼容 C++14 的编译器支持带返回类型推导的广义 lambda 函数,就可以避免这种情况(此代码归功于 Eric Niebler):

\begin{snip}{cpp}
auto const compose = [](auto m1, auto m2) {
    return [m1, m2](auto x) {
        auto p1 = m1(x);
        auto p2 = m2(p1.first);
        return make_pair(p2.first, p1.second + p2.second);
    };
};
\end{snip}
在这个新定义中,\code{process} 的实现简化为:

\begin{snip}{cpp}
Writer<vector<string>> process(string s) {
    return compose(toUpper, toWords)(s);
}
\end{snip}
但我们还没有完成。我们已经定义了新范畴中的组合,但是 identity morphisms (恒等态射) 是什么?这些不是我们常规的恒等函数!它们必须是从类型 A 回到类型 A 的态射,这意味着它们是以下形式的修饰过的函数:

\begin{snip}{cpp}
Writer<A> identity(A);
\end{snip}
它们必须表现得像组合下的单位元。如果你看一下我们的组合定义,你会发现恒等态射应该不加改变地传递其参数,并且只向日志贡献一个空字符串:

\begin{snip}{cpp}
template<class A> Writer<A> identity(A x) {
    return make_pair(x, "");
}
\end{snip}
你可以很容易地说服自己,我们刚刚定义的范畴确实是一个合法的范畴。特别是,我们的组合显然是 associative (结合的)。如果你跟踪每个对的第一个组件发生了什么,那只是常规的函数组合,它是结合的。第二个组件正在被连接,而连接也是结合的。

敏锐的读者可能会注意到,将这种构造推广到任何 monoid (幺半群),而不仅仅是字符串幺半群,是很容易的。我们将在\mbox{\code{compose}} 内部使用 \code{mappend},在\code{identity} 内部使用 \code{mempty}(代替\code{+} 和\code{""})。确实没有理由将自己局限于仅记录字符串。一个好的库编写者应该能够识别出使库工作的最低限度的约束——这里的日志库的唯一要求是日志具有幺半群属性。

\section{Haskell 中的 Writer}

在 Haskell 中做同样的事情会稍微简洁一些,而且我们还能从编译器那里得到更多帮助。让我们从定义\mbox{\code{Writer}} 类型开始:

\src{snippet01}
这里我只是定义了一个 type alias (类型别名),相当于 C++ 中的\code{typedef}(或\code{using})。类型\mbox{\code{Writer}} 由类型变量\code{a} 参数化,并且等价于\mbox{\code{a}} 和\code{String} 的一个对。对的语法非常简洁:只需将两项放在括号中,用逗号分隔。

我们的态射是从任意类型到某个 \code{Writer} 类型的函数:

\src{snippet02}
我们将把组合声明为一个有趣的中缀运算符,有时被称为“fish (鱼形)”运算符:

\src{snippet03}
它是一个接受两个参数的函数,每个参数本身都是一个函数,并返回一个函数。第一个参数的类型是 \code{(a -> Writer b)},第二个是 \code{(b -> Writer c)},结果是 \code{(a -> Writer c)}。

这是这个中缀运算符的定义——两个参数 \code{m1} 和 \code{m2} 出现在鱼形符号的两侧:

\src{snippet04}
结果是一个接受一个参数 \code{x} 的 lambda function (lambda 函数)。Lambda 用反斜杠表示——可以把它想象成被截掉一条腿的希腊字母 $\lambda$。

\code{let} 表达式允许你声明辅助变量。这里,调用 \code{m1} 的结果被 pattern matched (模式匹配) 到一对变量 \code{(y, s1)};而使用来自第一个模式的参数 \code{y} 调用 \code{m2} 的结果被匹配到 \code{(z, s2)}。

在 Haskell 中,对进行模式匹配是很常见的,而不是像我们在 C++ 中那样使用访问器。除此之外,两种实现之间存在相当直接的对应关系。

\code{let} 表达式的整体值在其 \code{in} 子句中指定:这里它是一个对,其第一个组件是 \code{z},第二个组件是两个字符串的连接 \code{s1++s2}。

我还将为我们的范畴定义恒等态射,但由于稍后才会明朗的原因,我将称之为\code{return}。

\src{snippet05}
为了完整起见,让我们给出修饰过的函数 \code{upCase} 和 \code{toWords} 的 Haskell 版本:

\src{snippet06}
函数\code{map} 对应于 C++ 的\code{transform}。它将字符函数\code{toUpper} 应用于字符串 \code{s}。辅助函数\mbox{\code{words}} 定义在标准的 Prelude 库中。

最后,这两个函数的组合是在鱼形操作符的帮助下完成的:

\src{snippet07}

\section{Kleisli 范畴}

你可能已经猜到我不是当场发明这个范畴的。它是所谓的 Kleisli category (Kleisli 范畴) 的一个例子——一个基于 monad (单子) 的范畴。我们还没有准备好讨论单子,但我想让你领略一下它们能做什么。对于我们有限的目的来说,一个 Kleisli 范畴以底层编程语言的类型作为对象。从类型 $A$ 到类型 $B$ 的态射是使用特定的修饰方式从 $A$ 映射到由 $B$ 派生出的类型的函数。每个 Kleisli 范畴都定义了自己组合这种态射的方式,以及相对于该组合的恒等态射。(稍后我们将看到,不精确的术语“修饰”对应于范畴中 endofunctor (自函子) 的概念。)

我在本章中用作范畴基础的特定单子被称为 \newterm{writer monad} (Writer 单子),它用于记录或追踪函数的执行。它也是一个更通用的机制的例子,用于在纯计算中嵌入 effects (效应)。你之前已经看到我们可以用集合范畴来建模编程语言的类型和函数(像往常一样忽略 bottom (底值))。这里我们将这个模型扩展到了一个稍微不同的范畴,一个态射由修饰过的函数表示,并且它们的组合不仅仅是将一个函数的输出传递给另一个函数的输入的范畴。我们还有一个可以玩的自由度:组合本身。事实证明,这正是使得能够为那些在命令式语言中传统上使用副作用实现的程序赋予简单 denotational semantics (指称语义) 的自由度。

\section{挑战}

一个没有为其参数的所有可能值定义的函数称为 partial function (偏函数)。它在数学意义上并非真正的函数,因此不符合标准的范畴论模式。然而,它可以由一个返回修饰过的类型\code{optional} 的函数来表示:

\begin{snip}{cpp}
template<class A> class optional {
    bool _isValid;
    A _value;
public:
    optional()    : _isValid(false) {}
    optional(A v) : _isValid(true), _value(v) {}
    bool isValid() const { return _isValid; }
    A value() const { return _value; }
};
\end{snip}
例如,这里是修饰过的函数 \texttt{safe\_root} 的实现:

\begin{snip}{cpp}
optional<double> safe_root(double x) {
    if (x >= 0) return optional<double>{sqrt(x)};
    else return optional<double>{};
}
\end{snip}
挑战如下:

\begin{enumerate}
  \tightlist
  \item
        为偏函数构建 Kleisli 范畴(定义组合和恒等)。
  \item
        实现修饰过的函数 \texttt{safe\_reciprocal},如果其参数不为零,则返回其有效的倒数。
  \item
        组合函数 \texttt{safe\_root} 和 \texttt{safe\_reciprocal} 来实现 \texttt{safe\_root\_reciprocal},在可能的情况下计算 \texttt{sqrt(1/x)}。
\end{enumerate}