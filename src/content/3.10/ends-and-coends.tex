% !TEX root = ../../ctfp-print.tex

\lettrine[lhang=0.17]{对}{于范畴中的态射 (morphisms)},我们可以赋予多种直观理解,但我们都能认同,如果存在从对象 $a$ 到对象 $b$ 的态射,那么这两个对象在某种程度上是“相关的 (related)”。在某种意义上,态射就是这种关系的证明 (proof)。这在任何偏序集范畴 (poset category) 中都清晰可见,其中态射 \emph{就是} 关系。通常,对于两个对象之间的同一关系,可能存在多个“证明”。这些证明构成一个集合,我们称之为 hom-集 (hom-set)。当我们改变对象时,我们得到一个从对象对到“证明”集合的映射。这个映射是函子性的 (functorial)——在第一个参数上是逆变的 (contravariant),在第二个参数上是协变的 (covariant)。我们可以将其视为在范畴中建立对象间的全局关系。这种关系由 hom-函子 (hom-functor) 描述:
\[\cat{C}(-, =) \Colon \cat{C}^\mathit{op}\times{}\cat{C} \to \Set\]
通常,任何像这样的函子都可以解释为在范畴中建立对象间的关系。关系也可能涉及两个不同的范畴 $\cat{C}$ 和 $\cat{D}$。描述这种关系的函子具有以下签名,并被称为 profunctor:
\[p \Colon \cat{D}^\mathit{op}\times{}\cat{C} \to \Set\]
数学家称其为从 $\cat{C}$ 到 $\cat{D}$ 的 profunctor(注意顺序反转),并使用带斜线的箭头作为符号:
\[\cat{C} \nrightarrow \cat{D}\]
你可以将 profunctor 视为 $\cat{C}$ 的对象和 $\cat{D}$ 的对象之间的 \newterm{证明相关关系 (proof-relevant relation)},其中集合的元素象征着关系的证明。当 $p\ a\ b$ 为空时,$a$ 和 $b$ 之间没有关系。请记住,关系不必是对称的。

另一个有用的直观理解是推广“自函子 (endofunctor) 是容器 (container)”这一想法。那么类型为 $p\ a\ b$ 的 profunctor 值可以被视为以类型 $a$ 的元素为键 (keyed) 的 $b$ 的容器。特别是,hom-profunctor 的一个元素是从 $a$ 到 $b$ 的函数 (function)。

在 Haskell 中,profunctor 被定义为一个双参数类型构造器 \code{p},配备了一个名为 \code{dimap} 的方法,它提升一对函数,其中第一个函数走向“错误”的方向:

\src{snippet01}
Profunctor 的函子性告诉我们,如果我们有一个证明 \code{a} 与 \code{b} 相关,那么只要存在从 \code{c} 到 \code{a} 的态射以及从 \code{b} 到 \code{d} 的另一个态射,我们就能得到 \code{c} 与 \code{d} 相关的证明。或者,我们可以将第一个函数视为将新键转换为旧键,第二个函数视为修改容器的内容。

对于在单一范畴内作用的 profunctor,我们可以从类型为 $p\ a\ a$ 的对角元素 (diagonal elements) 中提取大量信息。只要我们有一对态射 $b \to a$ 和 $a \to c$,我们就可以证明 $b$ 与 $c$ 相关。更好的是,我们可以使用单个态射来达到非对角值。例如,如果我们有一个态射 $f \Colon a \to b$,我们可以提升对 $\langle f, \idarrow[b] \rangle$ 从 $p\ b\ b$ 到 $p\ a\ b$:

\src{snippet02}
或者我们可以提升对 $\langle \idarrow[a], f \rangle$ 从 $p\ a\ a$ 到 $p\ a\ b$:

\src{snippet03}

\section{Dinatural Transformations}

由于 profunctor 是函子 (functors),我们可以用标准方式定义它们之间的自然变换 (natural transformations)。然而,在许多情况下,定义两个 profunctor 的对角元素之间的映射就足够了。这种变换称为 dinatural transformation (双自然变换),前提是它满足反映了我们将对角元素连接到非对角元素的两种方式的交换条件。两个 profunctor $p$ 和 $q$(它们是函子范畴 ${[}\cat{C}^\mathit{op}\times{}\cat{C}, \Set{]}$ 的成员)之间的 dinatural transformation 是一族态射:
\[\alpha_a \Colon p\ a\ a \to q\ a\ a\]
对于任何 $f \Colon a \to b$,以下图表交换:

\begin{figure}[H]
  \centering
  \includegraphics[width=0.35\textwidth]{images/end.jpg}
\end{figure}

\noindent
注意,这严格弱于自然性条件 (naturality condition)。如果 $\alpha$ 是 ${[}\cat{C}^\mathit{op}\times{}\cat{C}, \Set{]}$ 中的自然变换,那么上述图表可以由两个自然性方块 (naturality squares) 和一个函子性条件(profunctor $q$ 保持组合)构造出来:

\begin{figure}[H]
  \centering
  \includegraphics[width=0.4\textwidth]{images/end-1.jpg}
\end{figure}

\noindent
注意,${[}\cat{C}^\mathit{op}\times{}\cat{C}, \Set{]}$ 中自然变换 $\alpha$ 的一个分量由一对对象 $\alpha_{a b}$ 索引。而 dinatural transformation 则由一个对象索引,因为它只映射相应 profunctor 的对角元素。

\section{Ends}

我们现在准备从“代数 (algebra)”进阶到可以被认为是范畴论 (category theory) 的“微积分 (calculus)”的内容。end (和 coend) 的演算借鉴了传统微积分的思想甚至一些表示法。特别地,coend 可以理解为无限和或积分 (integral),而 end 类似于无限积 (infinite product)。甚至还有类似于 Dirac delta 函数的东西。

End 是极限 (limit) 的推广,其中函子被 profunctor 取代。它使用楔 (wedge) 而不是锥 (cone)。楔的底部 (base) 由 profunctor $p$ 的对角元素构成。楔的顶点 (apex) 是一个对象(这里是一个集合,因为我们考虑的是 $\Set$ 值 profunctor),而边 (sides) 是一族从顶点映射到底部集合的函数。你可以将这一族函数视为一个多态函数 (polymorphic function)——一个在其返回类型上多态的函数:
\[\alpha \Colon \forall a\ .\ \mathit{apex} \to p\ a\ a\]
与锥不同,在楔内部我们没有任何连接底部顶点的函数。然而,正如我们前面看到的,给定 $\cat{C}$ 中的任何态射 $f \Colon a \to b$,我们可以将 $p\ a\ a$ 和 $p\ b\ b$ 都连接到公共集合 $p\ a\ b$。因此,我们坚持以下图表交换:

\begin{figure}[H]
  \centering
  \includegraphics[width=0.4\textwidth]{images/end-2.jpg}
\end{figure}

\noindent
这被称为 \newterm{楔条件 (wedge condition)}。它可以写成:
\[p\ \idarrow[a]\ f \circ \alpha_a = p\ f\ \idarrow[b] \circ \alpha_b\]
或者,使用 Haskell 表示法:

\src{snippet04}
我们现在可以继续进行泛构造 (universal construction),并将 $p$ 的 end 定义为泛楔 (universal wedge)——一个集合 $e$ 以及一族函数 $\pi$,使得对于任何其他具有顶点 $a$ 和族 $\alpha$ 的楔,存在唯一的函数 $h \Colon a \to e$ 使得所有三角形都交换:
\[\pi_a \circ h = \alpha_a\]

\begin{figure}[H]
  \centering
  \includegraphics[width=0.4\textwidth]{images/end-21.jpg}
\end{figure}

\noindent
End 的符号是积分号,其“积分变量”位于下标位置:
\[\int_c p\ c\ c\]
$\pi$ 的分量称为 end 的投影映射 (projection maps):
\[\pi_a \Colon \int_c p\ c\ c \to p\ a\ a\]
注意,如果 $\cat{C}$ 是一个离散范畴 (discrete category)(除了恒等态射外没有其他态射),则 end 只是 $p$ 在整个范畴 $\cat{C}$ 上的所有对角项的全局积 (global product)。稍后我将展示,在更一般的情况下,end 和这个积之间通过等化子 (equalizer) 存在关系。

在 Haskell 中,end 公式直接转换为全称量词 (universal quantifier):

\src{snippet05}
严格来说,这只是 $p$ 的所有对角元素的积,但由于 \urlref{https://bartoszmilewski.com/2017/04/11/profunctor-parametricity/}{参数化性 (parametricity)},楔条件会自动满足。对于任何函数 $f \Colon a \to b$,楔条件读作:

\src{snippet06}
或者,带类型注解:

\begin{snipv}
dimap f id\textsubscript{b} . pi\textsubscript{b} = dimap id\textsubscript{a} f . pi\textsubscript{a}
\end{snipv}
其中等式两边都有类型:

\src{snippet07}
而 \code{pi} 是多态投影:

\src{snippet08}
这里,类型推断 (type inference) 自动选择 \code{e} 的正确分量。

正如我们将锥的所有交换条件表达为一个自然变换一样,类似地,我们可以将所有的楔条件归为一个 dinatural transformation。为此,我们需要将常函子 (constant functor) $\Delta_c$ 推广到常 profunctor,它将所有对象对映射到单个对象 $c$,并将所有态射对映射到该对象的恒等态射。楔是从该函子到 profunctor $p$ 的 dinatural transformation。确实,当我们意识到 $\Delta_c$ 将所有态射提升为一个恒等函数时,dinaturality 六边形就缩小为楔形菱形。

End 也可以为 $\Set$ 以外的目标范畴定义,但这里我们只考虑 $\Set$ 值 profunctor 及其 end。

\section{Ends 作为等化子}

End 定义中的交换条件可以用等化子 (equalizer) 来写。首先,让我们定义两个函数(我使用 Haskell 表示法,因为数学表示法在这种情况下似乎不太用户友好)。这些函数对应于楔条件的两个汇合分支:

\src{snippet09}[b]
这两个函数都将 profunctor \code{p} 的对角元素映射到类型为以下的多态函数:

\src{snippet10}
这些函数有不同的类型。然而,如果我们形成一个大的积类型,将 \code{p} 的所有对角元素聚集在一起,就可以统一它们的类型:

\src{snippet11}
函数 \code{lambda} 和 \code{rho} 从这个积类型导出两个映射:

\src{snippet12}
\code{p} 的 end 是这两个函数的等化子。记住,等化子选择两个函数相等的最大子集。在这种情况下,它选择所有对角元素之积的子集,在该子集上楔图交换。

\section{自然变换作为 Ends}

End 最重要的例子是自然变换 (natural transformations) 的集合。两个函子 $F$ 和 $G$ 之间的自然变换是从形式为 $\cat{C}(F a, G a)$ 的 hom-集中选取的一族态射。如果不是因为自然性条件,自然变换的集合就只是所有这些 hom-集的积。事实上,在 Haskell 中,它就是这样定义的:

\src{snippet13}
这在 Haskell 中有效的原因是自然性源于参数化性 (parametricity)。然而,在 Haskell 之外,并非所有穿过这些 hom-集的对角截面 (diagonal sections) 都能产生自然变换。但请注意,映射:
\[\langle a, b \rangle \to \cat{C}(F a, G b)\]
是一个 profunctor,因此研究它的 end 是有意义的。这就是楔条件:

\begin{figure}[H]
  \centering
  \includegraphics[width=0.4\textwidth]{images/end1.jpg}
\end{figure}

\noindent
让我们从集合 $\int_c \cat{C}(F c, G c)$ 中选取一个元素。两个投影会将这个元素映射到某个特定变换的两个分量,我们称之为:
\begin{align*}
  \tau_a & \Colon F a \to G a \\
  \tau_b & \Colon F b \to G b
\end{align*}
在左分支中,我们使用 hom-函子提升一对态射 $\langle \idarrow[a], G f \rangle$。你可能记得这种提升是通过同时进行前复合 (pre-composition) 和后复合 (post-composition) 来实现的。当作用于 $\tau_a$ 时,提升后的对给出:
\[G f \circ \tau_a \circ \idarrow[a]\]
图的另一个分支给出:
\[\idarrow[b] \circ \tau_b \circ F f\]
楔条件要求的它们的相等性,恰好就是 $\tau$ 的自然性条件。

\section{Coends}
不出所料,end 的对偶称为 coend。它由楔的对偶,称为 cowedge (余楔)(发音为 co-wedge,而不是 cow-edge)构造而成。

\begin{figure}[H]
  \centering
  \includegraphics[width=0.25\textwidth]{images/end-31.jpg}
  \caption{一头前卫的牛?(An edgy cow?)}
\end{figure}

\noindent
Coend 的符号是积分号,其“积分变量”位于上标位置:
\[\int^c p\ c\ c\]
就像 end 与积相关一样,coend 与余积 (coproduct) 或和相关(在这方面,它类似于积分 (integral),积分是和的极限)。它没有投影 (projections),而是有内射 (injections),从 profunctor 的对角元素向下注入到 coend。如果不是因为 cowedge 条件,我们可以说 profunctor $p$ 的 coend 要么是 $p\ a\ a$,要么是 $p\ b\ b$,要么是 $p\ c\ c$,等等。或者我们可以说存在这样一个 $a$,使得 coend 就是集合 $p\ a\ a$。我们在 end 定义中使用的全称量词 (universal quantifier) 对于 coend 变成了存在量词 (existential quantifier)。

这就是为什么在伪 Haskell 中,我们会这样定义 coend:

\begin{snip}{text}
exists a. p a a
\end{snip}
在 Haskell 中编码存在量词的标准方法是使用全称量化的数据构造器 (universally quantified data constructors)。因此我们可以定义:

\src{snippet14}
这背后的逻辑是,应该可以使用类型族 $p\ a\ a$ 中的任何一个类型的值来构造 coend,无论我们选择哪个 $a$。

就像 end 可以用等化子定义一样,coend 可以用 \newterm{coequalizer (余等化子)} 来描述。所有的 cowedge 条件可以通过对所有可能的函数 $b \to a$ 取一个巨大的 $p\ a\ b$ 的余积来概括。在 Haskell 中,这可以表示为一个存在类型 (existential type):

\src{snippet15}
有两种方法可以求值这个和类型 (sum type),通过使用 \code{dimap} 提升函数并将其应用于 profunctor $p$:

\src{snippet16}
其中 \code{DiagSum} 是 $p$ 的对角元素之和:

\src{snippet17}
这两个函数的余等化子就是 coend。余等化子是通过识别由将 \code{lambda} 或 \code{rho} 应用于相同参数所获得的值,从 \code{DiagSum p} 得到的。这里的参数是由一个函数 $b \to a$ 和 $p\ a\ b$ 的一个元素组成的对。\code{lambda} 和 \code{rho} 的应用产生类型为 \code{DiagSum p} 的两个可能不同的值。在 coend 中,这两个值被视为相同,使得 cowedge 条件自动满足。

识别集合中相关元素的过程在形式上称为取商 (taking a quotient)。为了定义商,我们需要一个 \newterm{equivalence relation (等价关系)} $\sim$,这是一个自反 (reflexive)、对称 (symmetric) 和传递 (transitive) 的关系:
\begin{align*}
   & a \sim a                                                         \\
   & \text{if}\ a \sim b\ \text{then}\ b \sim a                       \\
   & \text{if}\ a \sim b\ \text{and}\ b \sim c\ \text{then}\ a \sim c
\end{align*}
这种关系将集合划分为等价类 (equivalence classes)。每个类由相互关联的元素组成。我们通过从每个类中选取一个代表 (representative) 来形成商集 (quotient set)。一个经典的例子是有理数 (rational numbers) 的定义,即整数对,具有以下等价关系:
\[(a, b) \sim (c, d)\ \text{iff}\ a * d = b * c\]
很容易检查这确实是一个等价关系。一对 $(a, b)$ 被解释为分数 $\frac{a}{b}$,分子和分母有公约数的分数被视为相同。有理数就是这种分数的等价类。

你可能还记得我们之前讨论极限 (limits) 和余极限 (colimits) 时提到,hom-函子是连续的 (continuous),即它保持极限。对偶地,逆变 hom-函子将余极限 (colimits) 转换为极限。这些性质可以推广到分别是极限和余极限推广的 end 和 coend。特别是,我们得到一个非常有用的用于将 coend 转换为 end 的恒等式:
\[\Set(\int^x p\ x\ x, c) \cong \int_x \Set(p\ x\ x, c)\]
让我们用伪 Haskell 来看一下:

\begin{snipv}
(exists x. p x x) -> c \ensuremath{\cong} forall x. p x x -> c
\end{snipv}
它告诉我们,接受存在类型的函数等价于多态函数。这完全合理,因为这样的函数必须准备好处理可能编码在存在类型中的任何一种类型。这与告诉我们接受和类型 (sum type) 的函数必须实现为 case 语句(带有一组处理程序,和类型中的每种类型对应一个)的原理相同。在这里,和类型被 coend 替换,处理程序族变成了 end,或者说多态函数。

\section{Ninja Yoneda 引理}

出现在 Yoneda 引理 (Yoneda lemma) 中的自然变换集合可以用 end 来编码,得到以下表述:
\[\int_z \Set(\cat{C}(a, z), F z) \cong F a\]
还有一个对偶公式:
\[\int^z \cat{C}(z, a)\times{}F z \cong F a\]
这个恒等式强烈地让人联想到 Dirac delta 函数(一个在 $a=z$ 处有无限峰值的函数 $\delta(a - z)$,或者更确切地说是一个分布 (distribution))的公式。在这里,hom-函子扮演了 delta 函数的角色。

这两个恒等式合在一起有时被称为 Ninja Yoneda 引理 (Ninja Yoneda lemma)。

为了证明第二个公式,我们将使用 Yoneda 嵌入 (Yoneda embedding) 的推论,该推论指出两个对象同构当且仅当它们的 hom-函子同构。换句话说,$a \cong b$ 当且仅当存在类型为以下形式的自然变换:
\[[\cat{C}, \Set](\cat{C}(a, -), \cat{C}(b, =))\]
并且它是一个同构 (isomorphism)。

我们从将要证明的恒等式的左侧插入到一个映射到某个任意对象 $c$ 的 hom-函子内部开始:
\[\Set(\int^z \cat{C}(z, a)\times{}F z, c)\]
使用连续性论证 (continuity argument),我们可以将 coend 替换为 end:
\[\int_z \Set(\cat{C}(z, a)\times{}F z, c)\]
现在我们可以利用积和指数之间的伴随关系 (adjunction):
\[\int_z \Set(\cat{C}(z, a), c^{(F z)})\]
我们可以通过使用 Yoneda 引理来“执行积分”得到:
\[c^{(F a)}\]
(注意我们使用了 Yoneda 引理的逆变版本,因为函子 $c^{(F z)}$ 在 $z$ 上是逆变的。)
这个指数对象同构于 hom-集:
\[\Set(F a, c)\]
最后,我们利用 Yoneda 嵌入得到同构:
\[\int^z \cat{C}(z, a)\times{}F z \cong F a\]

\section{Profunctor 组合}

让我们进一步探讨 profunctor 描述关系的这个想法——更准确地说,是证明相关关系,意味着集合 $p\ a\ b$ 代表 $a$ 与 $b$ 相关的证明集合。如果我们有两个关系 $p$ 和 $q$,我们可以尝试组合它们。我们会说,$a$ 通过 $q$ 在 $p$ 之后的组合与 $b$ 相关,如果存在一个中间对象 $c$ 使得 $q\ b\ c$ 和 $p\ c\ a$ 都非空。这个新关系的证明是各个关系的所有证明对。因此,理解了存在量词对应于 coend,两个集合的笛卡尔积对应于“证明对”,我们可以使用以下公式定义 profunctor 的组合:
\[(q \circ p)\ a\ b = \int^c p\ c\ a\times{}q\ b\ c\]
这是来自 \code{Data.Profunctor.Composition} 的等价 Haskell 定义,经过一些重命名后:

\src{snippet18}
这里使用了广义代数数据类型 (generalized algebraic data type),或 \acronym{GADT} 语法,其中自由类型变量(这里是 \code{c})被自动存在量化。(非柯里化的)数据构造器 \code{Procompose} 因此等价于:

\begin{snip}{text}
exists c. (q a c, p c b)
\end{snip}
如此定义的组合的单位元是 hom-函子——这直接源于 Ninja Yoneda 引理。因此,提出这样一个问题是有意义的:是否存在一个范畴,其中 profunctor 充当态射?答案是肯定的,但需要注意 profunctor 组合的结合律和单位元定律仅在自然同构 (natural isomorphism) 意义下成立。这种定律在同构意义下有效的范畴称为双范畴 (bicategory)(它比 $\cat{2}$-范畴 ($\cat{2}$-category) 更一般)。所以我们有一个双范畴 $\cat{Prof}$,其中对象是范畴,态射是 profunctor,而态射之间的态射(又名,2-胞腔 (two-cells))是自然变换。事实上,我们甚至可以更进一步,因为除了 profunctor,我们还有常规函子作为范畴之间的态射。具有两种类型态射的范畴称为双重范畴 (double category)。

Profunctor 在 Haskell lens 库和 arrow 库中扮演着重要角色。