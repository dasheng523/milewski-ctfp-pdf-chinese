% !TEX root = ../../ctfp-print.tex

\lettrine[lhang=0.17]{通}{过研究各种各样的例子},你可以真正领会 categories (范畴)。范畴有各种形态和大小,并常常在意想不到的地方出现。我们将从一些非常简单的东西开始。

\section{没有对象}

最平凡的范畴是包含零个对象,因此也包含零个 morphisms (态射) 的范畴。它本身是一个非常可怜的范畴,但它在其他范畴的语境下可能很重要,例如,在所有范畴构成的范畴中(是的,确实存在这样一个范畴)。如果你认为空集有意义,那么为什么空范畴没有意义呢?

\section{简单图}

你可以仅仅通过用箭头连接对象来构建范畴。你可以想象从任何 \newterm{directed graph} (有向图) 开始,通过简单地添加更多箭头将其变成一个范畴。首先,在每个节点处添加一个 identity arrow (恒等箭头)。然后,对于任意两个箭头,如果一个箭头的终点与另一个箭头的起点重合(换句话说,任意两个 \newterm{composable} (可组合的) 箭头),则添加一个新的箭头作为它们的 composition (组合)。每次你添加一个新的箭头时,你还必须考虑它与任何其他箭头(恒等箭头除外)以及它自身的组合。通常最终会得到无限多的箭头,但这没关系。

看待这个过程的另一种方式是,你正在创建一个范畴,它为图中的每个节点包含一个对象,并以所有可能的可组合图边的 \newterm{chains} (链) 作为态射。(你甚至可以将恒等态射视为长度为零的特殊链。)

这样的范畴被称为由给定图生成的 \newterm{free category} (自由范畴)。这是一个 \newterm{free construction} (自由构造) 的例子,即通过用最少数量的项来扩展给定结构以满足其定律(这里是范畴的定律)来完成该结构的过程。我们将来会看到更多这样的例子。

\section{序关系}

现在来看一些完全不同的东西!一个范畴,其中态射是对象之间的 relation (关系):小于或等于的关系。让我们检查一下它是否确实是一个范畴。我们有恒等态射吗?每个对象都小于或等于自身:检查通过!我们有组合吗?如果 $a \leqslant b$ 且 $b \leqslant c$,那么 $a \leqslant c$:检查通过!组合是 associative (结合的) 吗?检查通过!具有这样关系的集合被称为 \newterm{preorder} (预序),所以一个预序确实是一个范畴。

你也可以有一个更强的关系,它满足一个附加条件:如果 $a \leqslant b$ 且 $b \leqslant a$,那么 $a$ 必须与 $b$ 相同。这被称为 \newterm{partial order} (偏序)。

最后,你可以施加一个条件,即任意两个对象之间都存在某种关系,无论是哪种方式;这就得到了 \newterm{linear order} (线性序) 或 \newterm{total order} (全序)。

让我们将这些有序集刻画为范畴。预序是一个从任何对象 $a$ 到任何对象 $b$ 最多只有一个态射的范畴。这种范畴的另一个名字是“thin (瘦)”。预序是一个瘦范畴。

在一个范畴 $\cat{C}$ 中,从对象 $a$ 到对象 $b$ 的态射集合被称为 \newterm{hom-set} (同态集),并写作 $\cat{C}(a, b)$(有时也写作 $\mathbf{Hom}_{\cat{C}}(a, b)$)。所以在预序中,每个同态集要么是空的,要么是 \newterm{singleton} (单例)。这包括同态集 $\cat{C}(a, a)$,即从 $a$ 到 $a$ 的态射集合,在任何预序中,它必须是一个单例,只包含恒等态射。然而,在预序中你可能有 \newterm{cycles} (循环)。循环在偏序中是被禁止的。

能够识别预序、偏序和全序非常重要,因为这关系到 sorting (排序)。排序算法,如快速排序、冒泡排序、归并排序等,只能在全序上正确工作。偏序可以使用 \newterm{topological sort} (拓扑排序) 进行排序。

\section{作为集合的幺半群}

\newterm{Monoid} (幺半群) 是一个极其简单但异常强大的概念。它是基本算术背后的概念:加法和乘法都构成幺半群。幺半群在编程中无处不在。它们以字符串、列表、可折叠的数据结构、并发编程中的 future、函数式反应式编程中的事件等形式出现。

传统上,幺半群被定义为一个带有 \newterm{binary operation} (二元运算) 的集合。对这个运算的全部要求是它是 associative (结合的),并且存在一个特殊的元素,它相对于该运算表现得像一个 unit (单位元)。

例如,带有零的 \newterm{natural numbers} (自然数) 在加法下形成一个幺半群。结合性意味着:
\[(a + b) + c = a + (b + c)\]
(换句话说,我们可以在加法时省略括号。)

\newterm{neutral element} (中性元) 是零,因为:
\[0 + a = a\]
和
\[a + 0 = a\]
第二个等式是多余的,因为加法是 \newterm{commutative} (交换的) $(a + b = b + a)$,但交换性不是幺半群定义的一部分。例如,\newterm{string concatenation} (字符串连接) 不是交换的,但它仍然形成一个幺半群。顺便说一句,字符串连接的中性元是 \newterm{empty string} (空字符串),它可以附加到字符串的任一侧而不改变它。

在 Haskell 中,我们可以为幺半群定义一个 \newterm{type class} (类型类) —— 一种存在称为 \code{mempty} 的中性元和称为 \code{mappend} 的二元运算的类型:

\src{snippet01}
一个双参数函数的类型签名,\code{m -> m -> m},起初可能看起来很奇怪,但在我们讨论 \newterm{currying} (柯里化) 之后,它将变得完全合理。你可以用两种基本方式解释带有多个箭头的签名:作为一个多参数函数,最右边的类型是返回类型;或者作为一个单参数函数(最左边的那个),返回一个函数。后一种解释可以通过添加括号来强调(括号是多余的,因为箭头是 \newterm{right-associative} (右结合) 的),如:\code{m -> (m -> m)}。我们稍后会回到这种解释。

注意,在 Haskell 中,没有办法表达 \code{mempty} 和 \code{mappend} 的幺半群性质(即 \code{mempty} 是中性元以及 \code{mappend} 是结合的这一事实)。确保它们被满足是程序员的责任。

Haskell 的类不像 C++ 的类那样具有侵入性。当你定义一个新类型时,你不必预先指定它的类。你可以自由地推迟,并在很久以后声明一个给定的类型是某个类的 \newterm{instance} (实例)。例如,让我们通过提供 \code{mempty} 和 \code{mappend} 的实现来声明 \code{String} 是一个幺半群(实际上,这在标准 Prelude 中已经为你完成了):

\src{snippet02}
这里,我们重用了列表连接运算符 \code{(++)},因为 \code{String} 只是一个字符列表。

关于 Haskell 语法的一点说明:任何 \newterm{infix operator} (中缀运算符) 都可以通过用括号括起来将其转换为一个双参数函数。给定两个字符串,你可以通过在它们之间插入 \code{++} 来连接它们:

\begin{snip}{haskell}
"Hello " ++ "world!"
\end{snip}
或者通过将它们作为两个参数传递给带括号的 \code{(++)}:

\begin{snip}{haskell}
(++) "Hello " "world!"
\end{snip}
注意,函数的参数不用逗号分隔,也不用括号括起来。(这可能是学习 Haskell 时最难适应的事情。)

值得强调的是,Haskell 允许你表达函数的相等性,如:

\begin{snip}{haskell}
mappend = (++)
\end{snip}
从概念上讲,这不同于表达函数产生的值的相等性,如:

\begin{snip}{haskell}
mappend s1 s2 = (++) s1 s2
\end{snip}
前者转化为范畴 $\Hask$(或者如果我们忽略 bottom (底值),即永不终止计算的名称,则是 $\Set$)中态射的相等性。这样的等式不仅更简洁,而且通常可以推广到其他范畴。后者被称为 \newterm{extensional equality} (外延相等性),并陈述了这样一个事实:对于任何两个输入字符串,\code{mappend} 和 \code{(++)} 的输出是相同的。由于参数的值有时被称为 \newterm{points} (点)(如:$f$ 在点 $x$ 处的值),这被称为 \newterm{point-wise equality} (逐点相等)。不指定参数的函数相等性被描述为 \newterm{point-free} (无点)。 (顺便说一句,无点等式通常涉及函数组合,而组合是用一个点来表示的,所以这可能会让初学者有点困惑。)

在 C++ 中,最接近声明幺半群的方式是使用 C++20 标准的 \newterm{concept} (概念) 特性。

\begin{snip}{cpp}
template<class T>
struct mempty;

template<class T>
T mappend(T, T) = delete;

template<class M>
concept Monoid = requires (M m) {
    { mempty<M>::value() } -> std::same_as<M>;
    { mappend(m, m) } -> std::same_as<M>;
};
\end{snip}
第一个定义是一个结构体,旨在为每个 \newterm{specialization} (特化) 保存中性元素。

关键字 \code{delete} 意味着没有定义默认值:它必须根据具体情况来指定。类似地,\code{mappend} 也没有默认值。

概念 \code{Monoid} 测试对于给定的类型 \code{M} 是否存在 \code{mempty} 和 \code{mappend} 的适当定义。

Monoid 概念的实例化可以通过提供适当的特化和 \newterm{overloads} (重载) 来完成:

\begin{snip}{cpp}
template<>
struct mempty<std::string> {
    static std::string value() { return ""; }
};

template<>
std::string mappend(std::string s1, std::string s2) {
    return s1 + s2;
}
\end{snip}

\section{作为范畴的幺半群}

那是基于集合元素的“熟悉的”幺半群定义。但是如你所知,在范畴论中,我们试图摆脱集合及其元素,而是讨论对象和态射。所以让我们稍微改变一下视角,把二元运算符的应用想象成在集合中“移动”或“平移”事物。

例如,有将每个自然数加 5 的运算。它将 0 映射到 5,1 映射到 6,2 映射到 7,依此类推。这是一个定义在自然数集合上的函数。这很好:我们有一个函数和一个集合。一般来说,对于任何数字 n,都有一个加 $n$ 的函数—— $n$ 的“加法器”。

加法器如何组合?加 5 的函数与加 7 的函数的组合是一个加 12 的函数。所以加法器的组合可以等同于加法规则。这也很棒:我们可以用函数组合来代替加法。

但是等等,还有更多:还有中性元素零的加法器。加零不会移动任何东西,所以它是自然数集合中的恒等函数。

我可以不给你传统的加法规则,而是给你组合加法器的规则,而不会丢失任何信息。注意,加法器的组合是结合的,因为函数的组合是结合的;并且我们有对应于恒等函数的零加法器。

敏锐的读者可能已经注意到,从整数到加法器的映射源于 \code{mappend} 类型签名的第二种解释,即 \code{m -> (m -> m)}。它告诉我们 \code{mappend} 将幺半群集合的一个元素映射到作用于该集合的一个函数。

现在我希望你忘记你正在处理自然数集合,而只是把它看作一个单一的对象,一个带有一堆态射——即加法器——的团块。幺半群是一个单对象范畴。事实上,monoid 这个名字来源于希腊语 \emph{mono},意思是单一的。每个幺半群都可以被描述为一个单对象范畴,其态射集合遵循适当的组合规则。

\begin{figure}[H]
  \centering
  \includegraphics[width=0.35\textwidth]{images/monoid.jpg}
\end{figure}

\noindent
字符串连接是一个有趣的例子,因为我们可以选择定义右附加器和左附加器(或者你愿意称之为 \emph{前置器})。这两种模型的组合表互为镜像。你可以很容易地说服自己,在 “foo” 之后附加 “bar” 对应于在附加 “bar” 之后前置 “foo”。

你可能会问这样一个问题:是否每个范畴幺半群——一个单对象范畴——都定义了一个唯一的“集合加二元运算符”的幺半群?事实证明,我们总是可以从一个单对象范畴中提取出一个集合。这个集合就是态射的集合——在我们例子中是加法器。换句话说,我们有范畴 $\cat{M}$ 中单个对象 $m$ 的同态集 $\cat{M}(m, m)$。我们可以很容易地在这个集合中定义一个二元运算符:两个集合元素的幺半群乘积是对应于相应态射组合的元素。如果你给我 $\cat{M}(m, m)$ 中对应于 $f$ 和 $g$ 的两个元素,它们的乘积将对应于组合 $f \circ g$。组合总是存在的,因为这些态射的源和目标是同一个对象。并且根据范畴的规则,它是结合的。恒等态射是这个乘积的中性元素。所以我们总是可以从范畴幺半群中恢复出集合幺半群。在所有意图和目的上,它们是同一个东西。

\begin{figure}[H]
  \centering
  \includegraphics[width=0.4\textwidth]{images/monoidhomset.jpg}
  \caption{幺半群同态集既可以看作态射,也可以看作集合中的点。}
\end{figure}

\noindent
只有一个小细节供数学家们挑剔:态射不一定构成一个集合。在范畴的世界里,存在比集合更大的东西。任意两个对象之间的态射构成一个集合的范畴称为 \newterm{locally small} (局部小) 范畴。如承诺的那样,我将主要忽略这些细微之处,但我想我应该提及它们以作记录。

范畴论中的许多有趣现象都源于这样一个事实:同态集的元素既可以被视为遵循组合规则的态射,也可以被视为集合中的点。在这里,$\cat{M}$ 中态射的组合转化为集合 $\cat{M}(m, m)$ 中的幺半群乘积。

\section{挑战}

\begin{enumerate}
  \tightlist
  \item
        从以下图生成自由范畴:

        \begin{enumerate}
          \tightlist
          \item
                一个只有一个节点且没有边的图
          \item
                一个只有一个节点和一条(有向)边的图(提示:这条边可以与自身组合)
          \item
                一个有两个节点和它们之间一条箭头的图
          \item
                一个只有一个节点和 26 条标有字母 a, b, c … z 的箭头的图。
        \end{enumerate}
  \item
        这是哪种类型的序关系?

        \begin{enumerate}
          \tightlist
          \item
                集合的集合,带有包含关系:如果 $A$ 的每个元素也是 $B$ 的元素,则 $A$ 包含在 $B$ 中。
          \item
                C++ 类型,带有以下子类型关系:如果一个指向 \code{T1} 的指针可以传递给一个期望指向 \code{T2} 的指针的函数而不会触发编译错误,则 \code{T1} 是 \code{T2} 的子类型。
        \end{enumerate}
  \item
        考虑到 \code{Bool} 是包含两个值 \code{True} 和 \code{False} 的集合,证明它分别相对于运算符 \code{\&\&} (AND) 和 \code{||} (OR) 形成了两个(集合论意义上的)幺半群。
  \item
        将带有 AND 运算符的 \code{Bool} 幺半群表示为一个范畴:列出态射及其组合规则。
  \item
        将模 3 加法表示为一个幺半群范畴。
\end{enumerate}