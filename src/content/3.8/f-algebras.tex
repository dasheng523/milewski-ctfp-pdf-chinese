% !TEX root = ../../ctfp-print.tex

\lettrine[lhang=0.17]{我}{们已经看到了} 幺半群 (monoid) 的几种表述方式:作为一个集合 (set),作为一个单对象范畴 (single-object category),作为一个幺半范畴 (monoidal category) 中的对象。我们还能从这个简单的概念中榨取出多少东西呢?

让我们试试。以幺半群的这个定义为例,它是一个集合 $m$ 配上一对函数:
\begin{align*}
  \mu  & \Colon m\times{}m \to m \\
  \eta & \Colon 1 \to m
\end{align*}
这里,1 是 $\Set$ 中的终对象 (terminal object) —— 即单例集 (singleton set)。第一个函数定义了乘法(它接受一对元素并返回它们的乘积),第二个函数从 $m$ 中选出单位元 (unit element)。并非所有具有这些签名的函数选择都能构成幺半群。为此,我们需要施加额外的条件:结合律 (associativity) 和单位元定律 (unit laws)。但让我们暂时忽略这些,只考虑“潜在的幺半群 (potential monoids)”。一对函数是两个函数集合的笛卡尔积 (Cartesian product) 的一个元素。我们知道这些集合可以表示为指数对象 (exponential objects):
\begin{align*}
  \mu  & \in m^{m\times{}m} \\
  \eta & \in m^1
\end{align*}
这两个集合的笛卡尔积是:
\[m^{m\times{}m}\times{}m^1\]
使用一些高中代数知识(这在每个笛卡尔闭范畴 (Cartesian closed category) 中都适用),我们可以将其重写为:
\[m^{m\times{}m + 1}\]
$+$ 号代表 $\Set$ 中的余积 (coproduct)。我们刚刚用一个单一的函数 —— 集合的一个元素 —— 替换了一对函数:
\[m\times{}m + 1 \to m\]
这个函数集合中的任何元素都是一个潜在的幺半群。

这种表述方式的优点在于它能引出有趣的推广。例如,我们如何用这种语言描述群 (group)?群是一个幺半群,外加一个为每个元素指定逆元 (inverse) 的附加函数。后者是一个类型为 $m \to m$ 的函数。例如,整数构成一个群,其二元运算是加法,单位元是零,逆元是取反。要定义一个群,我们会从一组三个函数开始:
\begin{align*}
  m\times{}m \to m \\
  m \to m          \\
  1 \to m
\end{align*}
像之前一样,我们可以将这三个函数合并到一个函数集合中:
\[m\times{}m + m + 1 \to m\]
我们从一个二元运算符 (binary operator)(加法)、一个一元运算符 (unary operator)(取反)和一个零元运算符 (nullary operator)(单位元——这里是零)开始。我们将它们组合成了一个函数。所有具有此签名的函数都定义了潜在的群。

我们可以这样继续下去。例如,要定义环 (ring),我们会再增加一个二元运算符和一个零元运算符,等等。每一次我们最终都会得到一个函数类型,其左侧是幂的和(可能包括零次幂——终对象),右侧是集合本身。

现在我们可以疯狂地进行推广了。首先,我们可以用对象 (objects) 替换集合,用态射 (morphisms) 替换函数。我们可以将 n 元运算符定义为来自 n 元积 (n-ary products) 的态射。这意味着我们需要一个支持有限积 (finite products) 的范畴。对于零元运算符,我们需要终对象存在。所以我们需要一个笛卡尔范畴 (Cartesian category)。为了组合这些运算符,我们需要指数对象,所以这是一个笛卡尔闭范畴。最后,我们需要余积来完成我们的代数构造。

或者,我们可以忘记推导公式的方式,专注于最终的产物。我们态射左侧的积之和定义了一个自函子 (endofunctor)。如果我们选择一个任意的自函子 $F$ 会怎样呢?在这种情况下,我们不必对范畴施加任何约束。我们得到的东西称为 F-代数 (F-algebra)。

一个 F-代数是一个三元组,包含一个自函子 $F$,一个对象 $a$,以及一个态射
\[F a \to a\]
这个对象通常称为载体 (carrier)、底层对象 (underlying object),或者在编程语境下,称为载体 \emph{类型} (carrier type)。该态射通常称为求值函数 (evaluation function) 或结构映射 (structure map)。可以将函子 $F$ 看作是形成表达式,而态射则是对它们进行求值。

以下是 F-代数的 Haskell 定义:

\src{snippet01}
它将代数与其求值函数等同起来。

在幺半群的例子中,所讨论的函子是:

\src{snippet02}
这是 $1 + a\times{}a$ 的 Haskell 表示(回想一下 \hyperref[simple-algebraic-data-types]{代数数据结构})。

环将使用以下函子定义:

\src{snippet03}
这是 $1 + 1 + a\times{}a + a\times{}a + a$ 的 Haskell 表示。

环的一个例子是整数集合。我们可以选择 \code{Integer} 作为载体类型,并将求值函数定义为:

\src{snippet04}
基于同一个函子 \code{RingF} 还有更多的 F-代数。例如,多项式 (polynomials) 构成一个环,方阵 (square matrices) 也是。

正如你所见,函子的作用是生成可以用代数的求值器求值的表达式。到目前为止,我们只看到了非常简单的表达式。我们通常对更复杂的、可以使用递归 (recursion) 定义的表达式感兴趣。

\section{递归}

生成任意表达式树的一种方法是用递归替换函子定义内部的变量 \code{a}。例如,环中的任意表达式由这种树状数据结构生成:

\src{snippet05}
我们可以用其递归版本替换原始的环求值器:

\src{snippet06}
这仍然不太实用,因为我们被迫将所有整数表示为 1 的和,但在紧要关头也能用。

但是我们如何使用 F-代数的语言来描述表达式树呢?我们必须以某种方式形式化这个过程:递归地用替换的结果替换我们函子定义中的自由类型变量。想象一下分步进行。首先,将深度为 1 的树定义为:

\src{snippet07}
我们正在用由 \code{RingF a} 生成的深度为 0 的树来填充 \code{RingF} 定义中的“洞”。深度为 2 的树类似地获得为:

\src{snippet08}
我们也可以将其写为:

\src{snippet09}
继续这个过程,我们可以写出一个符号方程:

\begin{snipv}
type RingF\textsubscript{n+1} a = RingF (RingF\textsubscript{n} a)
\end{snipv}
从概念上讲,无限次重复这个过程后,我们最终得到我们的 \code{Expr}。注意 \code{Expr} 不依赖于 \code{a}。我们旅程的起点无关紧要,我们总是到达同一个地方。这对于任意范畴中的任意自函子并非总是成立,但在范畴 $\Set$ 中情况很好。

当然,这是一个粗略的论证,我稍后会使其更严谨。

无限次应用一个自函子会产生一个 \newterm{fixed point (不动点)},该对象定义为:
\[\mathit{Fix}\ f = f\ (\mathit{Fix}\ f)\]
这个定义背后的直觉是,因为我们应用了无限次 $f$ 才得到 $\mathit{Fix}\ f$,再多应用一次也不会改变任何东西。在 Haskell 中,不动点的定义是:

\src{snippet10}
可以说,如果构造器的名称与所定义类型的名称不同,代码会更易读,例如:

\src{snippet11}
但我将坚持使用公认的表示法。构造器 \code{Fix}(或者如果你愿意,也可以是 \code{In})可以看作一个函数:

\src{snippet12}
还有一个函数可以剥离掉一层函子应用:

\src{snippet13}
这两个函数互为逆函数。我们稍后会用到这些函数。

\section{F-代数的范畴}

这是一个老把戏:每当你想到一种构造新对象的方法时,看看它们是否构成一个范畴。不出所料,给定自函子 $F$ 上的代数构成一个范畴。该范畴中的对象是代数——由载体对象 $a$ 和态射 $F a \to a$ 组成的对,两者都来自原始范畴 $\cat{C}$。

为了完成这幅图景,我们必须定义 F-代数范畴中的态射。态射必须将一个代数 $(a, f)$ 映射到另一个代数 $(b, g)$。我们将其定义为一个映射载体的态射 $m$ —— 它在原始范畴中从 $a$ 映到 $b$。并非任何态射都行:我们希望它与两个求值器兼容。(我们称这种保持结构的态射为 \newterm{homomorphism (同态)}。)以下是如何定义 F-代数的同态。首先,注意我们可以将 $m$ 提升为映射:
\[F m \Colon F a \to F b\]
然后我们可以接着应用 $g$ 到达 $b$。等价地,我们可以使用 $f$ 从 $F a$ 到 $a$,然后接着应用 $m$。我们希望这两条路径相等:
\[g \circ F m = m \circ f\]

\begin{figure}[H]
  \centering
  \includegraphics[width=0.3\textwidth]{images/alg.png}
\end{figure}

\noindent
很容易说服自己这确实是一个范畴(提示:来自 $\cat{C}$ 的恒等态射工作得很好,并且同态的组合也是同态)。

F-代数范畴中的初始对象 (initial object),如果存在的话,称为 \newterm{initial algebra (初始代数)}。让我们称这个初始代数的载体为 $i$,其求值器为 $j \Colon F i \to i$。事实证明,初始代数的求值器 $j$ 是一个同构 (isomorphism)。这个结果被称为 Lambek 定理。证明依赖于初始对象的定义,该定义要求从它到任何其他 F-代数存在唯一的同态 $m$。由于 $m$ 是一个同态,以下图表必须交换:

\begin{figure}[H]
  \centering
  \includegraphics[width=0.3\textwidth]{images/alg2.png}
\end{figure}

\noindent
现在让我们构造一个载体为 $F i$ 的代数。这样一个代数的求值器必须是从 $F (F i)$ 到 $F i$ 的态射。我们可以简单地通过提升 $j$ 来构造这样一个求值器:
\[F j \Colon F (F i) \to F i\]
因为 $(i, j)$ 是初始代数,所以必须存在一个从它到 $(F i, F j)$ 的唯一同态 $m$。以下图表必须交换:

\begin{figure}[H]
  \centering
  \includegraphics[width=0.3\textwidth]{images/alg3a.png}
\end{figure}

\noindent
但我们也有这个显然交换的图表(两条路径相同!):

\begin{figure}[H]
  \centering
  \includegraphics[width=0.3\textwidth]{images/alg3.png}
\end{figure}

\noindent
这可以解释为表明 $j$ 是一个代数同态,将 $(F i, F j)$ 映射到 $(i, j)$。我们可以将这两个图表粘合在一起得到:

\begin{figure}[H]
  \centering
  \includegraphics[width=0.6\textwidth]{images/alg4.png}
\end{figure}

\noindent
这个图表反过来可以解释为表明 $j \circ m$ 是一个代数同态。只是在这种情况下,两个代数是相同的。此外,因为 $(i, j)$ 是初始的,所以从它到自身的同态只能有一个,那就是恒等态射 $\id_i$ —— 我们知道它是一个代数同态。因此 $j \circ m = \id_i$。利用这个事实和左图的交换性质,我们可以证明 $m \circ j = \id_{Fi}$。这表明 $m$ 是 $j$ 的逆,因此 $j$ 是 $F i$ 和 $i$ 之间的一个同构:
\[F i \cong i\]
但这只是说 $i$ 是 $F$ 的一个不动点。这就是最初粗略论证背后的形式化证明。

回到 Haskell:我们认出 $i$ 就是我们的 \code{Fix f},$j$ 就是我们的构造器 \code{Fix},它的逆就是 \code{unFix}。Lambek 定理中的同构告诉我们,为了得到初始代数,我们取函子 $f$ 并将其参数 $a$ 替换为 \code{Fix f}。我们也明白了为什么不动点不依赖于 $a$。

\section{自然数}

自然数 (Natural numbers) 也可以定义为一个 F-代数。起点是一对态射:
\begin{align*}
  zero & \Colon 1 \to N \\
  succ & \Colon N \to N
\end{align*}
第一个选择零,第二个将所有数映射到它们的后继者 (successors)。像之前一样,我们可以将两者合并为一个:
\[1 + N \to N\]
左侧定义了一个函子,在 Haskell 中可以这样写:

\src{snippet14}
这个函子的不动点(它生成的初始代数)可以用 Haskell 编码如下:

\src{snippet15}
自然数要么是零,要么是另一个数的后继者。这被称为自然数的皮亚诺表示法 (Peano representation)。

\section{范畴态射 (Catamorphisms)}

让我们用 Haskell 表示法重写初始性条件。我们称初始代数为 \code{Fix f}。它的求值器是构造器 \code{Fix}。从初始代数到同一函子上的任何其他代数,存在唯一的态射 \code{m}。让我们选择一个载体为 \code{a}、求值器为 \code{alg} 的代数。

\begin{figure}[H]
  \centering
  \includegraphics[width=0.4\textwidth]{images/alg5.png}
\end{figure}

\noindent
顺便说一句,注意 \code{m} 是什么:它是不动点的求值器,是整个递归表达式树的求值器。让我们找到一种通用的实现方法。

Lambek 定理告诉我们构造器 \code{Fix} 是一个同构。我们称它的逆为 \code{unFix}。因此我们可以翻转此图中的一个箭头得到:

\begin{figure}[H]
  \centering
  \includegraphics[width=0.4\textwidth]{images/alg6.png}
\end{figure}

\noindent
让我们写下这个图的交换条件:

\begin{snip}{haskell}
m = alg . fmap m . unFix
\end{snip}
我们可以将这个方程解释为 \code{m} 的递归定义。对于使用函子 \code{f} 创建的任何有限树,递归必然会终止。我们可以通过注意到 \code{fmap m} 作用在函子 \code{f} 的顶层之下来看到这一点。换句话说,它作用于原始树的子节点。子节点总是比原始树浅一层。

以下是将 \code{m} 应用于使用 \code{Fix\ f} 构建的树时发生的情况。\code{unFix} 的作用是剥离构造器,暴露出树的顶层。然后我们将 \code{m} 应用于顶层节点的所有子节点。这会产生类型为 \code{a} 的结果。最后,我们通过应用非递归求值器 \code{alg} 来组合这些结果。关键点在于我们的求值器 \code{alg} 是一个简单的非递归函数。

因为我们可以对任何代数 \code{alg} 这样做,所以定义一个高阶函数是有意义的,该函数接受代数作为参数并给出我们称为 \code{m} 的函数。这个高阶函数称为范畴态射 (catamorphism):

\src{snippet16}
让我们看一个例子。以定义自然数的函子为例:

\src{snippet17}
让我们选择 \code{(Int, Int)} 作为载体类型,并将我们的代数定义为:

\src{snippet18}
你很容易可以让自己相信,这个代数的范畴态射 \code{cata fib} 计算的是斐波那契数 (Fibonacci numbers)。

一般来说,\code{NatF} 的代数定义了一个递推关系 (recurrence relation):当前元素的值用前一个元素表示。然后范畴态射计算该序列的第 n 个元素。

\section{折叠 (Folds)}

\code{e} 的列表 (list) 是以下函子的初始代数:

\src{snippet19}
确实,用递归的结果(我们称之为 \code{List e})替换变量 \code{a},我们得到:

\src{snippet20}
列表函子的代数选择一个特定的载体类型,并定义一个对两个构造器进行模式匹配 (pattern matching) 的函数。它对于 \code{NilF} 的值告诉我们如何求值空列表,而它对于 \code{ConsF} 的值告诉我们如何将当前元素与先前累积的值组合起来。

例如,这里有一个可以用来计算列表长度的代数(载体类型是 \code{Int}):

\src{snippet21}
确实,由此产生的范畴态射 \code{cata lenAlg} 计算了列表的长度。注意,求值器是 (1) 一个接受列表元素和累加器 (accumulator) 并返回新累加器的函数,以及 (2) 一个起始值(这里是零)的组合。值的类型和累加器的类型由载体类型给出。

将此与传统的 Haskell 定义进行比较:

\src{snippet22}
\code{foldr} 的两个参数恰好是代数的两个组成部分。

让我们试试另一个例子:

\src{snippet23}
再次,将此与以下进行比较:

\src{snippet24}
如你所见,\code{foldr} 只是范畴态射针对列表的一种方便的特化。

\section{余代数 (Coalgebras)}

像往常一样,我们有 F-余代数 (F-coalgebra) 的对偶构造,其中态射的方向相反:
\[a \to F a\]
给定函子的余代数也构成一个范畴,其同态保持余代数结构 (coalgebraic structure)。该范畴中的终对象 (terminal object) $(t, u)$ 称为终末 (terminal)(或最终 (final))余代数。对于每个其他余代数 $(a, f)$,存在唯一的同态 $m$ 使得以下图表交换:

\begin{figure}[H]
  \centering
  \includegraphics[width=0.4\textwidth]{images/alg7.png}
\end{figure}

\noindent
终末余代数是函子的不动点,意义在于态射 $u \Colon t \to F t$ 是一个同构(余代数的 Lambek 定理):
\[F t \cong t\]
终末余代数在编程中通常被解释为生成(可能无限的)数据结构或转移系统 (transition systems) 的配方。

就像范畴态射可以用来求值初始代数一样,展开态射 (anamorphism) 可以用来共求值 (coevaluate) 终末余代数:

\src{snippet25}
余代数的一个典型例子基于这样一个函子:其不动点是类型为 \code{e} 的元素的无限流。这就是那个函子:

\src{snippet26}
而这是它的不动点:

\src{snippet27}
\code{StreamF e} 的余代数是一个函数,它接受类型为 \code{a} 的种子 (seed),并产生一个由一个元素和下一个种子组成的对(\code{StreamF} 是对 (pair) 的一个花哨名称)。

你可以轻松生成产生无限序列的余代数的简单示例,比如平方数列表或倒数列表。

一个更有趣的例子是产生素数 (primes) 列表的余代数。诀窍是使用无限列表作为载体。我们的起始种子将是列表 \code{{[}2..{]}}。下一个种子将是此列表移除所有 2 的倍数后的尾部。这是一个从 3 开始的奇数列表。下一步,我们将取此列表的尾部并移除所有 3 的倍数,依此类推。你可能会认出这是埃拉托斯特尼筛法 (sieve of Eratosthenes) 的雏形。这个余代数由以下函数实现:

\src{snippet28}
这个余代数的展开态射生成素数列表:

\src{snippet29}
流是一个无限列表,所以应该可以将其转换为 Haskell 列表。为此,我们可以使用相同的函子 \code{StreamF} 来形成一个代数,并在其上运行一个范畴态射。例如,这是一个将流转换为列表的范畴态射:

\src{snippet30}
在这里,同一个不动点同时是同一个自函子的初始代数和终末余代数。在任意范畴中,情况并非总是如此。通常,一个自函子可能有多个(或没有)不动点。初始代数是所谓的最小不动点 (least fixed point),而终末余代数是最大不动点 (greatest fixed point)。然而,在 Haskell 中,两者都由相同的公式定义,并且它们是重合的。

列表的展开态射称为 unfold (展开)。为了创建有限列表,函子被修改为产生一个 \code{Maybe} 对:

\src{snippet31}
\code{Nothing} 的值将终止列表的生成。

余代数的一个有趣案例与透镜 (lenses) 相关。透镜可以表示为 getter (获取器) 和 setter (设置器) 的一对:

\src{snippet32}
这里,\code{a} 通常是某个具有 \code{s} 类型字段的积数据类型。获取器检索该字段的值,设置器用新值替换该字段。这两个函数可以合并为一个:

\src{snippet33}
我们可以将此函数进一步重写为:

\src{snippet34}
其中我们定义了一个函子:

\src{snippet35}
注意,这不是一个由积之和构造的简单代数函子。它涉及一个指数 $a^s$。

透镜是这个函子的一个余代数,其载体类型为 \code{a}。我们之前已经看到 \code{Store s} 也是一个余单子。事实证明,一个行为良好 (well-behaved) 的透镜对应于一个与余单子结构兼容的余代数。我们将在下一节讨论这个问题。

\section{挑战}

\begin{enumerate}
  \tightlist
  \item
        实现单变量多项式环的求值函数。你可以将多项式表示为 $x$ 幂次前的系数列表。例如,$4x^2-1$ 将表示为(从零次幂开始)\code{{[}-1, 0, 4{]}}。
  \item
        将先前的构造推广到多个独立变量的多项式,例如 $x^2y-3y^3z$。
  \item
        实现 $2\times{}2$ 矩阵环的代数。
  \item
        定义一个余代数,其展开态射产生自然数平方的列表。
  \item
        使用 \code{unfoldr} 生成前 $n$ 个素数的列表。
\end{enumerate}