% !TEX root = ctfp-print.tex

\lettrine[lhang=0.17]{这}{是 \emph{Category Theory for Programmers}} (《给程序员的范畴论》) 的 Scala 版本。本书取得了巨大的成功,它将 Bartosz Milewski 的系列博文整理成排版精美的 \acronym{PDF} 文档,并出版了精装实体书。众多贡献者通过修正笔误与错误、以及将代码片段 (code snippets) 翻译成其他编程语言等方式,为改进本书做出了贡献。

我非常激动能向大家呈献本书的这一版本,其中包含原始的 Haskell 代码,其后附有对应的 Scala 代码。Scala 代码片段由 \urlref{https://github.com/typelevel/CT_from_Programmers.scala}{Typelevel 贡献者} 慷慨惠赠,并根据本书的格式进行了少量修改。

为了支持多种语言的代码片段,我使用了一个 \LaTeX{} 宏 (macro) 从外部文件 (external files) 加载代码片段。这种方式便于用其他语言扩展本书,同时保持原文本内容的原封不动。因此,每当您在文中看到 “in Haskell” 时,请在心里补充上 “以及 Scala”。

代码按以下方式排版:首先是原始的 Haskell 代码,其后是 Scala 代码。为了区分两者,代码片段的左侧标有一条竖线,其颜色分别为相应语言 logo 的主色调,即 Haskell \raisebox{-.2mm}{\includegraphics[height=.3cm]{fig/icons/haskell.png}} 和 Scala \raisebox{-.2mm}{\includegraphics[height=.3cm]{fig/icons/scala.png}},例如:

\srcsnippet{content/3.6/code/haskell/snippet03.hs}{blue}{haskell}
\unskip
\srcsnippet{content/3.6/code/scala/snippet03.scala}{red}{scala}
\NoIndentAfterThis
此外,一些 Scala 代码片段使用了 \urlref{https://github.com/non/kind-projector}{Kind Projector} 编译器插件 (compiler plugin),以支持更美观的部分应用类型 (partially-applied types) 语法 (syntax)。