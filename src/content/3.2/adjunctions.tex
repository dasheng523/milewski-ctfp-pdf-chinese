% !TEX root = ../../ctfp-print.tex

\lettrine[lhang=0.17]{在}{数学中,我们有}各种方式来说明一个事物与另一个事物相似。最严格的是相等 (equality)。如果无法区分两个事物,则它们是相等的。在任何可以想象的语境中,一个都可以替换另一个。例如,你是否注意到每当我们谈论可换图 (commuting diagrams) 时,我们都使用了态射 (morphisms) 的 \newterm{相等}?这是因为态射形成一个集合 (hom-set),而集合元素可以比较是否相等。

但相等往往过于严格。有很多例子表明,事物在所有意图和目的上都是相同的,但实际上并不相等。例如,序对类型 \code{(Bool, Char)} 与 \code{(Char, Bool)} 并非严格相等,但我们理解它们包含相同的信息。这个概念最好用两种类型之间的 \newterm{同构 (isomorphism)}——一个可逆的态射——来捕捉。由于它是一个态射,它保持结构;而“同构”意味着它是往返过程的一部分,无论你从哪一边开始,最终都会回到同一个地方。在序对的情况下,这个同构称为 \code{swap}:

\src{snippet01}
\code{swap} 恰好是它自身的逆 (inverse)。

\section{伴随与单元/余单元对}

当我们谈论范畴是同构的时,我们是用范畴之间的映射,也就是函子 (functors),来表达的。我们希望能够说,如果存在一个从 $\cat{C}$ 到 $\cat{D}$ 的函子 $R$(“右”),它是可逆的,那么两个范畴 $\cat{C}$ 和 $\cat{D}$ 是同构的。换句话说,存在另一个从 $\cat{D}$ 回到 $\cat{C}$ 的函子 $L$(“左”),当它与 $R$ 复合时,等于恒等函子 (identity functor) $I$。有两种可能的复合,$R \circ L$ 和 $L \circ R$;以及两个可能的恒等函子:一个在 $\cat{C}$ 中,另一个在 $\cat{D}$ 中。

\begin{figure}[H]
  \centering
  \includegraphics[width=0.5\textwidth]{images/adj-1.jpg}
\end{figure}

\noindent
但棘手的部分在这里:两个函子 \emph{相等} 是什么意思?我们说的这个等式是什么意思:
\[R \circ L = I_{\cat{D}}\]
或者这个:
\[L \circ R = I_{\cat{C}}\]
根据对象的相等来定义函子相等似乎是合理的。两个函子作用于相等的对象时,应该产生相等的对象。但我们通常在任意范畴中没有对象相等的概念。它根本不是定义的一部分。(更深入地探讨“相等到底是什么”这个兔子洞,我们最终会到达同伦类型论 (Homotopy Type Theory)。)

你可能会争辩说函子 \emph{是} 范畴之范畴 ($\Cat$) 中的态射,所以它们应该是可比较相等的。确实,只要我们讨论的是小范畴 (small categories),其中对象形成一个集合,我们确实可以使用集合元素的相等性来比较对象的相等性。

但是,请记住,$\Cat$ 实际上是一个 $\cat{2}$-范畴 ($\cat{2}$-category)。$\cat{2}$-范畴中的 hom-集具有额外的结构——存在作用于 $1$-态射 ($1$-morphisms) 之间的 $2$-态射 ($2$-morphisms)。在 $\Cat$ 中,$1$-态射是函子,而 $2$-态射是自然变换 (natural transformations)。因此,在谈论函子时,将自然同构 (natural isomorphisms) 作为相等的替代品更为自然(无法避免这个双关语!)。

因此,与其考虑范畴的同构,不如考虑一个更普遍的概念——\newterm{等价 (equivalence)}。如果我们可以找到两个在它们之间来回映射的函子,它们的复合(无论哪种方式)都与恒等函子 \newterm{自然同构 (naturally isomorphic)},那么两个范畴 $\cat{C}$ 和 $\cat{D}$ 是 \emph{等价的}。换句话说,在复合 $R \circ L$ 和恒等函子 $I_{\cat{D}}$ 之间存在一个双向的自然变换,在 $L \circ R$ 和恒等函子 $I_{\cat{C}}$ 之间也存在另一个。

伴随 (Adjunction) 比等价更弱,因为它不要求两个函子的复合与恒等函子 \emph{同构}。相反,它规定了存在一个从 $I_{\cat{D}}$ 到 $R \circ L$ 的 \newterm{单向 (one way)} 自然变换,以及另一个从 $L \circ R$ 到 $I_{\cat{C}}$ 的自然变换。这两个自然变换的签名如下:
\begin{gather*}
  \eta \Colon I_{\cat{D}} \to R \circ L \\
  \varepsilon \Colon L \circ R \to I_{\cat{C}}
\end{gather*}
$\eta$ 被称为伴随的单元 (unit),$\varepsilon$ 被称为伴随的余单元 (counit)。

注意这两个定义之间的不对称性。通常,我们没有另外两个映射:
\begin{gather*}
  R \circ L \to I_{\cat{D}} \quad\quad\text{不一定存在} \\
  I_{\cat{C}} \to L \circ R \quad\quad\text{不一定存在}
\end{gather*}
由于这种不对称性,函子 $L$ 被称为函子 $R$ 的 \newterm{左伴随 (left adjoint)},而函子 $R$ 是 $L$ 的右伴随 (right adjoint)。(当然,左和右只有在你以特定的方式绘制图表时才有意义。)

伴随的紧凑表示法是:
\[L \dashv R\]
为了更好地理解伴随,让我们更详细地分析单元和余单元。

\begin{figure}[H]
  \centering
  \includegraphics[width=0.5\textwidth]{images/adj-unit.jpg}
\end{figure}

\noindent
让我们从单元开始。它是一个自然变换,所以它是一族态射。给定 $\cat{D}$ 中的一个对象 $d$,$\eta$ 的分量 (component) 是 $I d$(等于 $d$)和 $(R \circ L) d$(在图中称为 $d'$)之间的一个态射:
\[\eta_d \Colon d \to (R \circ L) d\]
注意复合 $R \circ L$ 是 $\cat{D}$ 中的一个自函子 (endofunctor)。

这个等式告诉我们,我们可以选择 $\cat{D}$ 中的任何对象 $d$ 作为起点,并使用往返函子 $R \circ L$ 来选择我们的目标对象 $d'$。然后我们射出一支箭——态射 $\eta_d$——指向我们的目标。

\begin{figure}[H]
  \centering
  \includegraphics[width=0.5\textwidth]{images/adj-counit.jpg}
\end{figure}

\noindent
同理,余单元 $\varepsilon$ 的分量可以描述为:
\[\varepsilon_{c} \Colon (L \circ R) c \to c\]
它告诉我们,我们可以选择 $\cat{C}$ 中的任何对象 $c$ 作为我们的目标,并使用往返函子 $L \circ R$ 来选择源 $c' = (L \circ R) c$。然后我们从源射出箭——态射 $\varepsilon_{c}$——指向目标。

看待单元和余单元的另一种方式是,单元允许我们在任何可以插入 $\cat{D}$ 上恒等函子的地方 \emph{引入 (introduce)} 复合 $R \circ L$;而余单元允许我们 \emph{消除 (eliminate)} 复合 $L \circ R$,用 $\cat{C}$ 上的恒等替换它。这导致了一些“显而易见”的一致性条件,确保引入后紧接着消除不会改变任何东西:
\begin{gather*}
  L = L \circ I_{\cat{D}} \to L \circ R \circ L \to I_{\cat{C}} \circ L = L \\
  R = I_{\cat{D}} \circ R \to R \circ L \circ R \to R \circ I_{\cat{C}} = R
\end{gather*}
这些被称为三角等式 (triangular identities),因为它们使得以下图表可换:

\begin{figure}[H]
  \centering

  \begin{subfigure}
    \centering
    \begin{tikzcd}[column sep=large, row sep=large]
      L \arrow[rd, equal] \arrow[r, "L \circ \eta"]
      & L \circ R \circ L \arrow[d, "\varepsilon \circ L"] \\
      & L
    \end{tikzcd}
  \end{subfigure}%
  \hspace{1cm}
  \begin{subfigure}
    \centering
    \begin{tikzcd}[column sep=large, row sep=large]
      R \arrow[rd, equal] \arrow[r, "\eta \circ R"]
      & R \circ L \circ R \arrow[d, "R \circ \varepsilon"] \\
      & R
    \end{tikzcd}
  \end{subfigure}
\end{figure}

\noindent
这些是函子范畴中的图表:箭头是自然变换,它们的复合是自然变换的水平复合 (horizontal composition)。在分量形式中,这些等式变为:
\begin{gather*}
  \varepsilon_{L d} \circ L \eta_d = \id_{L d} \\
  R \varepsilon_{c} \circ \eta_{R c} = \id_{R c}
\end{gather*}
我们经常在 Haskell 中以不同的名称看到单元和余单元。单元被称为 \code{return}(或者在 \code{Applicative} 的定义中称为 \code{pure}):

\src{snippet02}
而余单元被称为 \code{extract}:

\src{snippet03}
这里,\code{m} 是对应于 $R \circ L$ 的(自)函子,而 \code{w} 是对应于 $L \circ R$ 的(自)函子。正如我们稍后将看到的,它们分别是单子 (monad) 和余单子 (comonad) 定义的一部分。

如果你将自函子想象成一个容器,那么单元(或 \code{return})是一个多态函数,它围绕任意类型的值创建一个默认的盒子。余单元(或 \code{extract})则相反:它从容器中检索或产生单个值。

我们稍后将看到,每对伴随函子都定义了一个单子和一个余单子。反过来,每个单子或余单子都可以分解为一对伴随函子——尽管这种分解不是唯一的。

在 Haskell 中,我们经常使用单子,但很少将它们分解为伴随函子对,主要是因为那些函子通常会带我们离开 $\Hask$。

然而,我们可以在 Haskell 中定义 \newterm{自函子 (endofunctors)} 的伴随。这是从 \code{Data.Functor.Adjunction} 中获取的部分定义:

\src{snippet04}
这个定义需要一些解释。首先,它描述了一个多参数类型类 (multi-parameter type class)——两个参数是 \code{f} 和 \code{u}。它在这些类型构造器 (type constructors) 之间建立了一个称为 \code{Adjunction} 的关系。

竖线后面的附加条件指定了函数依赖 (functional dependencies)。例如,\code{f -> u} 意味着 \code{u} 由 \code{f} 确定(\code{f} 和 \code{u} 之间的关系是一个函数,这里作用于类型构造器)。反过来,\code{u -> f} 意味着,如果我们知道 \code{u},那么 \code{f} 就被唯一确定了。

我稍后会解释为什么在 Haskell 中,我们可以施加条件,要求右伴随 \code{u} 是一个 \newterm{可表示 (representable)} 函子。

\section{伴随与 Hom-集}

伴随的等价定义是基于 hom-集的自然同构。这个定义与我们到目前为止研究的泛构造 (universal constructions) 很好地联系在一起。每当你听到存在某个唯一态射,它分解了某个构造的陈述时,你应该将其视为某个集合到 hom-集 的映射。这就是“选择唯一态射”的含义。

此外,分解 (factorization) 通常可以用自然变换来描述。分解涉及可换图——某个态射等于两个态射(因子)的复合。自然变换将态射映射到可换图。因此,在泛构造中,我们从一个态射到一个可换图,然后再到一个唯一的态射。我们最终得到一个从态射到态射的映射,或者说从一个 hom-集到另一个(通常在不同的范畴中)的映射。如果这个映射是可逆的,并且可以自然地扩展到所有 hom-集,我们就得到了一个伴随。

泛构造和伴随之间的主要区别在于后者是全局定义的——对所有 hom-集都成立。例如,使用泛构造,你可以定义两个选定对象的积,即使该范畴中任何其他对象对的积不存在。正如我们很快将看到的,如果一个范畴中 \emph{任何一对} 对象的积都存在,那么它也可以通过伴随来定义。

\begin{figure}[H]
  \centering
  \includegraphics[width=0.5\textwidth]{images/adj-homsets.jpg}
\end{figure}

\noindent
这是使用 hom-集的伴随的替代定义。和以前一样,我们有两个函子 $L \Colon \cat{D} \to \cat{C}$ 和 $R \Colon \cat{C} \to \cat{D}$。我们选择两个任意对象:$\cat{D}$ 中的源对象 (source object) $d$,和 $\cat{C}$ 中的目标对象 (target object) $c$。我们可以使用 $L$ 将源对象 $d$ 映射到 $\cat{C}$。现在我们在 $\cat{C}$ 中有两个对象,$L d$ 和 $c$。它们定义了一个 hom-集:
\[\cat{C}(L d, c)\]
类似地,我们可以使用 $R$ 映射目标对象 $c$。现在我们在 $\cat{D}$ 中有两个对象,$d$ 和 $R c$。它们也定义了一个 hom 集:
\[\cat{D}(d, R c)\]
我们说 $L$ 是 $R$ 的左伴随,当且仅当存在一个 hom 集的同构:
\[\cat{C}(L d, c) \cong \cat{D}(d, R c)\]
这个同构在 $d$ 和 $c$ 两方面都是自然的。自然性意味着源 $d$ 可以在 $\cat{D}$ 中平滑变化;目标 $c$ 可以在 $\cat{C}$ 中平滑变化。更准确地说,我们有以下两个从 $\cat{C}$ 到 $\Set$ 的(协变)函子之间的自然变换 $\varphi$。以下是这些函子在对象上的作用:
\begin{gather*}
  c \to \cat{C}(L d, c) \\
  c \to \cat{D}(d, R c)
\end{gather*}
另一个自然变换 $\psi$ 作用于以下(逆变)函子之间:
\begin{gather*}
  d \to \cat{C}(L d, c) \\
  d \to \cat{D}(d, R c)
\end{gather*}
这两个自然变换都必须是可逆的。

很容易证明伴随的两种定义是等价的。例如,让我们从 hom-集的同构开始推导单元变换:
\[\cat{C}(L d, c) \cong \cat{D}(d, R c)\]
由于这个同构对任何对象 $c$ 都有效,它也必须对 $c = L d$ 有效:
\[\cat{C}(L d, L d) \cong \cat{D}(d, (R \circ L) d)\]
我们知道左侧必须至少包含一个态射,即恒等态射。自然变换将这个态射映射到 $\cat{D}(d, (R \circ L) d)$ 的一个元素,或者,插入恒等函子 $I$,即以下集合中的一个态射:
\[\cat{D}(I d, (R \circ L) d)\]
我们得到一族由 $d$ 参数化的态射。它们构成了函子 $I$ 和函子 $R \circ L$ 之间的自然变换(自然性条件很容易验证)。这正是我们的单元 $\eta$。

反过来,从单元和余单元的存在出发,我们可以定义 hom-集之间的变换。例如,让我们选择 hom-集 $\cat{C}(L d, c)$ 中的任意态射 $f$。我们想定义一个 $\varphi$,作用于 $f$ 时,产生 $\cat{D}(d, R c)$ 中的一个态射。

实际上没有太多选择。我们可以尝试的一件事是使用 $R$ 来提升 $f$。这将产生一个从 $R (L d)$ 到 $R c$ 的态射 $R f$——这个态射是 $\cat{D}((R \circ L) d, R c)$ 的一个元素。

我们需要的 $\varphi$ 的分量是从 $d$ 到 $R c$ 的一个态射。这不是问题,因为我们可以使用 $\eta_d$ 的一个分量从 $d$ 到达 $(R \circ L) d$。我们得到:
\[\varphi_f = R f \circ \eta_d\]
另一个方向是类似的,$\psi$ 的推导也是如此。

回到 \code{Adjunction} 的 Haskell 定义,自然变换 $\varphi$ 和 $\psi$ 分别被多态(在 \code{a} 和 \code{b} 上)函数 \code{leftAdjunct} 和 \code{rightAdjunct} 替代。函子 $L$ 和 $R$ 被称为 \code{f} 和 \code{u}:

\src{snippet05}
\code{unit}/\code{counit} 表述与 \code{leftAdjunct}/\allowbreak\code{rightAdjunct} 表述之间的等价性由以下映射证明:

\src{snippet06}
将伴随的范畴论描述翻译成 Haskell 代码是非常有启发性的。我强烈建议将此作为练习。

我们现在准备解释为什么在 Haskell 中,右伴随自动成为一个 \hyperref[representable-functors]{可表示函子}。原因在于,作为第一近似,我们可以将 Haskell 类型的范畴视为集合的范畴。

当右范畴 $\cat{D}$ 是 $\Set$ 时,右伴随 $R$ 是一个从 $\cat{C}$ 到 $\Set$ 的函子。如果我们能找到 $\cat{C}$ 中的一个对象 $\mathit{rep}$,使得 hom-函子 $\cat{C}(\mathit{rep}, \_)$ 与 $R$ 自然同构,则这样的函子是可表示的。事实证明,如果 $R$ 是某个从 $\Set$ 到 $\cat{C}$ 的函子 $L$ 的右伴随,那么这样的对象总是存在——它是单元集 $()$ 在 $L$ 下的像:
\[\mathit{rep} = L ()\]
确实,伴随告诉我们以下两个 hom-集是自然同构的:
\[\cat{C}(L (), c) \cong \Set((), R c)\]
对于给定的 $c$,右侧是从单元集 $()$ 到 $R c$ 的函数集合。我们之前已经看到,每个这样的函数都从集合 $R c$ 中挑选一个元素。这样的函数集合与集合 $R c$ 同构。所以我们有:
\[\cat{C}(L (), -) \cong R\]
这表明 $R$ 确实是可表示的。

\section{从伴随看积}

我们之前使用泛构造引入了几个概念。其中许多概念,当全局定义时,使用伴随更容易表达。最简单的非平凡例子是积 (product)。\hyperref[products-and-coproducts]{积的泛构造} 的要点是能够通过泛积来分解任何类似积的候选者。

更准确地说,两个对象 $a$ 和 $b$ 的积是对象 $(a\times{}b)$(或 Haskell 表示法中的 \code{(a, b)}),配备两个态射 $\mathit{fst}$ 和 $\mathit{snd}$,使得对于任何其他配备两个态射 $p \Colon c \to a$ 和 $q \Colon c \to b$ 的候选者 $c$,存在唯一的态射 $m \Colon c \to (a, b)$,它通过 $\mathit{fst}$ 和 $\mathit{snd}$ 分解 $p$ 和 $q$。

正如我们之前看到的,在 Haskell 中,我们可以实现一个 \code{factorizer} (分解子),它从两个投影 (projections) 生成这个态射:

\src{snippet07}
很容易验证分解条件成立:

\src{snippet08}
我们有一个映射,它接受一对态射 \code{p} 和 \code{q},并产生另一个态射 \code{m = factorizer p q}。

我们如何将其转化为定义伴随所需的两个 hom-集之间的映射?诀窍是走出 $\Hask$,将这对态射视为积范畴 (product category) 中的单个态射。

让我提醒你什么是积范畴。取两个任意范畴 $\cat{C}$ 和 $\cat{D}$。积范畴 $\cat{C}\times{}\cat{D}$ 中的对象是对象对,一个来自 $\cat{C}$,一个来自 $\cat{D}$。态射是态射对,一个来自 $\cat{C}$,一个来自 $\cat{D}$。

要在某个范畴 $\cat{C}$ 中定义积,我们应该从积范畴 $\cat{C}\times{}\cat{C}$ 开始。来自 $\cat{C}$ 的态射对是积范畴 $\cat{C}\times{}\cat{C}$ 中的单个态射。

\begin{figure}[H]
  \centering
  \includegraphics[width=0.5\textwidth]{images/adj-productcat.jpg}
\end{figure}

\noindent
起初可能会有点混淆,我们使用积范畴来定义积。然而,这些是非常不同的积。我们不需要泛构造来定义积范畴。我们所需要的只是对象对和态射对的概念。

然而,来自 $\cat{C}$ 的对象对 \emph{不是} $\cat{C}$ 中的对象。它是不同范畴 $\cat{C}\times{}\cat{C}$ 中的对象。我们可以将这对形式化地写为 $\langle a, b \rangle$,其中 $a$ 和 $b$ 是 $\cat{C}$ 的对象。另一方面,泛构造是必要的,以便定义对象 $a\times{}b$(或 Haskell 中的 \code{(a, b)}),它是 \emph{同一个} 范畴 $\cat{C}$ 中的对象。这个对象应该以泛构造指定的方式表示序对 $\langle a, b \rangle$。它并不总是存在,即使它对某些对象对存在,也可能对 $\cat{C}$ 中的其他对象对不存在。

现在让我们将 \code{factorizer} 视为 hom-集的映射。第一个 hom-集在积范畴 $\cat{C}\times{}\cat{C}$ 中,第二个在 $\cat{C}$ 中。$\cat{C}\times{}\cat{C}$ 中的一般态射将是一对态射 $\langle f, g \rangle$:
\begin{gather*}
  f \Colon c' \to a \\
  g \Colon c'' \to b
\end{gather*}
其中 $c''$ 可能与 $c'$ 不同。但是为了定义积,我们感兴趣的是 $\cat{C}\times{}\cat{C}$ 中的一个特殊态射,即共享相同源对象 $c$ 的序对 $p$ 和 $q$。没关系:在伴随的定义中,左侧 hom-集的源不是任意对象——它是左函子 $L$ 作用于右范畴某个对象的结果。符合要求的函子很容易猜到——它是从 $\cat{C}$ 到 $\cat{C}\times{}\cat{C}$ 的对角函子 (diagonal functor) $\Delta$,其在对象上的作用是:
\[\Delta c = \langle c, c \rangle\]
因此,我们伴随中的左侧 hom-集应该是:
\[(\cat{C}\times{}\cat{C})(\Delta c, \langle a, b \rangle)\]
它是积范畴中的一个 hom-集。它的元素是我们识别为 \code{factorizer} 参数的态射对:
\[(c \to a) \to (c \to b) \ldots{}\]
右侧 hom-集存在于 $\cat{C}$ 中,它介于源对象 $c$ 和某个函子 $R$ 作用于 $\cat{C}\times{}\cat{C}$ 中目标对象的结果之间。这个函子将序对 $\langle a, b \rangle$ 映射到我们的积对象 $a\times{}b$。我们将这个 hom-集的元素识别为 \code{factorizer} 的 \emph{结果}:
\[\ldots{} \to (c \to (a, b))\]

\begin{figure}[H]
  \centering
  \includegraphics[width=0.5\textwidth]{images/adj-product.jpg}
\end{figure}

\noindent
我们仍然没有一个完整的伴随。为此,我们首先需要我们的 \code{factorizer} 是可逆的——我们正在构建 hom-集之间的 \emph{同构}。\code{factorizer} 的逆应该从一个态射 $m$ 开始——一个从某个对象 $c$ 到积对象 $a\times{}b$ 的态射。换句话说,$m$ 应该是以下集合的元素:
\[\cat{C}(c, a\times{}b)\]
逆分解子 (inverse factorizer) 应该将 $m$ 映射到 $\cat{C}\times{}\cat{C}$ 中从 $\langle c, c \rangle$ 到 $\langle a, b \rangle$ 的态射 $\langle p, q \rangle$;换句话说,一个作为以下集合元素的态射:
\[(\cat{C}\times{}\cat{C})(\Delta\ c, \langle a, b \rangle)\]
如果该映射存在,我们得出结论,存在对角函子的右伴随。该函子定义了一个积。

在 Haskell 中,我们总是可以通过分别将 \code{m} 与 \code{fst} 和 \code{snd} 复合来构造 \code{factorizer} 的逆。

\begin{snip}{haskell}
p = fst . m
q = snd . m
\end{snip}
为了完成两种定义积的方式等价性的证明,我们还需要证明 hom-集之间的映射在 $a$、$b$ 和 $c$ 上是自然的。我将把这个留给专注的读者作为练习。

总结一下我们所做的:范畴积可以全局地定义为对角函子的 \newterm{右伴随}:
\[(\cat{C}\times{}\cat{C})(\Delta c, \langle a, b \rangle) \cong \cat{C}(c, a\times{}b)\]
这里,$a\times{}b$ 是我们的右伴随函子 $\mathit{Product}$ 作用于序对 $\langle a, b \rangle$ 的结果。请注意,任何从 $\cat{C}\times{}\cat{C}$ 出发的函子都是一个双函子 (bifunctor),所以 $\mathit{Product}$ 是一个双函子。在 Haskell 中,$\mathit{Product}$ 双函子简单地写为 \code{(,)}。你可以将它应用于两种类型并得到它们的积类型,例如:

\src{snippet09}

\section{从伴随看指数}

指数 (exponential) $b^a$,或函数对象 (function object) $a \Rightarrow b$,可以使用 \hyperref[function-types]{泛构造} 来定义。这个构造,如果对所有对象对都存在,可以看作是一个伴随。同样,诀窍在于关注这个陈述:

\begin{quote}
  对于任何其他具有态射 $g \Colon z\times{}a \to b$ 的对象 $z$,存在唯一的态射 $h \Colon z \to (a \Rightarrow b)$
\end{quote}
这个陈述建立了 hom-集之间的映射。

在这种情况下,我们处理的是同一范畴中的对象,所以两个伴随函子都是自函子。左(自)函子 $L$,当作用于对象 $z$ 时,产生 $z\times{}a$。它是一个对应于与某个固定的 $a$ 取积的函子。

右(自)函子 $R$,当作用于 $b$ 时,产生函数对象 $a \Rightarrow b$(或 $b^a$)。同样,$a$ 是固定的。这两个函子之间的伴随通常写为:
\[-\times{}a \dashv (-)^a\]
支撑这个伴随的 hom-集映射最好通过重绘我们在泛构造中使用的图表来看清。

\begin{figure}[H]
  \centering
  \includegraphics[width=0.4\textwidth]{images/adj-expo.jpg}
\end{figure}

\noindent
注意 $\mathit{eval}$ 态射\footnote{参见关于 \hyperref[function-types]{泛构造} 的第 9 章。} 正是这个伴随的余单元:
\[(a \Rightarrow b)\times{}a \to b\]
其中:
\[(a \Rightarrow b)\times{}a = (L \circ R) b\]
我之前提到过,泛构造定义的唯一对象是在同构意义下唯一的。这就是为什么我们有“这个”积和“这个”指数。这个性质也适用于伴随:如果一个函子有伴随,这个伴随在同构意义下是唯一的。

\section{挑战}

\begin{enumerate}
  \tightlist
  \item
        推导 $\psi$ 的自然性方块,$\psi$ 是以下两个(逆变)函子之间的变换:
        \begin{gather*}
          a \to \cat{C}(L a, b) \\
          a \to \cat{D}(a, R b)
        \end{gather*}
  \item
        从伴随的第二种定义中的 hom-集同构开始,推导余单元 $\varepsilon$。
  \item
        完成伴随两种定义等价性的证明。
  \item
        证明余积 (coproduct) 可以通过伴随来定义。从余积的分解子 (factorizer) 的定义开始。
  \item
        证明余积是对角函子的左伴随。
  \item
        在 Haskell 中定义积和函数对象之间的伴随。
\end{enumerate}