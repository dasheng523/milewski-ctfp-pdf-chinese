% !TEX root = ../../ctfp-print.tex

\lettrine[lhang=0.17]{程}{序员们围绕着} monad (幺半群) 建立了一整套神话体系。它被认为是编程中最抽象、最难的概念之一。有些人“懂了”,有些人则不然。对许多人来说,理解 monad 概念的那一刻就像一种神秘体验。Monad 抽象了如此多不同构造的本质,以至于我们在日常生活中根本找不到一个好的类比。我们只能在黑暗中摸索,就像那些盲人摸象一样,摸到大象的不同部位后得意地宣称:“这是一根绳子”、“这是一棵树干”或“这是一个墨西哥卷饼!”

让我澄清一下:围绕 monad 的所有神秘主义都是误解的结果。Monad 是一个非常简单的概念。是 monad 应用的多样性导致了混淆。

作为这篇文章研究的一部分,我查阅了 duct tape(又名 duck tape)及其应用。以下是你可以用它做的一些事情的小例子:

\begin{itemize}
  \tightlist
  \item
        密封管道
  \item
        修复阿波罗 13 号上的 CO\textsubscript{2} 洗涤器
  \item
        治疗疣
  \item
        修复苹果 iPhone 4 的掉话问题
  \item
        制作舞会礼服
  \item
        建造悬索桥
\end{itemize}

\noindent
现在想象一下,你不知道 duct tape 是什么,并试图根据这个列表弄清楚它。祝你好运!

所以我想在“monad 就像……”的陈词滥调集合中再加一项:Monad 就像 duct tape。它的应用广泛多样,但其原理非常简单:它将事物粘合在一起。更准确地说,它组合 (composes) 事物。

这部分解释了许多程序员,特别是那些有命令式 (imperative) 背景的程序员,在理解 monad 时遇到的困难。问题在于我们不习惯用函数组合 (function composition) 的方式来思考编程。这是可以理解的。我们经常给中间值命名,而不是直接将它们从一个函数传递到另一个函数。我们常常内联 (inline) 简短的胶水代码 (glue code) 片段,而不是将它们抽象成辅助函数 (helper functions)。下面是一个命令式风格的 C 语言向量长度函数实现:

\begin{snip}{cpp}
double vlen(double * v) {
    double d = 0.0;
    int n;
    for (n = 0; n < 3; ++n)
        d += v[n] * v[n];
    return sqrt(d);
}
\end{snip}
将其与(风格化的)Haskell 版本进行比较,该版本明确地使用了函数组合:

\src{snippet01}
(这里,为了让事情更神秘,我通过将其第二个参数设置为 \code{2} 来部分应用 (partially applied) 了指数运算符 \code{(\^{})}。)

我并不是说 Haskell 的 point-free 风格 (point-free style) 总是更好,只是函数组合是我们编程中所做一切的基础。尽管我们实际上是在组合函数,Haskell 还是不遗余力地提供了一种称为 \code{do} 表示法 (do notation) 的命令式风格语法,用于 monad 组合 (monadic composition)。我们稍后会看到它的用法。但首先,让我解释一下为什么我们首先需要 monad 组合。

\section{Kleisli 范畴}

我们之前通过修饰 (embellishing) 常规函数得到了 \hyperref[kleisli-categories]{writer monad} (writer monad)。具体的修饰是通过将它们的返回值与字符串,或者更一般地,与幺半群的元素配对来完成的。我们现在可以认识到这种修饰是一个函子:

\src{snippet02}
随后,我们找到了一种组合被修饰函数,或者说 Kleisli 箭头 (Kleisli arrows) 的方法,它们是形如下式的函数:

\src{snippet03}
正是在这种组合内部,我们实现了日志 (log) 的累积。

我们现在准备好给出 Kleisli 范畴 (Kleisli category) 的更一般定义了。我们从一个范畴 $\cat{C}$ 和一个自函子 (endofunctor) $m$ 开始。相应的 Kleisli 范畴 $\cat{K}$ 与 $\cat{C}$ 具有相同的对象,但其态射不同。$\cat{K}$ 中两个对象 $a$ 和 $b$ 之间的态射被实现为原始范畴 $\cat{C}$ 中的一个态射:
\[a \to m\ b\]
重要的是要记住,我们将 $\cat{K}$ 中的 Kleisli 箭头视为 $a$ 和 $b$ 之间的态射,而不是 $a$ 和 $m\ b$ 之间的。

在我们的例子中,$m$ 被特化为 \code{Writer w},其中 \code{w} 是某个固定的幺半群。

只有当我们能为 Kleisli 箭头定义合适的组合时,它们才能构成一个范畴。如果存在一种组合,它是结合的,并且每个对象都有一个恒等箭头 (identity arrow),那么函子 $m$ 就被称为一个 \newterm{monad},由此产生的范畴被称为 Kleisli 范畴。

在 Haskell 中,Kleisli 组合是用 fish 操作符 (fish operator) \code{>=>} 定义的,恒等箭头是一个称为 \code{return} 的多态函数。下面是使用 Kleisli 组合的 monad 定义:

\src{snippet04}
请记住,定义 monad 有许多等价的方式,这在 Haskell 生态系统中并非主要方式。我喜欢它的概念简单性和它提供的直觉,但还有其他定义在编程时更方便。我们稍后会讨论它们。

在这种表述中,monad 定律 (monad laws) 非常容易表达。它们在 Haskell 中无法强制执行,但可用于等式推理 (equational reasoning)。它们就是 Kleisli 范畴的标准组合定律:

\begin{snip}{haskell}
(f >=> g) >=> h = f >=> (g >=> h) -- associativity (结合律)
return >=> f = f                  -- left unit (左单位元)
f >=> return = f                  -- right unit (右单位元)
\end{snip}
这种定义也表达了 monad 的真正含义:它是一种组合被修饰函数的方法。它与副作用 (side effects) 或状态 (state) 无关。它关乎组合。正如我们稍后将看到的,被修饰的函数可用于表达各种效果或状态,但这并非 monad 的用途。Monad 是粘性的 duct tape,它将被修饰函数的一端与另一端连接起来。

回到我们的 \code{Writer} 例子:日志记录函数(\code{Writer} 函子的 Kleisli 箭头)构成一个范畴,因为 \code{Writer} 是一个 monad:

\src{snippet05}
只要 \code{w} 的幺半群定律得到满足,\code{Writer w} 的 monad 定律就得到满足(它们在 Haskell 中也无法强制执行)。

为 \code{Writer} monad 定义了一个有用的 Kleisli 箭头,称为 \code{tell}。它唯一的目的是将其参数添加到日志中:

\src{snippet06}
我们稍后将用它作为其他 monadic 函数的构建块。

\section{Fish 操作符剖析}

在为不同的 monad 实现 fish 操作符时,你很快会意识到许多代码是重复的,并且可以很容易地被提取出来。首先,两个函数的 Kleisli 组合必须返回一个函数,所以它的实现不妨从一个接受 \code{a} 类型参数的 lambda 开始:

\src{snippet07}
对于这个参数,我们唯一能做的就是将它传递给 \code{f}:

\src{snippet08}
此时,我们必须产生类型为 \code{m c} 的结果,而我们拥有的是类型为 \code{m b} 的对象和函数 \code{g :: b -> m c}。让我们定义一个为我们做这件事的函数。这个函数称为 \emph{bind} (绑定),通常写成中缀运算符 (infix operator) 的形式:

\src{snippet09}
对于每个 monad,我们可以定义 bind 而不是定义 fish 操作符。事实上,标准的 Haskell monad 定义使用 bind:

\src{snippet10}
下面是 \code{Writer} monad 的 bind 定义:

\src{snippet11}
它确实比 fish 操作符的定义更短。

利用 \code{m} 是一个函子的事实,可以进一步分解 bind。我们可以使用 \code{fmap} 将函数 \code{a -> m b} 应用于 \code{m a} 的内容。这会将 \code{a} 转换为 \code{m b}。因此,应用的结果类型是 \code{m (m b)}。这不完全是我们想要的——我们需要类型为 \code{m b} 的结果——但我们已经很接近了。我们所需要的只是一个能够折叠或扁平化 (flattens) \code{m} 的双重应用的函数。这样的函数称为 \code{join} (连接):

\src{snippet12}
使用 \code{join},我们可以将 bind 重写为:

\src{snippet13}
这就引出了定义 monad 的第三种选项:

\src{snippet14}
这里我们明确要求 \code{m} 是一个 \code{Functor}。在前两种 monad 定义中我们不必这样做。这是因为任何支持 fish 或 bind 操作符的类型构造器 \code{m} 自动就是一个函子。例如,可以用 bind 和 \code{return} 来定义 \code{fmap}:

\src{snippet15}
为了完整起见,下面是 \code{Writer} monad 的 \code{join}:

\src{snippet16}

\section{\texttt{do} 表示法}

使用 monad 编写代码的一种方法是使用 Kleisli 箭头——用 fish 操作符组合它们。这种编程模式是 point-free 风格的推广。Point-free 代码紧凑且通常相当优雅。但总的来说,它可能难以理解,近乎神秘。这就是为什么大多数程序员更喜欢给函数参数和中间值命名。

在处理 monad 时,这意味着倾向于使用 bind 操作符而不是 fish 操作符。Bind 接受一个 monadic 值 (monadic value) 并返回一个 monadic 值。程序员可以选择给这些值命名。但这几乎算不上改进。我们真正想要的是假装我们正在处理常规值,而不是封装它们的 monadic 容器 (monadic containers)。命令式代码就是这样工作的——副作用,例如更新全局日志,大多是隐藏不见的。而这正是 \code{do} 表示法在 Haskell 中所模拟的。

你可能想知道,那为什么还要用 monad 呢?如果我们想让副作用不可见,为什么不坚持使用命令式语言呢?答案是 monad 让我们能更好地控制副作用。例如,\code{Writer} monad 中的日志是在函数之间传递的,绝不会全局暴露。不可能搞乱日志或产生数据竞争 (data race)。此外,monadic 代码与程序的其余部分有明确的界限并被隔离开来。

\code{do} 表示法只是 monadic 组合的语法糖 (syntactic sugar)。表面上,它看起来很像命令式代码,但它直接转换为一系列 bind 和 lambda 表达式 (lambda expressions)。

例如,我们之前用来说明 \code{Writer} monad 中 Kleisli 箭头组合的例子。使用我们当前的定义,它可以重写为:

\src{snippet17}
这个函数将输入字符串中的所有字符转换为大写,并将其拆分为单词,同时生成其操作的日志。

在 \code{do} 表示法中,它看起来像这样:

\src{snippet18}
这里,\code{upStr} 只是一个 \code{String},尽管 \code{upCase} 产生一个 \code{Writer}:

\src{snippet19}
这是因为 \code{do} 块被编译器脱糖 (desugared) 为:

\src{snippet20}
\code{upCase} 的 monadic 结果被绑定到一个接受 \code{String} 的 lambda。正是这个字符串的名称出现在 \code{do} 块中。当读到这行时:

\src{snippet21}
我们说 \code{upStr} \emph{获取 (gets)} \code{upCase s} 的结果。

当我们内联 \code{toWords} 时,伪命令式 (pseudo-imperative) 风格更加明显。我们将其替换为对 \code{tell} 的调用,它记录字符串 \code{"toWords "},然后是对 \code{return} 的调用,其结果是使用 \code{words} 拆分字符串 \code{upStr}。注意 \code{words} 是一个作用于字符串的常规函数。

\src{snippet22}
这里,do 块中的每一行在脱糖后的代码中引入一个新的嵌套 bind:

\src{snippet23}
注意 \code{tell} 产生一个单位值 (unit value),所以它不必传递给后面的 lambda。忽略 monadic 结果的内容(但不是它的效果——这里是对日志的贡献)是很常见的,因此有一种特殊的运算符可以在这种情况下替换 bind:

\src{snippet24}
我们代码的实际脱糖看起来像这样:

\src{snippet25}
通常,\code{do} 块由一些行(或子块)组成,这些行要么使用左箭头引入新名称,这些名称随后在代码的其余部分可用,要么纯粹为了副作用而执行。Bind 操作符在代码行之间是隐式的。顺便说一下,在 Haskell 中,可以用大括号和分号替换 \code{do} 块中的格式。这为将 monad 描述为重载分号 (overloading the semicolon) 的一种方式提供了理由。

注意,在脱糖 \code{do} 表示法时,lambda 和 bind 操作符的嵌套具有根据每行结果影响 \code{do} 块其余部分执行的效果。此属性可用于引入复杂的控制结构 (control structures),例如模拟异常 (exceptions)。

有趣的是,\code{do} 表示法的等价物已经在命令式语言中找到了应用,特别是在 C++ 中。我指的是可恢复函数 (resumable functions) 或协程 (coroutines)。C++ 的 \urlref{https://bartoszmilewski.com/2014/02/26/c17-i-see-a-monad-in-your-future/}{future 构成一个 monad} 并非秘密。它是续延 monad (continuation monad) 的一个例子,我们稍后会讨论。续延 (continuations) 的问题在于它们很难组合。在 Haskell 中,我们使用 \code{do} 表示法将“我的处理器将调用你的处理器”这种意大利面条式的代码转换为看起来非常像顺序代码 (sequential code) 的东西。可恢复函数使得在 C++ 中可以进行同样的转换。同样的机制可以应用于将 \urlref{https://bartoszmilewski.com/2014/04/21/getting-lazy-with-c/}{嵌套循环的意大利面条} 转换为列表推导式 (list comprehensions) 或“生成器 (generators)”,它们本质上是列表 monad (list monad) 的 \code{do} 表示法。如果没有 monad 的统一抽象,这些问题中的每一个通常都是通过提供语言的自定义扩展 (custom extensions) 来解决的。在 Haskell 中,这一切都是通过库 (libraries) 来处理的。